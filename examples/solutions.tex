\documentclass{article}
\usepackage{zed-cm}
\usepackage{zed-maths}
\usepackage{zed-proof}
\usepackage{amsmath}
\begin{document}

\section*{Propositional logic}

\bigskip
\noindent
\textbf{Solution 1}

\medskip

(a)
\par
\vspace{11pt}
\bigskip

false (as (true $\Rightarrow$ false) $\Leftrightarrow$ false)

\bigskip

\medskip

(b)
\par
\vspace{11pt}
\bigskip

true (as (false $\Rightarrow$ false) $\Leftrightarrow$ true)

\bigskip

\medskip

(c)
\par
\vspace{11pt}
\bigskip

true (as (false $\Rightarrow$ true) $\Leftrightarrow$ true)

\bigskip

\medskip

(d)
\par
\vspace{11pt}
\bigskip

true (as (false $\Rightarrow$ false) $\Leftrightarrow$ true)

\bigskip

\bigskip

(Assuming that pigs can't fly . . . )

\bigskip

\medskip

\bigskip
\noindent
\textbf{Solution 2}

\medskip

(a)
\par
\vspace{11pt}
\begin{center}
\begin{tabular}{c c c c}
$p$ & $q$ & $p \land q$ & $( p \land q ) \Rightarrow p$ \\
\hline
t & t & t & t \\
t & f & f & t \\
f & t & f & t \\
f & f & f & t \\
\end{tabular}
\end{center}

\medskip

(b)
\par
\vspace{11pt}
\begin{center}
\begin{tabular}{c c c c c c}
$p$ & $q$ & $p \land q$ & $\lnot p$ & $\lnot p \Rightarrow ( p \land q )$ & $( \lnot p \Rightarrow ( p \land q ) ) \Leftrightarrow p$ \\
\hline
t & t & t & f & t & t \\
t & f & f & f & t & t \\
f & t & f & t & f & t \\
f & f & f & t & f & t \\
\end{tabular}
\end{center}

\medskip

(c)
\par
\vspace{11pt}
\begin{center}
\begin{tabular}{c c c c c}
$p$ & $q$ & $p \Rightarrow q$ & $p \land ( p \Rightarrow q )$ & $( p \land ( p \Rightarrow q ) ) \Rightarrow q$ \\
\hline
t & t & t & t & t \\
t & f & f & f & t \\
f & t & t & f & t \\
f & f & t & f & t \\
\end{tabular}
\end{center}

\medskip

\bigskip
\noindent
\textbf{Solution 3}

\medskip

(a)
\par
\vspace{11pt}
\vspace{-10pt}
\begin{align*}
p \Rightarrow \lnot p \\
&\Leftrightarrow \lnot p \lor \lnot p && \text{[$\Rightarrow$]} \\
&\Leftrightarrow \lnot p && \text{[idempotence]}
\end{align*}

\medskip

(b)
\par
\vspace{11pt}
\vspace{-10pt}
\begin{align*}
\lnot p \Rightarrow p \\
&\Leftrightarrow \lnot \lnot p \lor p && \text{[$\Rightarrow$]} \\
&\Leftrightarrow p \lor p && \text{[$\lnot$ $\lnot$]} \\
&\Leftrightarrow p && \text{[idempotence]}
\end{align*}

\medskip

(c)
\par
\vspace{11pt}
\vspace{-10pt}
\begin{align*}
p \Rightarrow (q \Rightarrow r) \\
&\Leftrightarrow \lnot p \lor (q \Rightarrow r) && \text{[$\Rightarrow$]} \\
&\Leftrightarrow \lnot p \lor \lnot q \lor r && \text{[$\Rightarrow$]} \\
&\Leftrightarrow \lnot p \lor \lnot q \lor r && \text{[associativity]} \\
&\Leftrightarrow \lnot (p \land q) \lor r && \text{[De Morgan]} \\
&\Leftrightarrow p \land q \Rightarrow r && \text{[$\Rightarrow$]}
\end{align*}

\medskip

(d)
\par
\vspace{11pt}
\vspace{-10pt}
\begin{align*}
q \Rightarrow (p \Rightarrow r) \\
&\Leftrightarrow \lnot q \lor (p \Rightarrow r) && \text{[$\Rightarrow$]} \\
&\Leftrightarrow \lnot q \lor \lnot p \lor r && \text{[$\Rightarrow$]} \\
&\Leftrightarrow \lnot p \lor \lnot q \lor r && \text{[associativity $\land$ commutativity]} \\
&\Leftrightarrow \lnot p \lor (q \Rightarrow r) && \text{[$\Rightarrow$]} \\
&\Leftrightarrow p \Rightarrow (q \Rightarrow r) && \text{[$\Rightarrow$]}
\end{align*}

\medskip

(e)
\par
\vspace{11pt}
\vspace{-10pt}
\begin{align*}
p \land q \Leftrightarrow p \\
&\Leftrightarrow (p \land q \Rightarrow p) \land (p \Rightarrow p \land q) && \text{[$\Leftrightarrow$]} \\
&\Leftrightarrow (\lnot (p \land q) \lor p) \land (\lnot p \lor p \land q) && \text{[$\Rightarrow$]} \\
&\Leftrightarrow (\lnot p \lor \lnot q \lor p) \land (\lnot p \lor p \land q) && \text{[De Morgan]} \\
&\Leftrightarrow (\lnot q \lor \lnot p \lor p) \land (\lnot p \lor p \land q) && \text{[associativity $\land$ comm .]} \\
&\Leftrightarrow (\lnot q \lor true) \land (\lnot p \lor p \land q) && \text{[excluded middle]} \\
&\Leftrightarrow true \land (\lnot p \lor p \land q) && \text{[$\lor$ $\land$ true]} \\
&\Leftrightarrow \lnot p \lor p \land q && \text{[$\land$ $\land$ true]} \\
&\Leftrightarrow (\lnot p \lor p) \land (\lnot p \lor q) && \text{[distribution]} \\
&\Leftrightarrow true \land (\lnot p \lor q) && \text{[excluded middle]} \\
&\Leftrightarrow \lnot p \lor q && \text{[$\land$ $\land$ true]} \\
&\Leftrightarrow p \Rightarrow q && \text{[$\Rightarrow$]}
\end{align*}

\medskip

(f)
\par
\vspace{11pt}
\vspace{-10pt}
\begin{align*}
p \lor q \Leftrightarrow p \\
&\Leftrightarrow (p \lor q \Rightarrow p) \land (p \Rightarrow p \lor q) && \text{[$\Leftrightarrow$]} \\
&\Leftrightarrow (\lnot (p \lor q) \lor p) \land (\lnot p \lor p \lor q) && \text{[$\Rightarrow$]} \\
&\Leftrightarrow (\lnot p \land \lnot q \lor p) \land (\lnot p \lor p \lor q) && \text{[De Morgan]} \\
&\Leftrightarrow (\lnot p \lor p) \land (\lnot q \lor p) \land (\lnot p \lor p \lor q) && \text{[distribution]} \\
&\Leftrightarrow true \land (\lnot q \lor p) \land (\lnot p \lor p \lor q) && \text{[excluded middle]} \\
&\Leftrightarrow (\lnot q \lor p) \land (\lnot p \lor p \lor q) && \text{[$\land$ $\land$ true]} \\
&\Leftrightarrow (\lnot q \lor p) \land (\lnot p \lor p \lor q) && \text{[associativity]} \\
&\Leftrightarrow (\lnot q \lor p) \land (true \lor q) && \text{[excluded middle]} \\
&\Leftrightarrow (\lnot q \lor p) \land true && \text{[$\lor$ $\land$ true]} \\
&\Leftrightarrow \lnot q \lor p && \text{[$\land$ $\land$ true]} \\
&\Leftrightarrow q \Rightarrow p && \text{[$\Rightarrow$]}
\end{align*}

\medskip

\bigskip
\noindent
\textbf{Solution 4}

\medskip

\bigskip

(a) (p or q) $\Leftrightarrow$ ((not p or not q) and q) is not a tautology. You might illustrate this via a truth table or via a chain of equivalences, showing that the proposition is not equivalent to true. Alternatively, you might try and find a combination of values for which the proposition is false. (In this case, the proposition is false when p and q are both true.)

\bigskip

\bigskip

(b) (p or q) $\Leftrightarrow$ ((not p and not q) or q) is not a tautology. In this case, the proposition is false when p is true and q is false.

\bigskip

\bigskip
\noindent
\textbf{Solution 5}

\medskip

\bigskip
\noindent
\textbf{Solution 6}

\medskip

(a)
\par
\vspace{11pt}
\bigskip

This is a true proposition: whatever the value of x, the expression x^2 - x + 1 denotes a natural number. If we choose y to be this natural number, we will find that p is true.

\bigskip

\medskip

(b)
\par
\vspace{11pt}
\bigskip

This is a false proposition. We cannot choose a large enough value for y such that p will hold for any value of x.

\bigskip

\medskip

(c)
\par
\vspace{11pt}
\bigskip

This is a false proposition. It is an implication whose antecedent part is true and whose consequent part is false.

\bigskip

\medskip

(d)
\par
\vspace{11pt}
\bigskip

This is a true proposition. It is an implication whose antecedent part is false and whose consequent part is true.

\bigskip

\medskip

\bigskip
\noindent
\textbf{Solution 7}

\medskip

(a)
\par
\vspace{11pt}
\bigskip

We must define a predicate p that is false for at least one value of x, and is true for at least one other value. A suitable solution would be p $\Leftrightarrow$ x > 1.

\bigskip

\medskip

(b)
\par
\vspace{11pt}
\bigskip

With the above choice of p, we require only that q is sometimes false when p is true (for else the universal quantification would hold). A suitable solution would be q $\Leftrightarrow$ x > 3.

\bigskip

\medskip

\bigskip
\noindent
\textbf{Solution 8}

\medskip

(a)
\par
\vspace{11pt}
$\forall x \colon N \bullet x \geq z$

\medskip

\section*{Equality}

\bigskip
\noindent
\textbf{Solution 9}

\medskip

(d)
\par
\vspace{11pt}
\vspace{-10pt}
\begin{align*}
\exists x \colon N \bullet x = 1 \land x > y \lor x = 2 \land x > z \\
&\Leftrightarrow \exists x \colon N \bullet x = 1 \land x > y \lor \exists x \colon N \bullet x = 2 \land x > z \\
&\Leftrightarrow 1 \in N \land 1 > y \lor \exists x \colon N \bullet x = 2 \land x > z \\
&\Leftrightarrow 1 \in N \land 1 > y \lor 2 \in N \land 2 > z \\
&\Leftrightarrow 1 > y \lor 2 > z
\end{align*}

\medskip

\bigskip
\noindent
\textbf{Solution 10}

\medskip

\bigskip
\noindent
\textbf{Solution 11}

\medskip

\bigskip
\noindent
\textbf{Solution 12}

\medskip

\section*{Deductive proofs}

\bigskip
\noindent
\textbf{Solution 13}

\medskip

\noindent
\[
\infer[$\Rightarrow$\text{-intro}^{[1]}]{p \land (p \Rightarrow q) \Rightarrow p \land q}{
  \ulcorner p \land (p \Rightarrow q) \urcorner^{[1]} & \infer[\text{$\land$ intro}]{p \land q}{
  \infer[$\land$\text{-elim}^{[1]}]{p}{
  p \land (p \Rightarrow q)
} & \infer[\text{$\Rightarrow$ elim}]{q}{
  \infer[$\land$\text{-elim}^{[1]}]{p \Rightarrow q}{
  p \land (p \Rightarrow q)
} & \infer[$\land$\text{-elim}^{[1]}]{p}{
  p \land (p \Rightarrow q)
}
}
}
}
\]

\bigskip
\noindent
\textbf{Solution 14}

\medskip

\bigskip

In one direction:

\bigskip

\noindent
\[
\infer[$\Rightarrow$\text{-intro}^{[1]}]{(p \land q \Leftrightarrow p) \Rightarrow (p \Rightarrow q)}{
  \ulcorner p \land q \Leftrightarrow p \urcorner^{[1]} & \infer[$\Rightarrow$\text{-intro}^{[2]}]{p \Rightarrow q}{
  \ulcorner p \urcorner^{[2]} & \infer[$\land$\text{-elim}^{[3]}]{q}{
  \infer[\text{$\Rightarrow$ elim from 1 $\land$ 2}]{p \land q}{
  \infer[derived]{p \land q}{}
}
}
}
}
\]

\bigskip

and the other:

\bigskip

\noindent
\[
\infer[$\Rightarrow$\text{-intro}^{[1]}]{(p \Rightarrow q) \Rightarrow (p \land q \Leftrightarrow p)}{
  \ulcorner p \Rightarrow q \urcorner^{[1]} & \infer[\text{$\Leftrightarrow$ intro}]{p \land q \Leftrightarrow p}{
  \infer[$\Rightarrow$\text{-intro}^{[2]}]{p \land q \Rightarrow p}{
  \ulcorner p \land q \urcorner^{[2]} & \ulcorner p \urcorner^{[2]}
} & \infer[$\Rightarrow$\text{-intro}^{[3]}]{p \Rightarrow p \land q}{
  \ulcorner p \urcorner^{[3]} & \ulcorner p \land q \urcorner^{[1]}
}
}
}
\]

\bigskip

We can then combine these two proofs with $\Leftrightarrow$ intro.

\bigskip

\bigskip
\noindent
\textbf{Solution 15}

\medskip

\noindent
\[
\infer[$\Rightarrow$\text{-intro}^{[1]}]{(p \Rightarrow q) \land \lnot q \Rightarrow \lnot p}{
  \ulcorner (p \Rightarrow q) \land \lnot q \urcorner^{[1]} & \infer[false\text{-elim}^{[2]}]{\lnot p}{
  \ulcorner p \urcorner^{[2]} & \infer[\text{false intro}]{false}{
  \infer[\text{$\Rightarrow$ elim}]{q}{
  \ulcorner p \Rightarrow q \urcorner^{[1]} & \ulcorner p \urcorner^{[2]}
} & \ulcorner \lnot q \urcorner^{[1]}
}
}
}
\]

\bigskip
\noindent
\textbf{Solution 16}

\medskip

\bigskip

In one direction:

\bigskip

\noindent
\[
\infer[$\Rightarrow$\text{-intro}^{[1]}]{p \land (q \lor r) \Rightarrow p \land q \lor p \land r}{
  \ulcorner p \land (q \lor r) \urcorner^{[1]} & \infer[$\lor$\text{-elim}^{[2]}]{p \land q \lor p \land r}{
  \ulcorner q \lor r \urcorner^{[1]} & \raiseproof{10ex}{\infer[\text{$\land$ intro}]{p \land q}{
  \ulcorner p \urcorner^{[1]} & \infer[\text{case assumption}]{q}{}
} & \infer[\text{$\lor$ intro}]{p \land q \lor p \land r}{}} & \hskip 6em \raiseproof{26ex}{\infer[\text{$\land$ intro}]{p \land r}{
  \ulcorner p \urcorner^{[1]} & \infer[\text{case assumption}]{r}{}
} & \infer[\text{$\lor$ intro}]{p \land q \lor p \land r}{}}
}
}
\]

\bigskip

In the other:

\bigskip

\noindent
\[
\infer[$\Rightarrow$\text{-intro}^{[3]}]{p \land q \lor p \land r \Rightarrow p \land (q \lor r)}{
  \ulcorner p \land q \lor p \land r \urcorner^{[3]} & \infer[$\lor$\text{-elim}^{[4]}]{p \land (q \lor r)}{
  \ulcorner case1 \lor case2 \urcorner^{[3]} & \raiseproof{8ex}{\infer[\text{$\land$ elim}]{p}{} & \infer[\text{$\lor$ intro}]{q \lor r}{} & \infer[\text{$\land$ intro}]{p \land (q \lor r)}{}} & \hskip 6em \raiseproof{22ex}{\infer[\text{$\land$ elim}]{p}{} & \infer[\text{$\lor$ intro}]{q \lor r}{} & \infer[\text{$\land$ intro}]{p \land (q \lor r)}{}}
}
}
\]

\bigskip
\noindent
\textbf{Solution 17}

\medskip

\bigskip

In one direction:

\bigskip

\noindent
\[
\infer[$\Rightarrow$\text{-intro}^{[3]}]{p \lor q \land r \Rightarrow (p \lor q) \land (p \lor r)}{
  \ulcorner p \lor q \land r \urcorner^{[3]} & \infer[\text{$\lor$ elim $\land$ $\land$ intro}]{(p \lor q) \land (p \lor r)}{}
}
\]

\bigskip

and the other:

\bigskip

\noindent
\[
\infer[$\Rightarrow$\text{-intro}^{[1]}]{(p \lor q) \land (p \lor r) \Rightarrow p \lor q \land r}{
  \ulcorner (p \lor q) \land (p \lor r) \urcorner^{[1]} & \ulcorner p \lor q \land r \urcorner^{[2]}
}
\]

\bigskip
\noindent
\textbf{Solution 18}

\medskip

\bigskip

In one direction:

\bigskip

\noindent
\[
\infer[$\Rightarrow$\text{-intro}^{[1]}]{(p \Rightarrow q) \Rightarrow \lnot p \lor q}{
  \ulcorner p \Rightarrow q \urcorner^{[1]} & \lnot p \lor q
}
\]

\bigskip

and the other:

\bigskip

\noindent
\[
\infer[$\Rightarrow$\text{-intro}^{[3]}]{\lnot p \lor q \Rightarrow (p \Rightarrow q)}{
  \ulcorner \lnot p \lor q \urcorner^{[3]} & \infer[$\Rightarrow$\text{-intro}^{[4]}]{p \Rightarrow q}{
  \ulcorner p \urcorner^{[4]} & \ulcorner q \urcorner^{[3]}
}
}
\]

\section*{Sets and types}

\bigskip
\noindent
\textbf{Solution 19}

\medskip

\bigskip
\noindent
\textbf{Solution 20}

\medskip

\bigskip
\noindent
\textbf{Solution 21}

\medskip

\bigskip
\noindent
\textbf{Solution 22}

\medskip

\bigskip
\noindent
\textbf{Solution 23}

\medskip

(a)
\par
\vspace{11pt}
\vspace{-10pt}
\begin{align*}
x \in a \cap a \\
&\Leftrightarrow x \in a \land x \in a \\
&\Leftrightarrow x \in a
\end{align*}

\medskip

(b)
\par
\vspace{11pt}
\vspace{-10pt}
\begin{align*}
x \in a \cup a \\
&\Leftrightarrow x \in a \lor x \in a \\
&\Leftrightarrow x \in a
\end{align*}

\medskip

\bigskip
\noindent
\textbf{Solution 24}

\medskip

\bigskip
\noindent
\textbf{Solution 25}

\medskip

\bigskip
\noindent
\textbf{Solution 26}

\medskip

\section*{Relations}

\bigskip
\noindent
\textbf{Solution 27}

\medskip

\bigskip
\noindent
\textbf{Solution 28}

\medskip

\bigskip
\noindent
\textbf{Solution 29}

\medskip

\bigskip
\noindent
\textbf{Solution 30}

\medskip

\bigskip
\noindent
\textbf{Solution 31}

\medskip

\bigskip
\noindent
\textbf{Solution 32}

\medskip

\section*{Functions}

\bigskip
\noindent
\textbf{Solution 33}

\medskip

\bigskip
\noindent
\textbf{Solution 34}

\medskip

\bigskip
\noindent
\textbf{Solution 35}

\medskip

\bigskip
\noindent
\textbf{Solution 36}

\medskip

\section*{Sequences}

\bigskip
\noindent
\textbf{Solution 37}

\medskip

\bigskip
\noindent
\textbf{Solution 38}

\medskip

\bigskip
\noindent
\textbf{Solution 39}

\medskip

\section*{Modelling}

\bigskip
\noindent
\textbf{Solution 40}

\medskip

\bigskip
\noindent
\textbf{Solution 41}

\medskip

\bigskip
\noindent
\textbf{Solution 42}

\medskip

\bigskip
\noindent
\textbf{Solution 43}

\medskip

\section*{Free types and induction}

\bigskip
\noindent
\textbf{Solution 44}

\medskip

\bigskip
\noindent
\textbf{Solution 45}

\medskip

\bigskip
\noindent
\textbf{Solution 46}

\medskip

\bigskip
\noindent
\textbf{Solution 47}

\medskip

\section*{Supplementary material : assignment practice}

\bigskip
\noindent
\textbf{Solution 48}

\medskip

\bigskip
\noindent
\textbf{Solution 49}

\medskip

\bigskip
\noindent
\textbf{Solution 50}

\medskip

\bigskip
\noindent
\textbf{Solution 51}

\medskip

\bigskip
\noindent
\textbf{Solution 52}

\medskip

\end{document}
\documentclass{article}
\usepackage{zed-cm}
\usepackage{zed-maths}
\usepackage{zed-proof}
\usepackage{amsmath}
\begin{document}

\section*{Propositional logic}

\bigskip
\noindent
\textbf{Solution 1}

\medskip

(a)
\par
\vspace{11pt}
\bigskip

false (as (true $\Rightarrow$ false) $\Leftrightarrow$ false)

\bigskip

\medskip

(b)
\par
\vspace{11pt}
\bigskip

true (as (false $\Rightarrow$ false) $\Leftrightarrow$ true)

\bigskip

\medskip

(c)
\par
\vspace{11pt}
\bigskip

true (as (false $\Rightarrow$ true) $\Leftrightarrow$ true)

\bigskip

\medskip

(d)
\par
\vspace{11pt}
\bigskip

true (as (false $\Rightarrow$ false) $\Leftrightarrow$ true)

\bigskip

\bigskip

(Assuming that pigs can't fly . . . )

\bigskip

\medskip

\bigskip
\noindent
\textbf{Solution 2}

\medskip

(a)
\par
\vspace{11pt}
\begin{center}
\begin{tabular}{c c c c}
$p$ & $q$ & $p \land q$ & $( p \land q ) \Rightarrow p$ \\
\hline
t & t & t & t \\
t & f & f & t \\
f & t & f & t \\
f & f & f & t \\
\end{tabular}
\end{center}

\medskip

(b)
\par
\vspace{11pt}
\begin{center}
\begin{tabular}{c c c c c c}
$p$ & $q$ & $p \land q$ & $\lnot p$ & $\lnot p \Rightarrow ( p \land q )$ & $( \lnot p \Rightarrow ( p \land q ) ) \Leftrightarrow p$ \\
\hline
t & t & t & f & t & t \\
t & f & f & f & t & t \\
f & t & f & t & f & t \\
f & f & f & t & f & t \\
\end{tabular}
\end{center}

\medskip

(c)
\par
\vspace{11pt}
\begin{center}
\begin{tabular}{c c c c c}
$p$ & $q$ & $p \Rightarrow q$ & $p \land ( p \Rightarrow q )$ & $( p \land ( p \Rightarrow q ) ) \Rightarrow q$ \\
\hline
t & t & t & t & t \\
t & f & f & f & t \\
f & t & t & f & t \\
f & f & t & f & t \\
\end{tabular}
\end{center}

\medskip

\bigskip
\noindent
\textbf{Solution 3}

\medskip

(a)
\par
\vspace{11pt}
\vspace{-10pt}
\begin{align*}
p \Rightarrow \lnot p \\
&\Leftrightarrow \lnot p \lor \lnot p && \text{[$\Rightarrow$]} \\
&\Leftrightarrow \lnot p && \text{[idempotence]}
\end{align*}

\medskip

(b)
\par
\vspace{11pt}
\vspace{-10pt}
\begin{align*}
\lnot p \Rightarrow p \\
&\Leftrightarrow \lnot \lnot p \lor p && \text{[$\Rightarrow$]} \\
&\Leftrightarrow p \lor p && \text{[$\lnot$ $\lnot$]} \\
&\Leftrightarrow p && \text{[idempotence]}
\end{align*}

\medskip

(c)
\par
\vspace{11pt}
\vspace{-10pt}
\begin{align*}
p \Rightarrow (q \Rightarrow r) \\
&\Leftrightarrow \lnot p \lor (q \Rightarrow r) && \text{[$\Rightarrow$]} \\
&\Leftrightarrow \lnot p \lor \lnot q \lor r && \text{[$\Rightarrow$]} \\
&\Leftrightarrow \lnot p \lor \lnot q \lor r && \text{[associativity]} \\
&\Leftrightarrow \lnot (p \land q) \lor r && \text{[De Morgan]} \\
&\Leftrightarrow p \land q \Rightarrow r && \text{[$\Rightarrow$]}
\end{align*}

\medskip

(d)
\par
\vspace{11pt}
\vspace{-10pt}
\begin{align*}
q \Rightarrow (p \Rightarrow r) \\
&\Leftrightarrow \lnot q \lor (p \Rightarrow r) && \text{[$\Rightarrow$]} \\
&\Leftrightarrow \lnot q \lor \lnot p \lor r && \text{[$\Rightarrow$]} \\
&\Leftrightarrow \lnot p \lor \lnot q \lor r && \text{[associativity $\land$ commutativity]} \\
&\Leftrightarrow \lnot p \lor (q \Rightarrow r) && \text{[$\Rightarrow$]} \\
&\Leftrightarrow p \Rightarrow (q \Rightarrow r) && \text{[$\Rightarrow$]}
\end{align*}

\medskip

(e)
\par
\vspace{11pt}
\vspace{-10pt}
\begin{align*}
p \land q \Leftrightarrow p \\
&\Leftrightarrow (p \land q \Rightarrow p) \land (p \Rightarrow p \land q) && \text{[$\Leftrightarrow$]} \\
&\Leftrightarrow (\lnot (p \land q) \lor p) \land (\lnot p \lor p \land q) && \text{[$\Rightarrow$]} \\
&\Leftrightarrow (\lnot p \lor \lnot q \lor p) \land (\lnot p \lor p \land q) && \text{[De Morgan]} \\
&\Leftrightarrow (\lnot q \lor \lnot p \lor p) \land (\lnot p \lor p \land q) && \text{[associativity $\land$ comm .]} \\
&\Leftrightarrow (\lnot q \lor true) \land (\lnot p \lor p \land q) && \text{[excluded middle]} \\
&\Leftrightarrow true \land (\lnot p \lor p \land q) && \text{[$\lor$ $\land$ true]} \\
&\Leftrightarrow \lnot p \lor p \land q && \text{[$\land$ $\land$ true]} \\
&\Leftrightarrow (\lnot p \lor p) \land (\lnot p \lor q) && \text{[distribution]} \\
&\Leftrightarrow true \land (\lnot p \lor q) && \text{[excluded middle]} \\
&\Leftrightarrow \lnot p \lor q && \text{[$\land$ $\land$ true]} \\
&\Leftrightarrow p \Rightarrow q && \text{[$\Rightarrow$]}
\end{align*}

\medskip

(f)
\par
\vspace{11pt}
\vspace{-10pt}
\begin{align*}
p \lor q \Leftrightarrow p \\
&\Leftrightarrow (p \lor q \Rightarrow p) \land (p \Rightarrow p \lor q) && \text{[$\Leftrightarrow$]} \\
&\Leftrightarrow (\lnot (p \lor q) \lor p) \land (\lnot p \lor p \lor q) && \text{[$\Rightarrow$]} \\
&\Leftrightarrow (\lnot p \land \lnot q \lor p) \land (\lnot p \lor p \lor q) && \text{[De Morgan]} \\
&\Leftrightarrow (\lnot p \lor p) \land (\lnot q \lor p) \land (\lnot p \lor p \lor q) && \text{[distribution]} \\
&\Leftrightarrow true \land (\lnot q \lor p) \land (\lnot p \lor p \lor q) && \text{[excluded middle]} \\
&\Leftrightarrow (\lnot q \lor p) \land (\lnot p \lor p \lor q) && \text{[$\land$ $\land$ true]} \\
&\Leftrightarrow (\lnot q \lor p) \land (\lnot p \lor p \lor q) && \text{[associativity]} \\
&\Leftrightarrow (\lnot q \lor p) \land (true \lor q) && \text{[excluded middle]} \\
&\Leftrightarrow (\lnot q \lor p) \land true && \text{[$\lor$ $\land$ true]} \\
&\Leftrightarrow \lnot q \lor p && \text{[$\land$ $\land$ true]} \\
&\Leftrightarrow q \Rightarrow p && \text{[$\Rightarrow$]}
\end{align*}

\medskip

\bigskip
\noindent
\textbf{Solution 4}

\medskip

\bigskip

(a) (p or q) $\Leftrightarrow$ ((not p or not q) and q) is not a tautology. You might illustrate this via a truth table or via a chain of equivalences, showing that the proposition is not equivalent to true. Alternatively, you might try and find a combination of values for which the proposition is false. (In this case, the proposition is false when p and q are both true.)

\bigskip

\bigskip

(b) (p or q) $\Leftrightarrow$ ((not p and not q) or q) is not a tautology. In this case, the proposition is false when p is true and q is false.

\bigskip

\bigskip
\noindent
\textbf{Solution 5}

\medskip

(a)
\par
\vspace{11pt}
$\exists d \colon Dog \bullet gentle(d) \land well_{trained(d)}$

\medskip

(b)
\par
\vspace{11pt}
$\forall d \colon Dog \bullet neat(d) \land well_{trained(d)} \Rightarrow attractive(d)$

\medskip

(c)
\par
\vspace{11pt}
\bigskip

(Requires nested quantifier in implication - parser limitation)

\bigskip

\medskip

\bigskip
\noindent
\textbf{Solution 6}

\medskip

(a)
\par
\vspace{11pt}
\bigskip

This is a true proposition: whatever the value of x, the expression x^2 - x + 1 denotes a natural number. If we choose y to be this natural number, we will find that p is true.

\bigskip

\medskip

(b)
\par
\vspace{11pt}
\bigskip

This is a false proposition. We cannot choose a large enough value for y such that p will hold for any value of x.

\bigskip

\medskip

(c)
\par
\vspace{11pt}
\bigskip

This is a false proposition. It is an implication whose antecedent part is true and whose consequent part is false.

\bigskip

\medskip

(d)
\par
\vspace{11pt}
\bigskip

This is a true proposition. It is an implication whose antecedent part is false and whose consequent part is true.

\bigskip

\medskip

\bigskip
\noindent
\textbf{Solution 7}

\medskip

(a)
\par
\vspace{11pt}
\bigskip

We must define a predicate p that is false for at least one value of x, and is true for at least one other value. A suitable solution would be p $\Leftrightarrow$ x > 1.

\bigskip

\medskip

(b)
\par
\vspace{11pt}
\bigskip

With the above choice of p, we require only that q is sometimes false when p is true (for else the universal quantification would hold). A suitable solution would be q $\Leftrightarrow$ x > 3.

\bigskip

\medskip

\bigskip
\noindent
\textbf{Solution 8}

\medskip

(a)
\par
\vspace{11pt}
$\forall x \colon N \bullet x \geq z$

\medskip

\section*{Equality}

\bigskip
\noindent
\textbf{Solution 9}

\medskip

(d)
\par
\vspace{11pt}
\vspace{-10pt}
\begin{align*}
\exists x \colon N \bullet x = 1 \land x > y \lor x = 2 \land x > z \\
&\Leftrightarrow \exists x \colon N \bullet x = 1 \land x > y \lor \exists x \colon N \bullet x = 2 \land x > z \\
&\Leftrightarrow 1 \in N \land 1 > y \lor \exists x \colon N \bullet x = 2 \land x > z \\
&\Leftrightarrow 1 \in N \land 1 > y \lor 2 \in N \land 2 > z \\
&\Leftrightarrow 1 > y \lor 2 > z
\end{align*}

\medskip

\bigskip
\noindent
\textbf{Solution 10}

\medskip

\bigskip

As discussed, the quantifier exists1 can help give rise to a 'test' or 'precondition' to ensure that an application of mu will work.

\bigskip

\bigskip

So, as a simple example, as the proposition

\bigskip

$\exists_1 n \colon N \bullet \forall m \colon N \bullet n \leq m$

\bigskip

is equivalent to true, we can be certain that the statement

\bigskip

$\mu n \colon N \bullet \forall m \colon N \bullet n \leq m$

\bigskip

will return a result (which happens to be 0).

\bigskip

\bigskip
\noindent
\textbf{Solution 11}

\medskip

(a)
\par
\vspace{11pt}
\bigskip

($\mu a \colon N \bullet a = a$) = 0 is a provable statement, since 0 is the only natural number with the specified property.

\bigskip

\medskip

(b)
\par
\vspace{11pt}
\bigskip

($\mu b \colon N \bullet b = b$) = 1 is not provable. The specified property is true of both 0 and 1, and thus the value of the mu-expression is undefined.

\bigskip

\medskip

(c)
\par
\vspace{11pt}
\bigskip

($\mu c \colon N \bullet c > c$) = ($\mu c \colon N \bullet c > c$) is a provable statement. Neither expression is properly defined, but we may conclude that they are equal; there is little else that we can prove about them.

\bigskip

\medskip

(d)
\par
\vspace{11pt}
\bigskip

($\mu d \colon N \bullet d = d$) = 1 is not a provable statement. We cannot confirm that 1 is the only natural number with the specified property; we do not know what value is taken by undefined operations.

\bigskip

\medskip

\bigskip
\noindent
\textbf{Solution 12}

\medskip

\bigskip

(Requires mu-operator with expression part - not yet implemented)

\bigskip

(a)
\par
\vspace{11pt}
\bigskip

(mu m : Mountain | ($\forall n \colon Mountain \bullet height(n) \leq height(m)$) . height(m))

\bigskip

\medskip

(b)
\par
\vspace{11pt}
\bigskip

(mu c : Chapter | ($\exists_1 d \colon Chapter \bullet length(d) > length(c)$) . length(c))

\bigskip

\medskip

(c)
\par
\vspace{11pt}
\bigskip

Assuming the existence of a suitable function, max: ($\mu n \colon N \bullet n = max$($\{ m \colon N \mid 8 * m < 100 \bullet 8 * m \}$) . 100 - n)

\bigskip

\medskip

\section*{Deductive proofs}

\bigskip
\noindent
\textbf{Solution 13}

\medskip

\noindent
\[
\infer[$\Rightarrow$\text{-intro}^{[1]}]{p \land (p \Rightarrow q) \Rightarrow p \land q}{
  \ulcorner p \land (p \Rightarrow q) \urcorner^{[1]} & \infer[\text{$\land$ intro}]{p \land q}{
  \infer[$\land$\text{-elim}^{[1]}]{p}{
  p \land (p \Rightarrow q)
} & \infer[\text{$\Rightarrow$ elim}]{q}{
  \infer[$\land$\text{-elim}^{[1]}]{p \Rightarrow q}{
  p \land (p \Rightarrow q)
} & \infer[$\land$\text{-elim}^{[1]}]{p}{
  p \land (p \Rightarrow q)
}
}
}
}
\]

\bigskip
\noindent
\textbf{Solution 14}

\medskip

\bigskip

In one direction:

\bigskip

\noindent
\[
\infer[$\Rightarrow$\text{-intro}^{[1]}]{(p \land q \Leftrightarrow p) \Rightarrow (p \Rightarrow q)}{
  \ulcorner p \land q \Leftrightarrow p \urcorner^{[1]} & \infer[$\Rightarrow$\text{-intro}^{[2]}]{p \Rightarrow q}{
  \ulcorner p \urcorner^{[2]} & \infer[$\land$\text{-elim}^{[3]}]{q}{
  \infer[\text{$\Rightarrow$ elim from 1 $\land$ 2}]{p \land q}{
  \infer[derived]{p \land q}{}
}
}
}
}
\]

\bigskip

and the other:

\bigskip

\noindent
\[
\infer[$\Rightarrow$\text{-intro}^{[1]}]{(p \Rightarrow q) \Rightarrow (p \land q \Leftrightarrow p)}{
  \ulcorner p \Rightarrow q \urcorner^{[1]} & \infer[\text{$\Leftrightarrow$ intro}]{p \land q \Leftrightarrow p}{
  \infer[$\Rightarrow$\text{-intro}^{[2]}]{p \land q \Rightarrow p}{
  \ulcorner p \land q \urcorner^{[2]} & \ulcorner p \urcorner^{[2]}
} & \infer[$\Rightarrow$\text{-intro}^{[3]}]{p \Rightarrow p \land q}{
  \ulcorner p \urcorner^{[3]} & \ulcorner p \land q \urcorner^{[1]}
}
}
}
\]

\bigskip

We can then combine these two proofs with $\Leftrightarrow$ intro.

\bigskip

\bigskip
\noindent
\textbf{Solution 15}

\medskip

\noindent
\[
\infer[$\Rightarrow$\text{-intro}^{[1]}]{(p \Rightarrow q) \land \lnot q \Rightarrow \lnot p}{
  \ulcorner (p \Rightarrow q) \land \lnot q \urcorner^{[1]} & \infer[false\text{-elim}^{[2]}]{\lnot p}{
  \ulcorner p \urcorner^{[2]} & \infer[\text{false intro}]{false}{
  \infer[\text{$\Rightarrow$ elim}]{q}{
  \ulcorner p \Rightarrow q \urcorner^{[1]} & \ulcorner p \urcorner^{[2]}
} & \ulcorner \lnot q \urcorner^{[1]}
}
}
}
\]

\bigskip
\noindent
\textbf{Solution 16}

\medskip

\bigskip

In one direction:

\bigskip

\noindent
\[
\infer[$\Rightarrow$\text{-intro}^{[1]}]{p \land (q \lor r) \Rightarrow p \land q \lor p \land r}{
  \ulcorner p \land (q \lor r) \urcorner^{[1]} & \infer[$\lor$\text{-elim}^{[2]}]{p \land q \lor p \land r}{
  \ulcorner q \lor r \urcorner^{[1]} & \raiseproof{10ex}{\infer[\text{$\land$ intro}]{p \land q}{
  \ulcorner p \urcorner^{[1]} & \infer[\text{case assumption}]{q}{}
} & \infer[\text{$\lor$ intro}]{p \land q \lor p \land r}{}} & \hskip 6em \raiseproof{26ex}{\infer[\text{$\land$ intro}]{p \land r}{
  \ulcorner p \urcorner^{[1]} & \infer[\text{case assumption}]{r}{}
} & \infer[\text{$\lor$ intro}]{p \land q \lor p \land r}{}}
}
}
\]

\bigskip

In the other:

\bigskip

\noindent
\[
\infer[$\Rightarrow$\text{-intro}^{[3]}]{p \land q \lor p \land r \Rightarrow p \land (q \lor r)}{
  \ulcorner p \land q \lor p \land r \urcorner^{[3]} & \infer[$\lor$\text{-elim}^{[4]}]{p \land (q \lor r)}{
  \ulcorner case1 \lor case2 \urcorner^{[3]} & \raiseproof{8ex}{\infer[\text{$\land$ elim}]{p}{} & \infer[\text{$\lor$ intro}]{q \lor r}{} & \infer[\text{$\land$ intro}]{p \land (q \lor r)}{}} & \hskip 6em \raiseproof{22ex}{\infer[\text{$\land$ elim}]{p}{} & \infer[\text{$\lor$ intro}]{q \lor r}{} & \infer[\text{$\land$ intro}]{p \land (q \lor r)}{}}
}
}
\]

\bigskip
\noindent
\textbf{Solution 17}

\medskip

\bigskip

In one direction:

\bigskip

\noindent
\[
\infer[$\Rightarrow$\text{-intro}^{[3]}]{p \lor q \land r \Rightarrow (p \lor q) \land (p \lor r)}{
  \ulcorner p \lor q \land r \urcorner^{[3]} & \infer[\text{$\lor$ elim $\land$ $\land$ intro}]{(p \lor q) \land (p \lor r)}{}
}
\]

\bigskip

and the other:

\bigskip

\noindent
\[
\infer[$\Rightarrow$\text{-intro}^{[1]}]{(p \lor q) \land (p \lor r) \Rightarrow p \lor q \land r}{
  \ulcorner (p \lor q) \land (p \lor r) \urcorner^{[1]} & \ulcorner p \lor q \land r \urcorner^{[2]}
}
\]

\bigskip
\noindent
\textbf{Solution 18}

\medskip

\bigskip

In one direction:

\bigskip

\noindent
\[
\infer[$\Rightarrow$\text{-intro}^{[1]}]{(p \Rightarrow q) \Rightarrow \lnot p \lor q}{
  \ulcorner p \Rightarrow q \urcorner^{[1]} & \lnot p \lor q
}
\]

\bigskip

and the other:

\bigskip

\noindent
\[
\infer[$\Rightarrow$\text{-intro}^{[3]}]{\lnot p \lor q \Rightarrow (p \Rightarrow q)}{
  \ulcorner \lnot p \lor q \urcorner^{[3]} & \infer[$\Rightarrow$\text{-intro}^{[4]}]{p \Rightarrow q}{
  \ulcorner p \urcorner^{[4]} & \ulcorner q \urcorner^{[3]}
}
}
\]

\section*{Sets and types}

\bigskip
\noindent
\textbf{Solution 19}

\medskip

(a)
\par
\vspace{11pt}
\bigskip

1 in {4, 3, 2, 1} is true.

\bigskip

\medskip

(b)
\par
\vspace{11pt}
\bigskip

{1} in {1, 2, 3, 4} is undefined.

\bigskip

\medskip

(c)
\par
\vspace{11pt}
\bigskip

{1} in {{1}, {2}, {3}, {4}} is true.

\bigskip

\medskip

(d)
\par
\vspace{11pt}
\bigskip

The empty set in {1, 2, 3, 4} is undefined.

\bigskip

\medskip

\bigskip
\noindent
\textbf{Solution 20}

\medskip

(a)
\par
\vspace{11pt}
$\{1\} \cross \{2, 3\}$

\bigskip

is the set {(1, 2), (1, 3)}

\bigskip

\medskip

(b)
\par
\vspace{11pt}
\bigskip

The empty set cross {2, 3} is the empty set

\bigskip

\medskip

(c)
\par
\vspace{11pt}
$\power~emptyset \cross \{1\}$

\bigskip

is the set {(emptyset, 1)}

\bigskip

\medskip

(d)
\par
\vspace{11pt}
\bigskip

{(1, 2)} cross {3, 4} is the set {((1, 2), 3), ((1, 2), 4)}

\bigskip

\medskip

\bigskip
\noindent
\textbf{Solution 21}

\medskip

\bigskip

There are various ways of describing these sets via set comprehensions. Examples are given below.

\bigskip

(a)
\par
\vspace{11pt}
$\{ z \colon Z \mid 0 \leq z \land z \leq 100 \}$

\medskip

(b)
\par
\vspace{11pt}
$\{ z \colon Z \mid z = 10 \}$

\medskip

(c)
\par
\vspace{11pt}
$\{ z \colon Z \mid z \bmod 2 = 0 \lor z \bmod 3 = 0 \lor z \bmod 5 = 0 \}$

\medskip

\bigskip
\noindent
\textbf{Solution 22}

\medskip

(a)
\par
\vspace{11pt}
$\{ n \colon N \mid n \leq 4 \bullet n^2 \}$

\medskip

(b)
\par
\vspace{11pt}
$\{ n \colon N \mid n \leq 4 \bullet (n, n^2) \}$

\medskip

(c)
\par
\vspace{11pt}
\bigskip

{ n : P {0, 1} } (set comprehension notation requires clarification)

\bigskip

\medskip

(d)
\par
\vspace{11pt}
\bigskip

{ n : P {0, 1} | true . (n, # n) } (alternative: map over powerset)

\bigskip

\medskip

\bigskip
\noindent
\textbf{Solution 23}

\medskip

(a)
\par
\vspace{11pt}
\vspace{-10pt}
\begin{align*}
x \in a \cap a \\
&\Leftrightarrow x \in a \land x \in a \\
&\Leftrightarrow x \in a
\end{align*}

\medskip

(b)
\par
\vspace{11pt}
\vspace{-10pt}
\begin{align*}
x \in a \cup a \\
&\Leftrightarrow x \in a \lor x \in a \\
&\Leftrightarrow x \in a
\end{align*}

\medskip

\bigskip
\noindent
\textbf{Solution 24}

\medskip

(a)
\par
\vspace{11pt}
\bigskip

The set of all pairs of integers is Z cross Z. To give it a name, we could write:

\bigskip

Pairs == Z \cross Z

\medskip

(b)
\par
\vspace{11pt}
\bigskip

The set of all integer pairs in which each element is strictly greater than zero could be defined by:

\bigskip

StrictlyPositivePairs == \{ m, n \colon Z \mid m > 0 \land n > 0 \bullet (m, n) \}

\medskip

(c)
\par
\vspace{11pt}
\bigskip

It is intuitive to use a singular noun for the name of a basic type; we define the set of all people by writing:

\bigskip

\begin{zed}[Person]\end{zed}

\medskip

(d)
\par
\vspace{11pt}
\bigskip

The set of all couples could be defined by:

\bigskip

Couples == \{ s \colon \power~Person \mid \# s = 2 \}

\medskip

\bigskip
\noindent
\textbf{Solution 25}

\medskip

\bigskip

(Requires generic set notation and Cartesian product)

\bigskip

\bigskip
\noindent
\textbf{Solution 26}

\medskip

\bigskip

(Requires generic parameters and relation type notation)

\bigskip

\section*{Relations}

\bigskip
\noindent
\textbf{Solution 27}

\medskip

(a)
\par
\vspace{11pt}
\bigskip

The power set of {(0,0), (0,1), (1,0), (1,1)} is:

\bigskip

\bigskip

{ emptyset, {(0, 0)}, {(0, 1)}, {(1, 0)}, {(1, 1)}, {(1, 0), (1, 1)}, {(0, 0), (0, 1)}, {(0, 1), (1, 1)}, {(0, 1), (1, 0)}, {(0, 0), (1, 1)}, {(0, 0), (1, 0)}, {(0, 0), (1, 0), (1, 1)}, {(0, 0), (0, 1), (1, 1)}, {(0, 0), (0, 1), (1, 0)}, {(0, 1), (1, 0), (1, 1)}, {(0, 0), (0, 1), (1, 0), (1, 1)} }

\bigskip

\medskip

(b)
\par
\vspace{11pt}
\bigskip

{ emptyset, {(0, 0)}, {(0, 1)}, {(0, 0), (0, 1)} }

\bigskip

\medskip

(c)
\par
\vspace{11pt}
\bigskip

{ emptyset }

\bigskip

\medskip

(d)
\par
\vspace{11pt}
\bigskip

{ emptyset }

\bigskip

\medskip

\bigskip
\noindent
\textbf{Solution 28}

\medskip

(a)
\par
\vspace{11pt}
\bigskip

dom R = {0, 1, 2}

\bigskip

\medskip

(b)
\par
\vspace{11pt}
\bigskip

ran R = {1, 2, 3}

\bigskip

\medskip

(c)
\par
\vspace{11pt}
\bigskip

{1, 2} <| R = {1 |-> 2, 1 |-> 3, 2 |-> 3}

\bigskip

\medskip

\bigskip
\noindent
\textbf{Solution 29}

\medskip

(a)
\par
\vspace{11pt}
$\{2 \mapsto 4, 3 \mapsto 3, 3 \mapsto 4, 4 \mapsto 2\}$

\medskip

(b)
\par
\vspace{11pt}
$\{1 \mapsto 3, 2 \mapsto 2, 2 \mapsto 3, 3 \mapsto 1\}$

\medskip

(c)
\par
\vspace{11pt}
$\{1 \mapsto 1, 2 \mapsto 2, 2 \mapsto 3, 3 \mapsto 2, 3 \mapsto 3, 4 \mapsto 4\}$

\medskip

(d)
\par
\vspace{11pt}
$\{1 \mapsto 4, 2 \mapsto 2, 2 \mapsto 3, 3 \mapsto 2, 3 \mapsto 3, 4 \mapsto 1\}$

\medskip

\bigskip
\noindent
\textbf{Solution 30}

\medskip

(a)
\par
\vspace{11pt}
parentOf == childOf^{-1}

\bigskip

This is a good example of how there are many different ways of writing the same thing. An alternative abbreviation is:

\bigskip

parentOf == \{ x, y \colon Person \mid x \mapsto y \in childOf \bullet y \mapsto x \}

\bigskip

Or, via an axiomatic definition:

\bigskip

\begin{axdef}
parentOf : Person \rel Person
\where
parentOf = childOf^{-1}
\end{axdef}

\medskip

(b)
\par
\vspace{11pt}
\bigskip

siblingOf == (childOf o9 parentOf) \ id

\bigskip

\medskip

(c)
\par
\vspace{11pt}
cousinOf == childOf \circ siblingOf \circ parentOf

\medskip

(d)
\par
\vspace{11pt}
ancestorOf == parentOf^+

\medskip

\bigskip
\noindent
\textbf{Solution 31}

\medskip

\bigskip

(Requires compound identifiers with operators - R+, R*)

\bigskip

(a)
\par
\vspace{11pt}
R == \{ a, b \colon N \mid b = a \lor b = a \}

\medskip

(b)
\par
\vspace{11pt}
S == \{ a, b \colon N \mid b = a \lor b = a \}

\medskip

(c)
\par
\vspace{11pt}
\bigskip

R+ == $\{ a, b \colon N \mid b > a \}$

\bigskip

\medskip

(d)
\par
\vspace{11pt}
\bigskip

R* == $\{ a, b \colon N \mid b \geq a \}$

\bigskip

\medskip

\bigskip
\noindent
\textbf{Solution 32}

\medskip

(a)
\par
\vspace{11pt}
\vspace{-10pt}
\begin{align*}
x \mapsto y \in A \dres B \dres R \\
&\Leftrightarrow x \in A \land x \mapsto y \in (B \dres R) \\
&\Leftrightarrow x \in A \land x \in B \land x \mapsto y \in R \\
&\Leftrightarrow x \in A \cap B \land x \mapsto y \in R \\
&\Leftrightarrow x \mapsto y \in A \cap B \dres R
\end{align*}

\medskip

(b)
\par
\vspace{11pt}
\vspace{-10pt}
\begin{align*}
x \mapsto y \in R \cup S \rres C \\
&\Leftrightarrow x \mapsto y \in R \cup S \land y \in C \\
&\Leftrightarrow (x \mapsto y \in R \lor x \mapsto y \in S) \land y \in C \\
&\Leftrightarrow x \mapsto y \in R \land y \in C \lor x \mapsto y \in S \land y \in C \\
&\Leftrightarrow x \mapsto y \in R \rres C \lor x \mapsto y \in S \rres C \\
&\Leftrightarrow x \mapsto y \in (R \rres C) \cup (S \rres C)
\end{align*}

\medskip

\section*{Functions}

\bigskip
\noindent
\textbf{Solution 33}

\medskip

\bigskip

The set of 9 functions:

\bigskip

\bigskip

{ emptyset, {(0, 0)}, {(0, 1)}, {(1, 1)}, {(1, 0)}, {(0, 0), (1, 1)}, {(0, 1), (1, 1)}, {(1, 0), (0, 0)}, {(0, 1), (1, 0)} }

\bigskip

(a)
\par
\vspace{11pt}
\bigskip

The set of total functions:

\bigskip

\bigskip

{ {(0, 0), (1, 1)}, {(0, 1), (1, 1)}, {(1, 0), (0, 0)}, {(0, 1), (1, 0)} }

\bigskip

\medskip

(b)
\par
\vspace{11pt}
\bigskip

The set of functions which are neither injective nor surjective:

\bigskip

\bigskip

{ {(0, 1), (1, 1)}, {(0, 0), (1, 0)} }

\bigskip

\medskip

(c)
\par
\vspace{11pt}
\bigskip

The set of functions which are injective but not surjective:

\bigskip

\bigskip

{ emptyset, {(0, 0)}, {(0, 1)}, {(1, 0)}, {(1, 1)} }

\bigskip

\medskip

(d)
\par
\vspace{11pt}
\bigskip

There are no functions (of this type) which are surjective but not injective.

\bigskip

\medskip

(e)
\par
\vspace{11pt}
\bigskip

The set of bijective functions:

\bigskip

\bigskip

{ {(0, 0), (1, 1)}, {(0, 1), (1, 0)} }

\bigskip

\medskip

\bigskip
\noindent
\textbf{Solution 34}

\medskip

(a)
\par
\vspace{11pt}
$\{1 \mapsto a, 2 \mapsto b, 3 \mapsto c, 4 \mapsto b\}$

\medskip

(b)
\par
\vspace{11pt}
$\{1 \mapsto c, 2 \mapsto b, 3 \mapsto c, 4 \mapsto d\}$

\medskip

(c)
\par
\vspace{11pt}
$\{1 \mapsto c, 2 \mapsto b, 3 \mapsto c, 4 \mapsto b\}$

\medskip

(d)
\par
\vspace{11pt}
$\{1 \mapsto c, 2 \mapsto b, 3 \mapsto c, 4 \mapsto b\}$

\medskip

\bigskip
\noindent
\textbf{Solution 35}

\medskip

\bigskip

(Requires power set notation P and relational image)

\bigskip

(a)
\par
\vspace{11pt}
\bigskip

axdef

\bigskip

\bigskip

children : Person -> P Person

\bigskip

\bigskip

where

\bigskip

\bigskip

children = {p : Person . p |-> parentOf(| {p} |)}

\bigskip

\bigskip

end

\bigskip

\medskip

(b)
\par
\vspace{11pt}
\bigskip

axdef

\bigskip

\bigskip

number_of_grandchildren : Person -> N

\bigskip

\bigskip

where

\bigskip

\bigskip

number_of_grandchildren = {p : Person . p |-> # (parentOf o9 parentOf)(| {p} |)}

\bigskip

\bigskip

end

\bigskip

\medskip

\bigskip
\noindent
\textbf{Solution 36}

\medskip

\bigskip

(Requires power set, function types, and ran keyword)

\bigskip

\bigskip

axdef

\bigskip

\bigskip

number_of_drivers : (Drivers <-> Cars) -> (Cars -> N)

\bigskip

\bigskip

where

\bigskip

\bigskip

forall r : Drivers <-> Cars | number_of_drivers(r) = {c : ran r . c |-> #$\{ d \colon Drivers \mid d \mapsto c \in r \}$}

\bigskip

\bigskip

end

\bigskip

\section*{Sequences}

\bigskip
\noindent
\textbf{Solution 37}

\medskip

\bigskip
\noindent
\textbf{Solution 38}

\medskip

\bigskip
\noindent
\textbf{Solution 39}

\medskip

\section*{Modelling}

\bigskip
\noindent
\textbf{Solution 40}

\medskip

\bigskip
\noindent
\textbf{Solution 41}

\medskip

\bigskip
\noindent
\textbf{Solution 42}

\medskip

\bigskip
\noindent
\textbf{Solution 43}

\medskip

\section*{Free types and induction}

\bigskip
\noindent
\textbf{Solution 44}

\medskip

\bigskip
\noindent
\textbf{Solution 45}

\medskip

\bigskip
\noindent
\textbf{Solution 46}

\medskip

\bigskip
\noindent
\textbf{Solution 47}

\medskip

\section*{Supplementary material : assignment practice}

\bigskip
\noindent
\textbf{Solution 48}

\medskip

\bigskip
\noindent
\textbf{Solution 49}

\medskip

\bigskip
\noindent
\textbf{Solution 50}

\medskip

\bigskip
\noindent
\textbf{Solution 51}

\medskip

\bigskip
\noindent
\textbf{Solution 52}

\medskip

\end{document}
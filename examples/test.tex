\documentclass[a4paper,12pt]{article}

\usepackage{zed-cm}

\usepackage{zed-float}

\usepackage{zed-maths}

\usepackage{zed-proof}

\usepackage{url}

\def\fuzz{{\large\it f\kern0.1em}{\normalsize\sc uzz}}

\begin{document}

\section{Introduction}

This is a simple document to get you up and running with \LaTeX.  See the \fuzz~manual (\url{http://spivey.oriel.ox.ac.uk/mike/fuzz/fuzzman.pdf}) for a good introduction to the symbols we use.

\section{Maths}

We drop into math-font simply: the variable, $x \in \{ 1, 2, 3 \}$.

\section{Types, etc.}

This is a basic definition:
\begin{zed}
[A]
\end{zed}

This is a free type definition:
\begin{zed}
B ::= b1 \mid b2
\end{zed}

This is an axiomatic definition:
\begin{axdef}
var_1, var_2 : B \where
var_1 \neq var_2
\end{axdef}

This is an abbreviation:
\begin{zed}
Pair == A \times B
\end{zed}

And here's a generic axiomatic definition:
\begin{gendef}[X,Y]
first : (X \times Y) \fun X \\
second : (X \times Y) \fun Y \where
\forall x : X; y : Y \spot first ~ (x, y) = x \\
\forall x : X; y : Y \spot second ~ (x, y) = y \\
\end{gendef}

\section{Proofs}

This is how to format a proof:
\begin{zed}
\begin{array}{lll}
& (1 \times 2) + 3 \\
= & 2 + 3 & [\mbox{definition of multliplication}]\\
= & 5 & [\mbox{definition of addition}]
\end{array}
\end{zed}

\section{Proof trees}

\[
\infer[\impliesI\discharge{1}]{
(p \land p) \implies p
}
{
\infer[\landE]{
p
}
{
\assume{p \land p}{1}
}
}
\]

\[
\infer[\lorE\discharge{1}]{
q \lor p
}
{
p \lor q
&
\infer[\lorI]
{
q \lor p
}
{
\assume{p}{1}
}
&
\infer[\lorI]{
q \lor p
}
{
\assume{q}{1}
}
}
\]

Here's an example from the solutions.

    In one direction:
    \[
    \infer[\impliesI\discharge{3}]{%
      (( p \land q ) \lor ( p \land r)) \implies (p \land ( q \lor r
      ))
      }{%
      \infer[\lorE\discharge{4}]{%
        p \land (q \lor r)
        }{%
        \assume{(p \land q) \lor (p \land r)}{3}
        &
        \raiseproof{6ex}{%
          \hskip -2em 
          \infer[\landI]{%
            p \land (q \lor r)
            }{%
            \infer[\landE{1}]{%
              p 
              }{%
              \assume{p \land q}{4}
              }
            &
            \infer[\lorI{1}]{%
              q \lor r
              }{%
              \infer[\landE{2}]{%
                q 
                }{%
                \assume{p \land q}{4}
                }
              }
            }
          }
        &
        \hskip 8em
        \raiseproof{18ex}{%
          \hskip -2em 
          \infer[\landI]{%
            p \land (q \lor r)
            }{%
            \infer[\landE{1}]{%
              p 
              }{%
              \assume{p \land r}{4}
              }
            &
            \infer[\lorI{2}]{%
              q \lor r
              }{%
              \infer[\landE{2}]{%
                r 
                }{%
                \assume{p \land r}{4}
                }
              }
            }
          }
        }
      }
  \]

  and the other: 
  \[
  \infer[\impliesI\discharge{1}]{%
    (p \land ( q \lor r )) \implies (( p \land q ) \lor ( p \land r
    ))
    }{%
    \infer[\lorE\discharge{2}]{%
      ( p \land q ) \lor ( p \land r)
      }{%
      \infer[\landE{2}]{%
        q \lor r
        }{%
        \assume{p \land (q \lor r)}{1}
        }
      &
      \raiseproof{8ex}{%
        \hskip -8em
        \infer[\lorI{1}]{%
          ( p \land q ) \lor ( p \land r)
          }{%
          \infer[\landI]{%
            p \land q
            }{%
            \infer[\landE{1}]{%
              p 
              }{%
              \assume{p \land (q \lor r)}{1}
              }
            &
            \assume{q}{2}
            }
          }
        }
      &
      \hskip 6em
      \raiseproof{22ex}{%
        \hskip -8em
        \infer[\lorI{2}]{%
          ( p \land q ) \lor ( p \land r)
          }{%
          \infer[\landI]{%
            p \land r
            }{%
            \infer[\landE{1}]{%
              p 
              }{%
              \assume{p \land (q \lor r)}{1}
              }
            &
            \assume{r}{2}
            }
          }
        }
      }
    }
  \]
 
\end{document}


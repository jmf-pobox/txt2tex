\documentclass[a4paper,10pt,fleqn]{article}
\usepackage[margin=1in]{geometry}
\usepackage{amssymb}
\usepackage{adjustbox}
\usepackage{natbib}
\usepackage{fuzz}
\usepackage{zed-maths}
\usepackage{zed-proof}
\usepackage[colorlinks=true,linkcolor=blue,citecolor=blue,urlcolor=blue]{hyperref}
\newdimen\savedleftskip
\begin{document}

\section*{Phase 11 . 8 : Relational Image}
\addcontentsline{toc}{section}{Phase 11 . 8 : Relational Image}

\noindent The relational image $(R \limg S \rimg)$ gives the image of
set S under relation R.

\bigskip

\section*{Example 1 : Simple relational image}
\addcontentsline{toc}{section}{Example 1 : Simple relational image}

\noindent
$(R \limg S \rimg)$

\section*{Example 2 : Relational image with set literal}
\addcontentsline{toc}{section}{Example 2 : Relational image with set literal}

\noindent
$(R \limg \{1, 2, 3\} \rimg)$

\section*{Example 3 : Relational image with singleton set}
\addcontentsline{toc}{section}{Example 3 : Relational image with singleton set}

\noindent
$(parentOf \limg \{john\} \rimg)$

\section*{Example 4 : Relational image with composition}
\addcontentsline{toc}{section}{Example 4 : Relational image with composition}

\noindent
$((R \circ S) \limg A \rimg)$

\section*{Example 5 : Relational image with inverse}
\addcontentsline{toc}{section}{Example 5 : Relational image with inverse}

\noindent
$(R^{-1} \limg S \rimg)$

\section*{Example 6 : Set comprehension with relational image (
Solution 35 pattern )}
\addcontentsline{toc}{section}{Example 6 : Set comprehension with
relational image ( Solution 35 pattern )}

\noindent
$\{ p : Person | p \in \dom parentOf @ (p, (parentOf \limg \{p\} \rimg)) \}$

\section*{Example 7 : Chained relational images}
\addcontentsline{toc}{section}{Example 7 : Chained relational images}

\noindent
$((R \limg S \rimg) \limg T \rimg)$

\end{document}

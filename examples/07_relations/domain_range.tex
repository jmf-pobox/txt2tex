\documentclass[a4paper,10pt,fleqn]{article}
\usepackage[margin=1in]{geometry}
\usepackage{amssymb}
\usepackage{adjustbox}
\usepackage{natbib}
\usepackage[colorlinks=true,linkcolor=blue,citecolor=blue,urlcolor=blue]{hyperref}
\usepackage{fuzz}
\usepackage{zed-maths}
\usepackage{zed-proof}
\newdimen\savedleftskip
\begin{document}

\section*{Phase 11 a : Function Types}

\begin{zed}
  [XDR, YDR, Z]
\end{zed}

\noindent Total functions (every element in domain has a mapping):

\bigskip

\begin{axdef}
  totalF1 : XDR \fun YDR \\
  totalF2 : \nat \fun \nat
\end{axdef}

\noindent Partial functions (some elements may not have a mapping):

\bigskip

\begin{axdef}
  partialF : XDR \pfun YDR
\end{axdef}

\noindent Injections (one-to-one mappings):

\bigskip

\begin{axdef}
  injTotal : XDR \inj YDR \\
  injPartial : XDR \pinj YDR
\end{axdef}

\noindent Surjections (onto mappings):

\bigskip

\begin{axdef}
  surjTotal : XDR \surj YDR \\
  surjPartial : XDR \psurj YDR
\end{axdef}

\noindent Bijections (one-to-one and onto):

\bigskip

\begin{axdef}
  bijection : XDR \bij YDR
\end{axdef}

\noindent Mixed operators (relations and function types work together):

\bigskip

\begin{axdef}
  relationDR : XDR \rel YDR \\
  function : XDR \fun YDR
\end{axdef}

\end{document}

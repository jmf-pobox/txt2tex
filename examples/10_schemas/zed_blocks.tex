\documentclass[a4paper,10pt,fleqn]{article}
\usepackage[margin=1in]{geometry}
\usepackage{amssymb}
\usepackage{adjustbox}
\usepackage{natbib}
\usepackage[colorlinks=true,linkcolor=blue,citecolor=blue,urlcolor=blue]{hyperref}
\usepackage{fuzz}
\usepackage{zed-maths}
\usepackage{zed-proof}
\newdimen\savedleftskip
\begin{document}

\section*{Zed Block Examples}

\section*{Example 1 : Basic Variable Declarations}
\addcontentsline{toc}{section}{Example 1 : Basic Variable Declarations}

\noindent Use axdef for variable declarations with types:

\bigskip

\begin{zed}
[Person, Department]
\end{zed}

\begin{axdef}
assignment : Person \pfun Department \\
employees : \finset Person \\
departments : \finset Department
\end{axdef}

\noindent This groups related variable declarations in a single block (given types must be declared outside).

\bigskip

\section*{Example 2 : Abbreviations}
\addcontentsline{toc}{section}{Example 2 : Abbreviations}

\begin{zed}
N\_PLUS == \{~ n : \nat | n > 0 ~\}
\\
N\_EVEN == \{~ n : \nat | n \mod 2 = 0 ~\}
\\
N\_ODD == \{~ n : \nat | n \mod 2 = 1 ~\}
\end{zed}

\noindent Abbreviations can be defined standalone (zed blocks for abbreviations may not be fully supported yet).

\bigskip

\section*{Example 3 : Type Parameters}
\addcontentsline{toc}{section}{Example 3 : Type Parameters}

\begin{zed}
[Company]
\end{zed}

\noindent Declare given types using the given keyword.

\bigskip

\section*{Example 4 : Predicates elem Zed Block}
\addcontentsline{toc}{section}{Example 4 : Predicates elem Zed Block}

\begin{axdef}
x : \nat \\
y : \nat
\where
x > 0 \\
y > 0 \\
x + y < 100
\end{axdef}

\noindent Multiple predicates can constrain variables defined in axdef.

\bigskip

\section*{Example 5 : Free Type Definitions}
\addcontentsline{toc}{section}{Example 5 : Free Type Definitions}

\begin{zed}
Color ::= red | green | blue
\\
Size ::= small | medium | large
\end{zed}

\begin{schema}{Product}
color : Color \\
size : Size \\
price : \nat
\where
price > 0
\end{schema}

\noindent Free types are defined standalone, then used in schemas.

\bigskip

\section*{Example 6 : Multiple Schemas}
\addcontentsline{toc}{section}{Example 6 : Multiple Schemas}

\begin{zed}
[Account, Balance]
\end{zed}

\begin{schema}{BankAccount}
accountNumber : \nat \\
balance : \num
\where
balance \geq 0
\end{schema}

\begin{schema}{Transaction}
from : BankAccount \\
to : BankAccount \\
amount : \nat
\where
amount > 0 \\
from.balance \geq amount
\end{schema}

\noindent Multiple schemas can be defined separately, each with their own constraints.

\bigskip

\section*{Example 7 : Schema with External Variables}
\addcontentsline{toc}{section}{Example 7 : Schema with External Variables}

\begin{zed}
[Student, Course]
\end{zed}

\begin{axdef}
enrolled : Student \rel Course \\
maxCourses : \nat \\
maxStudents : \nat
\where
maxCourses = 7 \\
maxStudents = 500
\end{axdef}

\begin{schema}{Enrollment}
students : \finset Student \\
courses : \finset Course
\where
students = \dom enrolled \\
courses = \ran enrolled \\
\# students \leq maxStudents \\
\# courses \leq maxCourses
\end{schema}

\noindent Schemas can reference variables defined in axdef blocks.

\bigskip

\section*{Example 8 : Nested Schemas}
\addcontentsline{toc}{section}{Example 8 : Nested Schemas}

\begin{zed}
[Char]
\end{zed}

\begin{schema}{Address}
street : \seq~Char \\
city : \seq~Char \\
zipCode : \nat
\where
zipCode \geq 10000 \land zipCode \leq 99999
\end{schema}

\begin{schema}{PersonData}
name : \seq~Char \\
address : Address \\
age : \nat
\where
age \geq 0 \land age \leq 150
\end{schema}

\begin{axdef}
population : \finset PersonData
\end{axdef}

\noindent Schema Address is used within schema PersonData. Variables using schemas go in axdef.

\bigskip

\section*{Example 9 : Constants and Functions}
\addcontentsline{toc}{section}{Example 9 : Constants and Functions}

\begin{zed}
PI == 3
\\
E == 2
\end{zed}

\begin{axdef}
circumference : \nat \fun \nat \\
area : \nat \fun \nat
\where
\forall r : \nat @ circumference(r) = 2 * PI * r \\
\forall r : \nat @ area(r) = PI * r * r
\end{axdef}

\noindent Mathematical constants defined as abbreviations, functions in axdef.

\bigskip

\section*{Example 10 : Permission System Example}
\addcontentsline{toc}{section}{Example 10 : Permission System Example}

\begin{zed}
[UserID, DocumentID]
\\
PermissionType ::= read | write | admin
\end{zed}

\begin{schema}{User}
userId : UserID \\
documents : \power DocumentID \\
permissions : DocumentID \pfun PermissionType
\where
\dom permissions \subseteq documents
\end{schema}

\begin{schema}{System}
users : UserID \pfun User \\
allDocuments : \finset DocumentID
\where
allDocuments = \bigcup \{~ u : \ran users @ u.documents ~\}
\end{schema}

\begin{axdef}
users : UserID \pfun User
\end{axdef}

\noindent A permission system specification using free types and schemas.

\bigskip

\section*{Example 11 : Best Practices}
\addcontentsline{toc}{section}{Example 11 : Best Practices}

\noindent Use zed blocks for:

\bigskip

\noindent 1. Type parameter declarations [X, Y]

\bigskip

\noindent 2. Abbreviations (N\_PLUS, N\_EVEN, etc.)

\bigskip

\noindent 3. Simple predicates without variable declarations

\bigskip

\noindent Use axdef for:

\bigskip

\noindent 1. Variable declarations with types

\bigskip

\noindent 2. Constraints on those variables

\bigskip

\noindent 3. Function definitions

\bigskip

\noindent Use schema for:

\bigskip

\noindent 1. State spaces with declarations

\bigskip

\noindent 2. Operations on state

\bigskip

\noindent 3. Reusable component specifications

\bigskip

\end{document}
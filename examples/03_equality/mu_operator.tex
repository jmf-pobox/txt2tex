\documentclass[a4paper,10pt,fleqn]{article}
\usepackage[margin=1in]{geometry}
\usepackage{amssymb}
\usepackage{adjustbox}
\usepackage{natbib}
\usepackage[colorlinks=true,linkcolor=blue,citecolor=blue,urlcolor=blue]{hyperref}
\usepackage{fuzz}
\usepackage{zed-maths}
\usepackage{zed-proof}
\newdimen\savedleftskip
\begin{document}

\section*{Mu Operator ( μ )}

\section*{Example 1 : Basic Mu Expression}
\addcontentsline{toc}{section}{Example 1 : Basic Mu Expression}

\noindent The mu operator returns "the unique value that" satisfies a predicate:

\bigskip

\noindent
$(\mu x : \nat | x = 5)$

\noindent
$(\mu n : \nat | \forall m : \nat @ n \leq m)$

\section*{Example 2 : Mu with Precondition}
\addcontentsline{toc}{section}{Example 2 : Mu with Precondition}

\noindent The $\exists_1$ quantifier provides a precondition to
ensure mu is well-defined:

\bigskip

\noindent Since the proposition

\bigskip

\noindent
$\exists_1 n : \nat @ \forall m : \nat @ n \leq m$

\noindent is equivalent to true, we can be certain that

\bigskip

\noindent
$(\mu n : \nat | \forall m : \nat @ n \leq m)$

\noindent will return a result (which is 0).

\bigskip

\section*{Example 3 : Mu elem Expressions}
\addcontentsline{toc}{section}{Example 3 : Mu elem Expressions}

\noindent Using mu to select unique values:

\bigskip

\noindent
$(\mu a : \nat | a = a) = 0$

\noindent
$(\mu z : \num | z = 10) = 10$

\section*{Example 4 : Undefined Mu Expressions}
\addcontentsline{toc}{section}{Example 4 : Undefined Mu Expressions}

\noindent When no unique value $\exists$, mu is undefined:

\bigskip

\noindent
$(\mu b : \nat | b = b)$

\noindent This is undefined because the property holds for all
natural numbers, not just one.

\bigskip

\noindent
$(\mu c : \nat | c > c)$

\noindent This is also undefined because no natural number satisfies
this property.

\bigskip

\end{document}

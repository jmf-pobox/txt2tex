\documentclass[a4paper,10pt,fleqn]{article}
\usepackage[margin=1in]{geometry}
\usepackage{amssymb}
\usepackage{zed-cm}
\usepackage{zed-maths}
\usepackage{zed-proof}
\begin{document}

\section*{Mu Operator ( μ )}

\bigskip
\noindent
\textbf{Example 1 : Basic Mu Expression}

\medskip

\bigskip

The mu operator returns "the unique value that" satisfies a predicate:

\bigskip

\noindent
$\mu x \colon \mathbb{N} \bullet x = 5$


\noindent
$\mu n \colon \mathbb{N} \bullet \forall m \colon \mathbb{N} \bullet n \leq m$


\bigskip
\noindent
\textbf{Example 2 : Mu with Precondition}

\medskip

\bigskip

The $\exists_1$ quantifier provides a precondition to ensure mu is well-defined:

\bigskip

\bigskip

Since the proposition

\bigskip

\noindent
$\exists_1 n \colon \mathbb{N} \bullet \forall m \colon \mathbb{N} \bullet n \leq m$


\bigskip

is equivalent to true, we can be certain that

\bigskip

\noindent
$\mu n \colon \mathbb{N} \bullet \forall m \colon \mathbb{N} \bullet n \leq m$


\bigskip

will return a result (which is 0).

\bigskip

\bigskip
\noindent
\textbf{Example 3 : Mu in Expressions}

\medskip

\bigskip

Using mu to select unique values:

\bigskip

\noindent
$\mu a \colon \mathbb{N} \bullet a = a = 0$


\noindent
$\mu z \colon \mathbb{Z} \bullet z = 10 = 10$


\bigskip
\noindent
\textbf{Example 4 : Undefined Mu Expressions}

\medskip

\bigskip

When no unique value exists, mu is undefined:

\bigskip

\noindent
$\mu b \colon \mathbb{N} \bullet b = b$


\bigskip

This is undefined because the property holds for all natural numbers, not just one.

\bigskip

\noindent
$\mu c \colon \mathbb{N} \bullet c > c$


\bigskip

This is also undefined because no natural number satisfies this property.

\bigskip

\end{document}
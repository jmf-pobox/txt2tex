\documentclass[a4paper,10pt,fleqn]{article}
\usepackage[margin=1in]{geometry}
\usepackage{amssymb}
\usepackage{zed-cm}
\usepackage{zed-maths}
\usepackage{zed-proof}
\begin{document}

\section*{Unique Existence}

\bigskip
\noindent
\textbf{Example 1 : Unique Quantifier ( exists1 )}

\medskip

\bigskip

The unique existence quantifier asserts there is exactly one value satisfying the predicate:

\bigskip

\noindent
$\exists_1 x \colon \mathbb{N} \bullet x = 5$


\noindent
$\exists_1 n \colon \mathbb{N} \bullet \forall m \colon \mathbb{N} \bullet n \leq m$


\bigskip
\noindent
\textbf{Example 2 : Multiple Conditions}

\medskip

\bigskip

Unique existence with compound predicates:

\bigskip

\noindent
$\exists_1 y \colon \mathbb{N} \bullet y \in \{0, 1\} \land y \neq 1$


\noindent
$\exists_1 z \colon \mathbb{Z} \bullet z^2 = 4 \land z > 0$


\bigskip
\noindent
\textbf{Example 3 : One - Point Rule}

\medskip

\bigskip

When a unique value is specified by equality, we can apply the one-point rule:

\bigskip

\[
\begin{array}{ll@{\hspace{2em}}l}
& \exists y \colon \mathbb{N} \bullet y \in \{0, 1\} \land y \neq 1 \land x \neq y \\
&\Leftrightarrow \exists y \colon \mathbb{N} \bullet y = 0 \land x \neq y & [\mbox{arithmetic}] \\
&\Leftrightarrow 0 \in \mathbb{N} \land x \neq 0 & [\mbox{one - point rule}] \\
&\Leftrightarrow x \neq 0
\end{array}
\]

\bigskip
\noindent
\textbf{Example 4 : Unique in Domain}

\medskip

\bigskip

Testing uniqueness within a specific domain:

\bigskip

\noindent
$\exists_1 p \colon \power~\{0, 1\} \bullet \# p = 1$


\noindent
$\exists_1 x \colon \{1, 2, 3, 4\} \bullet x > 3$


\end{document}
\documentclass[a4paper,10pt,fleqn]{article}
\usepackage[margin=1in]{geometry}
\usepackage{amssymb}
\usepackage{adjustbox}
\usepackage{natbib}
\usepackage[colorlinks=true,linkcolor=blue,citecolor=blue,urlcolor=blue]{hyperref}
\usepackage{fuzz}
\usepackage{zed-maths}
\usepackage{zed-proof}
\newdimen\savedleftskip
\begin{document}

\section*{Unique Existence}

\section*{Example 1 : Unique Quantifier ( exists1 )}
\addcontentsline{toc}{section}{Example 1 : Unique Quantifier ( exists1 )}

\noindent The unique existence quantifier asserts there is exactly
one value satisfying the predicate:

\bigskip

\noindent
$\exists_1 x : \nat @ x = 5$

\noindent
$\exists_1 n : \nat @ \forall m : \nat @ n \leq m$

\section*{Example 2 : Multiple Conditions}
\addcontentsline{toc}{section}{Example 2 : Multiple Conditions}

\noindent Unique existence with compound predicates:

\bigskip

\noindent
$\exists_1 y : \nat @ y \in \{0, 1\} \land y \neq 1$

\noindent
$\exists_1 z : \num @ z^2 = 4 \land z > 0$

\section*{Example 3 : One - Point Rule}
\addcontentsline{toc}{section}{Example 3 : One - Point Rule}

\noindent When a unique value is specified by equality, we can apply
the one-point rule:

\bigskip

\begin{center}
  \adjustbox{max width=\textwidth}{%
    $\displaystyle
    \begin{array}{l@{\hspace{2em}}r}
      \exists y : \nat @ y \in \{0, 1\} \land y \neq 1 \land x \neq y \\
      \Leftrightarrow \exists y : \nat @ y = 0 \land x \neq y &
      [\mbox{arithmetic}] \\
      \Leftrightarrow 0 \in \nat \land x \neq 0 & [\mbox{one - point rule}] \\
      \Leftrightarrow x \neq 0
    \end{array}$%
  }
\end{center}
\bigskip

\section*{Example 4 : Unique in Domain}
\addcontentsline{toc}{section}{Example 4 : Unique in Domain}

\noindent Testing uniqueness within a specific domain:

\bigskip

\noindent
$\exists_1 p : \power \{0, 1\} @ \# p = 1$

\noindent
$\exists_1 x : \{1, 2, 3, 4\} @ x > 3$

\end{document}

\documentclass[a4paper,10pt,fleqn]{article}
\usepackage[margin=1in]{geometry}
\usepackage{amssymb}
\usepackage{adjustbox}
\usepackage{natbib}
\usepackage[colorlinks=true,linkcolor=blue,citecolor=blue,urlcolor=blue]{hyperref}
\usepackage{fuzz}
\usepackage{zed-maths}
\usepackage{zed-proof}
\newdimen\savedleftskip
\begin{document}

\section*{Equality Operators}

\section*{Example 1 : Basic Equality}
\addcontentsline{toc}{section}{Example 1 : Basic Equality}

\noindent The equality operator compares values:

\bigskip

\noindent
$x = y$

\noindent
$x \neq y$

\section*{Example 2 : Equality in Predicates}
\addcontentsline{toc}{section}{Example 2 : Equality in Predicates}

\noindent Equality is commonly used in quantified predicates:

\bigskip

\noindent
$\forall x : \nat @ x = x$

\noindent
$\exists y : \num @ y = 0$

\section*{Example 3 : Equality in Set Comprehension}
\addcontentsline{toc}{section}{Example 3 : Equality in Set Comprehension}

\noindent Using equality to define sets:

\bigskip

\noindent
$\{~ z : \num | z = 10 ~\}$

\noindent
$\{~ n : \nat | n = n ~\}$

\section*{Example 4 : Equality with Expressions}
\addcontentsline{toc}{section}{Example 4 : Equality with Expressions}

\noindent Equality can compare complex expressions:

\bigskip

\noindent
$\forall x, y : \nat @ x + y = y + x$

\noindent
$\exists a, b : \num @ a * b = 0$

\end{document}

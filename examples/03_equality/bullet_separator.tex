\documentclass[a4paper,10pt,fleqn]{article}
\usepackage[margin=1in]{geometry}
\usepackage{amssymb}
\usepackage{adjustbox}
\usepackage{natbib}
\usepackage[colorlinks=true,linkcolor=blue,citecolor=blue,urlcolor=blue]{hyperref}
\usepackage{fuzz}
\usepackage{zed-maths}
\usepackage{zed-proof}
\newdimen\savedleftskip
\begin{document}

\section*{Bullet Separator Examples}

\section*{Example 1 : Set Comprehension with Bullet}
\addcontentsline{toc}{section}{Example 1 : Set Comprehension with Bullet}

\noindent The bullet separator distinguishes the constraint (filter)
from the term (selector):

\bigskip

\noindent
$\{~ x : \nat | x > 0 @ x * x ~\}$

\noindent This reads: "the set of x squared, for all x in N where x
is positive". The constraint $x > 0$ filters the domain, and x * x is
the value selected.

\bigskip

\section*{Example 2 : Set Comprehension Without Bullet}
\addcontentsline{toc}{section}{Example 2 : Set Comprehension Without Bullet}

\noindent Without the bullet, the set contains the values that
satisfy the predicate:

\bigskip

\noindent
$\{~ x : \nat | x > 0 \land x < 10 ~\}$

\noindent This is the set $\{1, 2, 3, 4, 5, 6, 7, 8, 9\}$. No
transformation is applied.

\bigskip

\section*{Example 3 : Comparing With and Without Bullet}
\addcontentsline{toc}{section}{Example 3 : Comparing With and Without Bullet}

\noindent With bullet (maps/transforms values):

\bigskip

\noindent
$\{~ n : \nat | n < 5 @ n * n ~\}$

\noindent Result: $\{0, 1, 4, 9, 16\}$ (squares of 0, 1, 2, 3, 4)

\bigskip

\noindent Without bullet (filters values):

\bigskip

\noindent
$\{~ n : \nat | n < 5 ~\}$

\noindent Result: $\{0, 1, 2, 3, 4\}$ (the values themselves)

\bigskip

\section*{Example 4 : Complex Set Comprehension}
\addcontentsline{toc}{section}{Example 4 : Complex Set Comprehension}

\noindent
$\{~ x : \num | x \mod 2 = 0 \land x \geq -10 \land x \leq 10 @ x(div)(2) ~\}$

\noindent This gives the set $\{-5, -4, -3, -2, -1, 0, 1, 2, 3, 4,
5\}$ by taking even integers from -10 to 10 and dividing by 2.

\bigskip

\section*{Example 5 : Mu Operator with Bullet}
\addcontentsline{toc}{section}{Example 5 : Mu Operator with Bullet}

\noindent The mu operator finds the unique value satisfying a condition:

\bigskip

\noindent
$(\mu x : \nat | x > 5 \land x < 7 @ x)$

\noindent This evaluates to 6, the unique natural number strictly
between 5 and 7. The bullet separates the uniqueness constraint from
the selected expression.

\bigskip

\section*{Example 6 : Mu with Complex Expression}
\addcontentsline{toc}{section}{Example 6 : Mu with Complex Expression}

\noindent
$(\mu n : \nat | n * n = 16 @ n + 1)$

\noindent This finds the unique n where n squared equals 16 (which is
$n = 4$), then evaluates n + 1, giving 5.

\bigskip

\section*{Example 7 : Nested Expressions}
\addcontentsline{toc}{section}{Example 7 : Nested Expressions}

\noindent
$\{~ x : \nat | x < 3 @ \{~ y : \nat | y < x @ (x, y) ~\} ~\}$

\noindent This creates a set of sets of tuples. For each $x < 3$, we
create the set of pairs (x, y) where $y < x$.

\bigskip

\section*{Example 8 : Quantifiers with Bullet ( Limited Support )}
\addcontentsline{toc}{section}{Example 8 : Quantifiers with Bullet (
Limited Support )}

\noindent Standard Z notation supports bullet in quantifiers to
separate constraints from body:

\bigskip

\noindent PROPOSED: $\forall x : \nat @ x > 0$. $x \geq 0$

\bigskip

\noindent This would read: "for all positive x, x is non-negative".
The constraint $x > 0$ restricts the quantification domain.

\bigskip

\noindent NOTE: Full bullet support in $\forall$/$\exists$
quantifiers is not yet implemented (see STATUS.md). Currently use
implication instead:

\bigskip

\noindent
$\forall x : \nat @ x > 0 \implies x \geq 0$

\section*{Example 9 : Partial Functions with Bullet}
\addcontentsline{toc}{section}{Example 9 : Partial Functions with Bullet}

\noindent In pattern matching and function definitions:

\bigskip

\begin{axdef}
  predecessor : \nat \pfun \nat
  \where
  \forall n : \nat | n > 0 @ predecessor(n) = n - 1
\end{axdef}

\noindent The bullet indicates predecessor is only defined when the
constraint $n > 0$ holds.

\bigskip

\section*{Example 10 : Practical Example - Data Filtering}
\addcontentsline{toc}{section}{Example 10 : Practical Example - Data Filtering}

\begin{zed}
  [Employee]
\end{zed}

\begin{axdef}
  salaries : Employee \fun \nat \\
  high\_earners : \power Employee \\
  high\_salaries : \power \nat
  \where
  high\_earners = \{~ e : Employee | salaries(e) > 100000 ~\} \\
  high\_salaries = \{~ e : Employee | salaries(e) > 100000 @ salaries(e) ~\}
\end{axdef}

\noindent high\_earners is the set of employees earning over 100K
(the employees themselves). high\_salaries is the set of salary
values over 100K (the transformed values).

\bigskip

\section*{Example 11 : Mathematical Sets}
\addcontentsline{toc}{section}{Example 11 : Mathematical Sets}

\noindent The set of perfect squares less than 100:

\bigskip

\noindent
$\{~ n : \nat | n < 10 @ n * n ~\}$

\noindent Result: $\{0, 1, 4, 9, 16, 25, 36, 49, 64, 81\}$

\bigskip

\noindent The set of even numbers:

\bigskip

\noindent
$\{~ n : \nat | n \mod 2 = 0 ~\}$

\noindent Result: {0, 2, 4, 6, 8, ...} (infinite set, no bullet needed)

\bigskip

\section*{Example 12 : Design Pattern}
\addcontentsline{toc}{section}{Example 12 : Design Pattern}

\noindent The bullet separator follows a consistent pattern across Z notation:

\bigskip

\noindent Before $bullet : constraint$ (declaration and filter)

\bigskip

\noindent After $bullet : term$ (what to compute or select)

\bigskip

\noindent This separation makes specifications clearer by explicitly
distinguishing the domain restriction from the value computation.

\bigskip

\end{document}

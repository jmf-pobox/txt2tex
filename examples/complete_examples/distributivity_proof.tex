\documentclass[a4paper,10pt,fleqn]{article}
\usepackage[margin=1in]{geometry}
\usepackage{amssymb}
\usepackage{adjustbox}
\usepackage{natbib}
\usepackage[colorlinks=true,linkcolor=blue,citecolor=blue,urlcolor=blue]{hyperref}
\usepackage{fuzz}
\usepackage{zed-maths}
\usepackage{zed-proof}
\newdimen\savedleftskip
\begin{document}

\section*{Distributivity of AND over OR}

\section*{Problem Statement}
\addcontentsline{toc}{section}{Problem Statement}

\noindent Prove that conjunction (and) distributes over disjunction (or):

\bigskip

\noindent (p and (q or r)) $\Leftrightarrow$ ((p and q) or (p and r))

\bigskip

\noindent This requires proving both directions:

\bigskip

\noindent 1. Forward direction: (p and (q or r)) $\Rightarrow$ ((p
and q) or (p and r))

\bigskip

\noindent 2. Backward direction: ((p and q) or (p and r))
$\Rightarrow$ (p and (q or r))

\bigskip

\section*{Solution : Forward Direction}
\addcontentsline{toc}{section}{Solution : Forward Direction}

\noindent First, we prove the forward implication. Starting from p and (q or r),

\bigskip

\noindent we extract p and (q or r) separately, then case-analyze on (q or r).

\bigskip

\begin{center}
  \adjustbox{max width=\textwidth}{%
    $\displaystyle
    \infer[\Rightarrow\textrm{-intro}^{[1]}]{p(and(q(or)(r)))
    \implies p(and)(q)(or(p(and)(r)))}{
      \ulcorner p(and(q(or)(r))) \urcorner^{[1]}
      &
      \infer[\lor\textrm{-elim}^{[2]}]{p(and)(q)(or(p(and)(r)))}{
        \ulcorner q(or)(r) \urcorner^{[1]}
        &
        \raiseproof{10ex}{\infer[\lor
          \mbox{intro}]{p(and)(q)(or(p(and)(r)))}{\infer[\land
            \mbox{intro}]{p(and)(q)}{
              \infer[\land\textrm{-elim}^{[1]}]{p}{}
              &
              \infer[\mbox{case assumption}]{q}{}
        }}}
        &
        \hskip 6em \raiseproof{26ex}{\infer[\lor
          \mbox{intro}]{p(and)(q)(or(p(and)(r)))}{\infer[\land
            \mbox{intro}]{p(and)(r)}{
              \infer[\land\textrm{-elim}^{[1]}]{p}{}
              &
              \infer[\mbox{case assumption}]{r}{}
        }}}
      }
    }
    $%
  }
\end{center}
\bigskip

\noindent Key $insight : We extract p once$, then use it in both
cases after splitting on (q or r).

\bigskip

\section*{Solution : Backward Direction}
\addcontentsline{toc}{section}{Solution : Backward Direction}

\noindent For the backward direction, we start with (p and q) or (p and r) and

\bigskip

\noindent case-analyze on the outer disjunction. In each case, we extract p and

\bigskip

\noindent one of q or r, then rebuild p and (q or r).

\bigskip

\begin{center}
  \adjustbox{max width=\textwidth}{%
    $\displaystyle
    \infer[\Rightarrow\textrm{-intro}^{[3]}]{p(and)(q)(or(p(and)(r)))
    \implies p(and(q(or)(r)))}{
      \ulcorner p(and)(q)(or(p(and)(r))) \urcorner^{[3]}
      &
      \infer[\lor\textrm{-elim}^{[4]}]{p(and(q(or)(r)))}{
        \ulcorner case1(or)(case2) \urcorner^{[3]}
        &
        \raiseproof{8ex}{\infer[\land
          \mbox{intro}]{p(and(q(or)(r)))}{\infer[\land
        \mbox{elim}]{p}{} & \infer[\lor \mbox{intro}]{q(or)(r)}{}}}
        &
        \hskip 6em \raiseproof{22ex}{\infer[\land
          \mbox{intro}]{p(and(q(or)(r)))}{\infer[\land
        \mbox{elim}]{p}{} & \infer[\lor \mbox{intro}]{q(or)(r)}{}}}
      }
    }
    $%
  }
\end{center}
\bigskip

\noindent Key $insight : Both cases yield$ the same conclusion p and
(q or r), since

\bigskip

\noindent we can inject q or r into (q or r) using or introduction.

\bigskip

\section*{Combining Both Directions}
\addcontentsline{toc}{section}{Combining Both Directions}

\noindent Having proven both directions:

\bigskip

\noindent - (p and (q or r)) $\Rightarrow$ ((p and q) or (p and r))

\bigskip

\noindent - ((p and q) or (p and r)) $\Rightarrow$ (p and (q or r))

\bigskip

\noindent

\bigskip

\noindent We can combine them $using \iff introduction$ to conclude:

\bigskip

\noindent (p and (q or r)) $\Leftrightarrow$ ((p and q) or (p and r))

\bigskip

\section*{Reflection}
\addcontentsline{toc}{section}{Reflection}

\noindent This proof demonstrates several important techniques:

\bigskip

\noindent

\bigskip

\noindent 1. **Bidirectional proof structure**: Equivalences require
proving both directions

\bigskip

\noindent 2. **Case analysis**: Essential for handling disjunctions (or)

\bigskip

\noindent 3. **Assumption management**: Clear labeling prevents confusion

\bigskip

\noindent 4. **Proof symmetry**: Both directions use similar patterns
but different orders

\bigskip

\noindent 5. **Combining inference rules**: Multiple rules work
together naturally

\bigskip

\noindent

\bigskip

\noindent The distributive law is fundamental in logic and appears
frequently in:

\bigskip

\noindent - Simplifying logical expressions

\bigskip

\noindent - Converting to disjunctive normal form (DNF)

\bigskip

\noindent - Optimizing boolean circuits

\bigskip

\noindent - Reasoning about program correctness

\bigskip

\section*{Exercise for the Reader}
\addcontentsline{toc}{section}{Exercise for the Reader}

\noindent Try proving the dual distributive law:

\bigskip

\noindent p or (q and r) $\Leftrightarrow$ (p or q) and (p or r)

\bigskip

\noindent

\bigskip

\noindent This proof will use similar techniques but with the roles
of and/or reversed.

\bigskip

\end{document}

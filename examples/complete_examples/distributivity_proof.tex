\documentclass[a4paper,10pt,fleqn]{article}
\usepackage[margin=1in]{geometry}
\usepackage{amssymb}
\usepackage{adjustbox}
\usepackage{natbib}
\usepackage{fuzz}
\usepackage{zed-maths}
\usepackage{zed-proof}
\usepackage[colorlinks=true,linkcolor=blue,citecolor=blue,urlcolor=blue]{hyperref}
\newdimen\savedleftskip
\begin{document}

\section*{Distributivity of AND over OR}

\section*{Problem Statement}
\addcontentsline{toc}{section}{Problem Statement}

\noindent Prove that conjunction (and) distributes over disjunction (or):

\bigskip

\noindent $(p \land (q \lor r))$ $\Leftrightarrow$ $((p \land q) \lor
(p \land r))$

\bigskip

\noindent This requires proving both directions:

\bigskip

\noindent 1. Forward direction: $(p \land (q \lor r))$ $\Rightarrow$
$((p \land q) \lor (p \land r))$

\bigskip

\noindent 2. Backward direction: $((p \land q) \lor (p \land r))$
$\Rightarrow$ $(p \land (q \lor r))$

\bigskip

\section*{Solution : Forward Direction}
\addcontentsline{toc}{section}{Solution : Forward Direction}

\noindent First, we prove the forward implication. Starting from p
and $(q \lor r)$,

\bigskip

\noindent we extract p and $(q \lor r)$ separately, then case-analyze
on $(q \lor r)$.

\bigskip

\begin{center}
  \adjustbox{max width=\textwidth}{%
    $\displaystyle
    \infer[\Rightarrow\textrm{-intro}^{[1]}]{(p \land (q \lor r))
    \implies ((p \land q) \lor (p \land r))}{
      \ulcorner p \land (q \lor r) \urcorner^{[1]}
      &
      \infer[\lor\textrm{-elim}^{[2]}]{(p \land q) \lor (p \land r)}{
        \ulcorner q \lor r \urcorner^{[1]}
        &
        \raiseproof{10ex}{\infer[\lor \mathrm{intro}]{(p \land q)
          \lor (p \land r)}{\infer[\land \mathrm{intro}]{p \land q}{
              \infer[\land\textrm{-elim}^{[1]}]{p}{}
              &
              \infer[\mathrm{case} \mathrm{assumption}]{q}{}
        }}}
        &
        \hskip 6em \raiseproof{26ex}{\infer[\lor \mathrm{intro}]{(p
          \land q) \lor (p \land r)}{\infer[\land \mathrm{intro}]{p \land r}{
              \infer[\land\textrm{-elim}^{[1]}]{p}{}
              &
              \infer[\mathrm{case} \mathrm{assumption}]{r}{}
        }}}
      }
    }
    $%
  }
\end{center}
\bigskip

\noindent Key $insight : We extract p once$, then use it in both
cases after splitting on $(q \lor r)$.

\bigskip

\section*{Solution : Backward Direction}
\addcontentsline{toc}{section}{Solution : Backward Direction}

\noindent For the backward direction, we start with $(p \land q)$ or
$(p \land r)$ and

\bigskip

\noindent case-analyze on the outer disjunction. In each case, we extract p and

\bigskip

\noindent one of q or r, then rebuild p and $(q \lor r)$.

\bigskip

\begin{center}
  \adjustbox{max width=\textwidth}{%
    $\displaystyle
    \infer[\Rightarrow\textrm{-intro}^{[3]}]{((p \land q) \lor (p
    \land r)) \implies (p \land (q \lor r))}{
      \ulcorner (p \land q) \lor (p \land r) \urcorner^{[3]}
      &
      \infer[\lor\textrm{-elim}^{[4]}]{p \land (q \lor r)}{
        \ulcorner case1 \lor case2 \urcorner^{[3]}
        &
        \raiseproof{8ex}{\infer[\land \mathrm{intro}]{p \land (q \lor
          r)}{\infer[\land \mathrm{elim}]{p}{} & \infer[\lor
        \mathrm{intro}]{q \lor r}{}}}
        &
        \hskip 6em \raiseproof{22ex}{\infer[\land \mathrm{intro}]{p
          \land (q \lor r)}{\infer[\land \mathrm{elim}]{p}{} &
        \infer[\lor \mathrm{intro}]{q \lor r}{}}}
      }
    }
    $%
  }
\end{center}
\bigskip

\noindent Key $insight : Both cases yield$ the same conclusion p and
$(q \lor r)$, since

\bigskip

\noindent we can inject q or r into $(q \lor r)$ using or introduction.

\bigskip

\section*{Combining Both Directions}
\addcontentsline{toc}{section}{Combining Both Directions}

\noindent Having proven both directions:

\bigskip

\noindent - $(p \land (q \lor r))$ $\Rightarrow$ $((p \land q) \lor
(p \land r))$

\bigskip

\noindent - $((p \land q) \lor (p \land r))$ $\Rightarrow$ $(p \land
(q \lor r))$

\bigskip

\noindent

\bigskip

\noindent We can combine them $using \iff introduction$ to conclude:

\bigskip

\noindent $(p \land (q \lor r))$ $\Leftrightarrow$ $((p \land q) \lor
(p \land r))$

\bigskip

\section*{Reflection}
\addcontentsline{toc}{section}{Reflection}

\noindent This proof demonstrates several important techniques:

\bigskip

\noindent

\bigskip

\noindent 1. **Bidirectional proof structure**: Equivalences require
proving both directions

\bigskip

\noindent 2. **Case analysis**: Essential for handling disjunctions (or)

\bigskip

\noindent 3. **Assumption management**: Clear labeling prevents confusion

\bigskip

\noindent 4. **Proof symmetry**: Both directions use similar patterns
but different orders

\bigskip

\noindent 5. **Combining inference rules**: Multiple rules work
together naturally

\bigskip

\noindent

\bigskip

\noindent The distributive law is fundamental in logic and appears
frequently in:

\bigskip

\noindent - Simplifying logical expressions

\bigskip

\noindent - Converting to disjunctive normal form (DNF)

\bigskip

\noindent - Optimizing boolean circuits

\bigskip

\noindent - Reasoning about program correctness

\bigskip

\section*{Exercise for the Reader}
\addcontentsline{toc}{section}{Exercise for the Reader}

\noindent Try proving the dual distributive law:

\bigskip

\noindent p or $(q \land r)$ $\Leftrightarrow$ $(p \lor q)$ and $(p \lor r)$

\bigskip

\noindent

\bigskip

\noindent This proof will use similar techniques but with the roles
of and/or reversed.

\bigskip

\end{document}

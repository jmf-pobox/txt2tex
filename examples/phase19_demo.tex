\documentclass{article}
\usepackage{zed-cm}
\usepackage{zed-maths}
\usepackage{zed-proof}
\usepackage{amsmath}
\begin{document}

\section*{Phase 19 Demo : Finite Set Types}

\bigskip
\noindent
\textbf{Example 1 : F ( finite sets )}

\medskip

\bigskip

F X denotes all finite subsets of X:

\bigskip

\begin{zed}[SongId, UserId]\end{zed}

\begin{axdef}
songs : \finset~SongId
users : \finset~UserId
\end{axdef}

\bigskip
\noindent
\textbf{Example 2 : F1 ( non - empty finite sets )}

\medskip

\bigskip

F1 X denotes all non-empty finite subsets of X:

\bigskip

\begin{axdef}
activeUsers : \finset_1~UserId
\where
activeUsers
subseteq
users
\end{axdef}

\bigskip
\noindent
\textbf{Example 3 : Combined with function types}

\medskip

\bigskip

Using F and F1 with partial functions:

\bigskip

\begin{zed}[PlaylistId, Playlist]\end{zed}

\begin{axdef}
playlists : PlaylistId \pfun Playlist
playlistOwner : PlaylistId \pfun UserId
playlistSubscribers : PlaylistId \pfun \finset_1~UserId
\where
\dom playlistOwner
subseteq
\dom playlists
\dom playlistSubscribers
subseteq
\dom playlists
\forall i \colon \dom playlistSubscribers \bullet playlistSubscribers(i)
subseteq
users
\end{axdef}

\bigskip
\noindent
\textbf{Example 4 : Comparison with P ( power set )}

\medskip

\bigskip

F X is a subset of P X (all finite subsets are subsets):

\bigskip

\begin{axdef}
\where
\finset~users
subseteq
\power~users
\finset_1~users
subseteq
\finset~users
\end{axdef}

\end{document}
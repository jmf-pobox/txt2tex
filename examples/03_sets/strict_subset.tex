\documentclass[a4paper,10pt,fleqn]{article}
\usepackage[margin=1in]{geometry}
\usepackage{amssymb}
\usepackage{natbib}
\usepackage{fuzz}
\usepackage{zed-maths}
\usepackage{zed-proof}
\begin{document}

\section*{Phase 39 : Strict Subset Operator}

\bigskip

The strict subset operator (psubset, ⊂) represents proper subset
relationships, where A is a subset of B AND A is not equal to B.

\bigskip

\bigskip

Distinction from subset (⊆):

\bigskip

\bigskip

- A subset B means: every element of A $is \in B$ (may be equal)

\bigskip

\bigskip

- A psubset B means: every element of A $is \in B$ AND $A \neq B$
(strict inequality)

\bigskip

\bigskip
\noindent
\textbf{Example 1 : Basic Strict Subset}

\medskip

\begin{axdef}
  A : \power \nat \\
  B : \power \nat
  \where
  A = \{1, 2, 3\} \land B = \{1, 2, 3, 4, 5\} \land A \subset B
\end{axdef}

\bigskip

Here A psubset B holds because all elements of A $are \in B$, and A
contains fewer elements than B.

\bigskip

\bigskip
\noindent
\textbf{Example 2 : Comparing subset and psubset}

\medskip

\begin{axdef}
  X : \power \nat \\
  Y : \power \nat \\
  W : \power \nat
  \where
  X = \{1, 2\} \land Y = \{1, 2, 3\} \land W = \{1, 2\} \land X
  \subset Y \land X \subseteq W
\end{axdef}

\bigskip

X psubset Y is true (X is strictly $contained \in Y$)

\bigskip

\bigskip

X subset W is true (X is a subset of W, including the case where $X = W$)

\bigskip

\bigskip

X psubset W would be FALSE (they are equal sets)

\bigskip

\bigskip
\noindent
\textbf{Example 3 : Empty Set}

\medskip

\begin{axdef}
  emptySet : \power \nat \\
  anySet : \power \nat
  \where
  emptySet = \{\} \land anySet = \{1, 2, 3\} \land emptySet \subset anySet
\end{axdef}

\bigskip

The empty set is a strict subset of any non-empty set.

\bigskip

\bigskip
\noindent
\textbf{Example 4 : Transitive Property}

\medskip

\bigskip

If A psubset B and B psubset C, then A psubset C.

\bigskip

\begin{axdef}
  A1 : \power \nat \\
  B1 : \power \nat \\
  C1 : \power \nat
  \where
  A1 = \{1\} \land B1 = \{1, 2\} \land C1 = \{1, 2, 3\} \land A1
  \subset B1 \land B1 \subset C1 \land A1 \subset C1
\end{axdef}

\bigskip
\noindent
\textbf{Example 5 : Set Hierarchy}

\medskip

\begin{zed}[Person]
\end{zed}

\begin{axdef}
  students : \power Person \\
  grads : \power Person \\
  phds : \power Person
  \where
  phds \subset grads \land grads \subset students
\end{axdef}

\bigskip

PhD students are a strict subset of graduate students, which are a
strict subset of all students.

\bigskip

\end{document}

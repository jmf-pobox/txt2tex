\documentclass[a4paper,10pt,fleqn]{article}
\usepackage[margin=1in]{geometry}
\usepackage{amssymb}
\usepackage{natbib}
\usepackage{fuzz}
\usepackage{zed-maths}
\usepackage{zed-proof}
\begin{document}

\section*{Phase 4 : Z Notation Basics}

\bigskip
\noindent
\textbf{Example 1 : Given Types}

\medskip

\begin{zed}[Person, Company]
\end{zed}

\bigskip
\noindent
\textbf{Example 2 : Free Types}

\medskip

\begin{zed}Status ::= active | inactive | suspended
\end{zed}

\begin{zed}Answer ::= yes | no
\end{zed}

\bigskip
\noindent
\textbf{Example 3 : Abbreviations}

\medskip

\begin{zed}
  MaxCapacity == 100
\end{zed}

\begin{zed}
  DefaultSize == 10
\end{zed}

\bigskip
\noindent
\textbf{Example 4 : Axiomatic Definitions}

\medskip

\begin{axdef}
  population : \nat \\
  capacity : \nat
  \where
  population \leq capacity \\
  capacity > 0
\end{axdef}

\bigskip
\noindent
\textbf{Example 5 : Schema Definitions}

\medskip

\begin{schema}{Library}
  books : \nat \\
  members : \nat
  \where
  books > 0 \\
  members \geq 0
\end{schema}

\begin{schema}{BankAccount}
  balance : \nat \\
  overdraft : \nat
  \where
  balance \geq 0 \\
  overdraft \geq 0
\end{schema}

\end{document}

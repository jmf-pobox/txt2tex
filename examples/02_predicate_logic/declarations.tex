\documentclass[a4paper,10pt,fleqn]{article}
\usepackage[margin=1in]{geometry}
\usepackage{amssymb}
\usepackage{adjustbox}
\usepackage{natbib}
\usepackage[colorlinks=true,linkcolor=blue,citecolor=blue,urlcolor=blue]{hyperref}
\usepackage{fuzz}
\usepackage{zed-maths}
\usepackage{zed-proof}
\newdimen\savedleftskip
\begin{document}

\section*{Phase 9 : Generic Parameters}

\noindent This example demonstrates Z notation definitions with
generic (polymorphic) type parameters.

\bigskip

\noindent Basic generic abbreviation for a Pair type:

\bigskip

\begin{zed}
  Pair[X] == X \cross X
\end{zed}

\noindent Generic abbreviation with two type parameters:

\bigskip

\begin{zed}
  Product[X, Y] == X \cross Y
\end{zed}

\noindent Generic axiomatic definition with constraints:

\bigskip

\begin{gendef}[T]
  identity: T
  \where
  identity = identity
\end{gendef}

\noindent Generic schema for a collection:

\bigskip

\begin{schema}{Collection}[X]
  items : \seq~X \\
  count : \nat
  \where
  count = \# items
\end{schema}

\noindent Generic schema with multiple parameters:

\bigskip

\begin{schema}{Tuple}[X, Y]
  first : X \\
  second : Y
\end{schema}

\noindent Non-generic definitions still work as before:

\bigskip

\begin{zed}
  Naturals == \nat
\end{zed}

\begin{axdef}
  zero : \nat
  \where
  zero = 0
\end{axdef}

\begin{schema}{Counter}
  value : \nat
  \where
  value \geq 0
\end{schema}

\end{document}

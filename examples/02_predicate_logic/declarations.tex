\documentclass[a4paper,10pt,fleqn]{article}
\usepackage[margin=1in]{geometry}
\usepackage{amssymb}
\usepackage{natbib}
\usepackage{fuzz}
\usepackage{zed-maths}
\usepackage{zed-proof}
\begin{document}

\section*{Phase 9 : Generic Parameters}

\bigskip

This example demonstrates Z notation definitions with generic (polymorphic) type parameters.

\bigskip

\bigskip

Basic generic abbreviation for a Pair type:

\bigskip

\begin{zed}
Pair[X] == X \cross X
\end{zed}

\bigskip

Generic abbreviation with two type parameters:

\bigskip

\begin{zed}
Product[X, Y] == X \cross Y
\end{zed}

\bigskip

Generic axiomatic definition with constraints:

\bigskip

\begin{gendef}[T]
  identity: T
\where
  identity = identity
\end{gendef}

\bigskip

Generic schema for a Stack data structure:

\bigskip

\begin{schema}{Stack}[X]
items : X \\
top : X
\where
top \in items
\end{schema}

\bigskip

Generic schema with multiple parameters:

\bigskip

\begin{schema}{Relation}[X, Y]
domain : X \\
range : Y
\where
domain \in X \\
range \in Y
\end{schema}

\bigskip

Non-generic definitions still work as before:

\bigskip

\begin{zed}
Naturals == \nat
\end{zed}

\begin{axdef}
zero : \nat
\where
zero = 0
\end{axdef}

\begin{schema}{Counter}
value : \nat
\where
value \geq 0
\end{schema}

\end{document}
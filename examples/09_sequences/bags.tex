\documentclass[a4paper,10pt,fleqn]{article}
\usepackage[margin=1in]{geometry}
\usepackage{amssymb}
\usepackage{natbib}
\usepackage{fuzz}
\usepackage{zed-maths}
\usepackage{zed-proof}
\begin{document}

\section*{Bags ( Multisets )}

\bigskip
\noindent
\textbf{Example 1 : Bag Type}

\medskip

\bigskip

Bags are unordered collections that allow duplicates:

\bigskip

\noindent
$\bag \nat$

\noindent
$\bag \num$

\bigskip
\noindent
\textbf{Example 2 : Bag Literals}

\medskip

\bigskip

Bags are written with double square brackets:

\bigskip

\noindent
$\lbag x \rbag$

\noindent
$\lbag a, b, c \rbag$

\noindent
$\lbag 1, 2, 2, 3, 3, 3 \rbag$

\bigskip
\noindent
\textbf{Example 3 : Bags vs Sets}

\medskip

\bigskip

Unlike sets, bags preserve multiplicity:

\bigskip

\bigskip

The bag [[1, 2, 2, 3]] is different from the set $\{1, 2, 3\}$

\bigskip

\bigskip

The bag [[a, a, a]] contains three copies of a

\bigskip

\bigskip
\noindent
\textbf{Example 4 : Bags in Specifications}

\medskip

\bigskip

Bags can model collections where order doesn't matter but quantity does:

\bigskip

\noindent
$coins \in \bag Coin$

\noindent
$items \in \bag Item$

\end{document}

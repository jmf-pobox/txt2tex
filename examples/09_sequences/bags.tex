\documentclass[a4paper,10pt,fleqn]{article}
\usepackage[margin=1in]{geometry}
\usepackage{amssymb}
\usepackage{adjustbox}
\usepackage{natbib}
\usepackage[colorlinks=true,linkcolor=blue,citecolor=blue,urlcolor=blue]{hyperref}
\usepackage{fuzz}
\usepackage{zed-maths}
\usepackage{zed-proof}
\newdimen\savedleftskip
\begin{document}

\section*{Bags ( Multisets )}

\section*{Example 1 : Bag Type}
\addcontentsline{toc}{section}{Example 1 : Bag Type}

\noindent Bags are unordered collections that allow duplicates:

\bigskip

\noindent
$\bag~\nat$

\noindent
$\bag~\num$

\section*{Example 2 : Bag Literals}
\addcontentsline{toc}{section}{Example 2 : Bag Literals}

\noindent Bags are written with double square brackets:

\bigskip

\noindent
$\lbag x \rbag$

\noindent
$\lbag a, b, c \rbag$

\noindent
$\lbag 1, 2, 2, 3, 3, 3 \rbag$

\section*{Example 3 : Bags vs Sets}
\addcontentsline{toc}{section}{Example 3 : Bags vs Sets}

\noindent Unlike sets, bags preserve multiplicity:

\bigskip

\noindent The bag [[1, 2, 2, 3]] is different from the set $\{1, 2, 3\}$

\bigskip

\noindent The bag [[a, a, a]] contains three copies of a

\bigskip

\section*{Example 4 : Bags in Specifications}
\addcontentsline{toc}{section}{Example 4 : Bags in Specifications}

\noindent Bags can model collections where order doesn't matter but
quantity does:

\bigskip

\noindent
$coins \in \bag~Coin$

\noindent
$items \in \bag~Item$

\end{document}

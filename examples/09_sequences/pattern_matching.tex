\documentclass[a4paper,10pt,fleqn]{article}
\usepackage[margin=1in]{geometry}
\usepackage{amssymb}
\usepackage{zed-cm}
\usepackage{zed-maths}
\usepackage{zed-proof}
\begin{document}

\section*{Pattern Matching with Sequences}

\bigskip
\noindent
\textbf{Example 1 : Empty Sequence Pattern}

\medskip

\bigskip

Pattern matching on the empty sequence:

\bigskip

\begin{axdef}
f : \seq~\mathbb{N} \fun \mathbb{N}
\where
f(\langle \rangle) = 0
\end{axdef}

\bigskip
\noindent
\textbf{Example 2 : Cons Pattern ( Head and Tail )}

\medskip

\bigskip

Decomposing a sequence into head and tail:

\bigskip

\begin{axdef}
g : \seq~\mathbb{N} \fun \mathbb{N}
\where
g(\langle \rangle) = 0
\forall x \colon \mathbb{N} \bullet \forall s \colon \seq~\mathbb{N} \bullet g(\langle x \rangle \cat s) = x
\end{axdef}

\bigskip
\noindent
\textbf{Example 3 : Recursive Sum}

\medskip

\bigskip

Computing the sum of elements using pattern matching:

\bigskip

\begin{axdef}
total : \seq~\mathbb{N} \fun \mathbb{N}
\where
total(\langle \rangle) = 0
\forall x \colon \mathbb{N} \bullet \forall s \colon \seq~\mathbb{N} \bullet total(\langle x \rangle \cat s) = x + total(s)
\end{axdef}

\bigskip
\noindent
\textbf{Example 4 : Cumulative Total}

\medskip

\bigskip

A more descriptive example from the solutions:

\bigskip

\begin{axdef}
\mathit{cumulative\_total} : \seq~\mathbb{N} \fun \mathbb{N}
\where
\mathit{cumulative\_total}(\langle \rangle) = 0
\forall x \colon \mathbb{N} \bullet \forall s \colon \seq~\mathbb{N} \bullet \mathit{cumulative\_total}(\langle x \rangle \cat s) = x + \mathit{cumulative\_total}(s)
\end{axdef}

\bigskip
\noindent
\textbf{Example 5 : Filter Pattern}

\medskip

\bigskip

Filtering sequences based on a condition:

\bigskip

\begin{axdef}
positives : \seq~\mathbb{Z} \fun \seq~\mathbb{Z}
\where
positives(\langle \rangle) = \langle \rangle
\forall x \colon \mathbb{Z} \bullet \forall s \colon \seq~\mathbb{Z} \bullet positives(\langle x \rangle \cat s) = (\mbox{if } x > 0 \mbox{ then } \langle x \rangle \cat positives(s) \mbox{ else } positives(s))
\end{axdef}

\end{document}
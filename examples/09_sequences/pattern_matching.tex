\documentclass[a4paper,10pt,fleqn]{article}
\usepackage[margin=1in]{geometry}
\usepackage{amssymb}
\usepackage{adjustbox}
\usepackage{natbib}
\usepackage[colorlinks=true,linkcolor=blue,citecolor=blue,urlcolor=blue]{hyperref}
\usepackage{fuzz}
\usepackage{zed-maths}
\usepackage{zed-proof}
\newdimen\savedleftskip
\begin{document}

\section*{Pattern Matching with Sequences}

\section*{Example 1 : Empty Sequence Pattern}
\addcontentsline{toc}{section}{Example 1 : Empty Sequence Pattern}

\noindent Pattern matching on the empty sequence:

\bigskip

\begin{axdef}
f : \seq~\nat \fun \nat
\where
f(\langle \rangle) = 0
\end{axdef}

\section*{Example 2 : Cons Pattern ( Head and Tail )}
\addcontentsline{toc}{section}{Example 2 : Cons Pattern ( Head and Tail )}

\noindent Decomposing a sequence into head and tail:

\bigskip

\begin{axdef}
g : \seq~\nat \fun \nat
\where
g(\langle \rangle) = 0 \\
\forall x : \nat @ \forall s : \seq~\nat @ g(\langle x \rangle \cat s) = x
\end{axdef}

\section*{Example 3 : Recursive Sum}
\addcontentsline{toc}{section}{Example 3 : Recursive Sum}

\noindent Computing the sum of elements using pattern matching:

\bigskip

\begin{axdef}
total : \seq~\nat \fun \nat
\where
total(\langle \rangle) = 0 \\
\forall x : \nat @ \forall s : \seq~\nat @ total(\langle x \rangle \cat s) = x + total(s)
\end{axdef}

\section*{Example 4 : Cumulative Total}
\addcontentsline{toc}{section}{Example 4 : Cumulative Total}

\noindent A more descriptive example from the solutions:

\bigskip

\begin{axdef}
cumulative\_total : \seq~\nat \fun \nat
\where
cumulative\_total(\langle \rangle) = 0 \\
\forall x : \nat @ \forall s : \seq~\nat @ cumulative\_total(\langle x \rangle \cat s) = x + cumulative\_total(s)
\end{axdef}

\section*{Example 5 : Filter Pattern}
\addcontentsline{toc}{section}{Example 5 : Filter Pattern}

\noindent Filtering sequences based on a condition:

\bigskip

\begin{axdef}
positives : \seq~\num \fun \seq~\num
\where
positives(\langle \rangle) = \langle \rangle \\
\forall x : \num @ \forall s : \seq~\num @ positives(\langle x \rangle \cat s) = \IF x > 0 \THEN \langle x \rangle \cat positives(s) \ELSE positives(s)
\end{axdef}

\end{document}
\documentclass[a4paper,10pt,fleqn]{article}
\usepackage[margin=1in]{geometry}
\usepackage{amssymb}
\usepackage{adjustbox}
\usepackage{natbib}
\usepackage[colorlinks=true,linkcolor=blue,citecolor=blue,urlcolor=blue]{hyperref}
\usepackage{fuzz}
\usepackage{zed-maths}
\usepackage{zed-proof}
\newdimen\savedleftskip
\begin{document}

\section*{Sequence Basics}

\section*{Example 1 : Sequence Types}
\addcontentsline{toc}{section}{Example 1 : Sequence Types}

\noindent Sequence types define collections of ordered elements:

\bigskip

\noindent
$\seq~\nat$


\noindent
$\seq~\num$


\noindent
$\iseq~\nat$


\section*{Example 2 : Sequence Literals ( Unicode )}
\addcontentsline{toc}{section}{Example 2 : Sequence Literals ( Unicode )}

\noindent Sequences can be written with Unicode angle brackets:

\bigskip

\noindent
$\langle \rangle$


\noindent
$\langle a \rangle$


\noindent
$\langle a, b, c \rangle$


\noindent
$\langle 1, 2, 3, 4 \rangle$


\section*{Example 3 : Sequence Literals ( ASCII )}
\addcontentsline{toc}{section}{Example 3 : Sequence Literals ( ASCII )}

\noindent ASCII angle brackets are also supported:

\bigskip

\noindent
$\langle \rangle$


\noindent
$\langle a \rangle$


\noindent
$\langle a, b, c \rangle$


\noindent
$\langle 1, 2, 3, 4 \rangle$


\section*{Example 4 : Sequences elem Expressions}
\addcontentsline{toc}{section}{Example 4 : Sequences elem Expressions}

\noindent Sequences can appear in various contexts:

\bigskip

\noindent
$s \in \seq~\nat$


\noindent
$\# \langle \rangle$


\noindent
$\# \langle a, b, c, d \rangle$


\end{document}
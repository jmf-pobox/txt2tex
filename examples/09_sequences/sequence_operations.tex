\documentclass[a4paper,10pt,fleqn]{article}
\usepackage[margin=1in]{geometry}
\usepackage{amssymb}
\usepackage{natbib}
\usepackage{fuzz}
\usepackage{zed-maths}
\usepackage{zed-proof}
\begin{document}

\section*{Sequence Operations}

\bigskip
\noindent
\textbf{Example 1 : Sequence Length}

\medskip

\bigskip

The cardinality operator gives the length of a sequence:

\bigskip

\noindent
$\# \langle \rangle$


\noindent
$\# \langle a, b, c \rangle$


\noindent
$\# s$


\bigskip
\noindent
\textbf{Example 2 : Head and Tail}

\medskip

\bigskip

Basic sequence decomposition:

\bigskip

\noindent
$\head \langle a, b, c \rangle$


\noindent
$\tail \langle a, b, c \rangle$


\noindent
$\head s$


\noindent
$\tail s$


\bigskip
\noindent
\textbf{Example 3 : Last and Front}

\medskip

\bigskip

Operations on the end of sequences:

\bigskip

\noindent
$\last \langle a, b, c \rangle$


\noindent
$\front \langle a, b, c \rangle$


\noindent
$\last s$


\noindent
$\front s$


\bigskip
\noindent
\textbf{Example 4 : Reverse}

\medskip

\bigskip

Reversing a sequence:

\bigskip

\noindent
$\rev \langle \rangle$


\noindent
$\rev \langle a \rangle$


\noindent
$\rev \langle a, b, c \rangle$


\noindent
$\rev \rev s$


\bigskip
\noindent
\textbf{Example 5 : Domain and Range}

\medskip

\bigskip

Sequences are functions from indices to elements:

\bigskip

\noindent
$\dom \langle \rangle$


\noindent
$\dom \langle a, b, c, d \rangle$


\noindent
$\ran \langle a, b, c \rangle$


\end{document}
\documentclass[a4paper,10pt,fleqn]{article}
\usepackage[margin=1in]{geometry}
\usepackage{amssymb}
\usepackage{adjustbox}
\usepackage{natbib}
\usepackage[colorlinks=true,linkcolor=blue,citecolor=blue,urlcolor=blue]{hyperref}
\usepackage{fuzz}
\usepackage{zed-maths}
\usepackage{zed-proof}
\newdimen\savedleftskip
\begin{document}

\section*{Sequence Operations}

\section*{Example 1 : Sequence Length}
\addcontentsline{toc}{section}{Example 1 : Sequence Length}

\noindent The cardinality operator gives the length of a sequence:

\bigskip

\noindent
$\# \langle \rangle$


\noindent
$\# \langle a, b, c \rangle$


\noindent
$\# s$


\section*{Example 2 : Head and Tail}
\addcontentsline{toc}{section}{Example 2 : Head and Tail}

\noindent Basic sequence decomposition:

\bigskip

\noindent
$\head \langle a, b, c \rangle$


\noindent
$\tail \langle a, b, c \rangle$


\noindent
$\head s$


\noindent
$\tail s$


\section*{Example 3 : Last and Front}
\addcontentsline{toc}{section}{Example 3 : Last and Front}

\noindent Operations on the end of sequences:

\bigskip

\noindent
$\last \langle a, b, c \rangle$


\noindent
$\front \langle a, b, c \rangle$


\noindent
$\last s$


\noindent
$\front s$


\section*{Example 4 : Reverse}
\addcontentsline{toc}{section}{Example 4 : Reverse}

\noindent Reversing a sequence:

\bigskip

\noindent
$\rev \langle \rangle$


\noindent
$\rev \langle a \rangle$


\noindent
$\rev \langle a, b, c \rangle$


\noindent
$\rev \rev s$


\section*{Example 5 : Domain and Range}
\addcontentsline{toc}{section}{Example 5 : Domain and Range}

\noindent Sequences are functions from indices to elements:

\bigskip

\noindent
$\dom \langle \rangle$


\noindent
$\dom \langle a, b, c, d \rangle$


\noindent
$\ran \langle a, b, c \rangle$


\end{document}
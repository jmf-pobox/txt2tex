\documentclass[a4paper,10pt,fleqn]{article}
\usepackage[margin=1in]{geometry}
\usepackage{amssymb}
\usepackage{natbib}
\usepackage{fuzz}
\usepackage{zed-maths}
\usepackage{zed-proof}
\begin{document}

\section*{Sequence Concatenation}

\bigskip
\noindent
\textbf{Example 1 : Simple Concatenation ( ASCII )}

\medskip

\bigskip

The caret operator after a sequence means concatenation:

\bigskip

\noindent
$\langle a \rangle \cat \langle b \rangle$


\noindent
$\langle 1, 2 \rangle \cat \langle 3, 4 \rangle$


\noindent
$\langle x \rangle \cat \langle \rangle$


\bigskip
\noindent
\textbf{Example 2 : Concatenation with Variables}

\medskip

\bigskip

Concatenation with sequence variables:

\bigskip

\noindent
$\langle x \rangle \cat s$


\noindent
$s \cat t$


\noindent
$\langle \rangle \cat t$


\bigskip
\noindent
\textbf{Example 3 : Cons Pattern}

\medskip

\bigskip

The cons pattern builds sequences incrementally:

\bigskip

\noindent
$\langle x \rangle \cat \langle \rangle$


\noindent
$\langle x \rangle \cat \langle y \rangle \cat \langle z \rangle$


\noindent
$\langle 1 \rangle \cat \langle 2 \rangle \cat \langle 3 \rangle \cat \langle \rangle$


\bigskip
\noindent
\textbf{Example 4 : Nested Concatenation}

\medskip

\bigskip

Concatenation is associative:

\bigskip

\noindent
$\langle a \rangle \cat \langle b \rangle \cat \langle c \rangle$


\noindent
$\langle a \rangle \cat (\langle b \rangle \cat \langle c \rangle)$


\noindent
$s \cat t \cat u$


\end{document}
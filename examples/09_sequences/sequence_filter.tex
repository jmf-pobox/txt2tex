\documentclass[a4paper,10pt,fleqn]{article}
\usepackage[margin=1in]{geometry}
\usepackage{amssymb}
\usepackage{natbib}
\usepackage{fuzz}
\usepackage{zed-maths}
\usepackage{zed-proof}
\begin{document}

\section*{Phase 35 : Sequence Filter Operator}

\bigskip

The sequence$ \filter $operator restricts a sequence to elements from
a specified set. It removes elements not in the$ \filter $set while
preserving order.

\bigskip

\bigskip

ASCII notation: s$ \filter $A

\bigskip

\bigskip

Unicode $alternative : s $$\filter$ A (U+21BE)

\bigskip

\bigskip
\noindent
\textbf{Example 1 : Basic Filter Operation}

\medskip

\bigskip

Filter a sequence to only include elements from a set.

\bigskip

\begin{axdef}
  s1 : \seq \nat \\
  evens1 : \power \nat \\
  filtered1 : \seq \nat
  \where
  s1 = \langle 1, 2, 3, 4, 5, 6 \rangle \land evens1 = \{2, 4, 6, 8\}
  \land filtered1 = (s1 \filter evens1)
\end{axdef}

\bigskip

The filtered sequence is $\langle 2, 4, 6 \rangle$, keeping only
elements that are in evens and preserving their original order.

\bigskip

\bigskip
\noindent
\textbf{Example 2 : Filter with Set Comprehension}

\medskip

\begin{zed}[Title2, Length2]
\end{zed}

\begin{axdef}
  all\_titles : \seq Title2 \\
  long\_titles : \power Title2 \\
  filtered\_titles : \seq Title2
  \where
  long\_titles = \{ t : Title2 | true \} \land filtered\_titles =
  all\_titles \filter long\_titles
\end{axdef}

\bigskip

This filters a sequence of titles to include only titles with length
greater than 120 minutes.

\bigskip

\bigskip
\noindent
\textbf{Example 3 : Comparison with Range Restriction}

\medskip

\bigskip

Sequence$ \filter $(filter) is different from range restriction ($\rres$):

\bigskip

\bigskip

- filter: operates on sequences, preserves order

\bigskip

\bigskip

- $\rres$: operates on relations, restricts range

\bigskip

\begin{axdef}
  s3 : \seq \nat \\
  R3 : \nat \rel \nat \\
  A3 : \power \nat \\
  filtered\_seq3 : \seq \nat \\
  restricted\_rel3 : \nat \rel \nat
  \where
  s3 = \langle 1, 2, 3, 4, 5 \rangle \land R3 = \{1 \mapsto 10, 2
  \mapsto 20, 3 \mapsto 30\} \land A3 = \{2, 3, 4\} \land
  filtered\_seq3 = s3 \filter A3 \land restricted\_rel3 = R3 \rres \{10, 20\}
\end{axdef}

\bigskip

filtered\_seq is $\langle 2, 3, 4 \rangle$ (sequence $elements \in A$)

\bigskip

\bigskip

restricted\_rel is $\{1 \mapsto 10, 2 \mapsto 20\}$ (relation pairs
with range in $\{10, 20\}$)

\bigskip

\bigskip
\noindent
\textbf{Example 4 : Practical Example - Video Database}

\medskip

\begin{zed}[Title4, Length4]
\end{zed}

\begin{zed}Rating4 ::= G4 | PG4 | PG13_4 | R4
\end{zed}

\begin{axdef}
  catalog4 : \seq (Title4 \cross Length4 \cross Rating4) \\
  family\_friendly4 : \power Rating4
  \where
  family\_friendly4 = \{G4, PG4\}
\end{axdef}

\bigskip

The family\_friendly set contains ratings G and PG. A$ \filter $could
be applied to the catalog sequence to only include entries with
ratings in this set.

\bigskip

\bigskip
\noindent
\textbf{Example 5 : Multiple Filters}

\medskip

\begin{axdef}
  s5 : \seq \nat \\
  evens5 : \power \nat \\
  large5 : \power \nat \\
  result5 : \seq \nat
  \where
  s5 = \langle 1, 2, 3, 4, 5, 6, 7, 8, 9, 10 \rangle \land evens5 =
  \{2, 4, 6, 8, 10, 12\} \land large5 = \{5, 6, 7, 8, 9, 10, 11\}
  \land result5 = (s5 \filter evens5) \filter large5
\end{axdef}

\bigskip

Multiple filters can be applied sequentially. First$ \filter $to
evens gives $\langle 2, 4, 6, 8, 10 \rangle$, then$ \filter $to large
gives $\langle 6, 8, 10 \rangle$.

\bigskip

\bigskip
\noindent
\textbf{Example 6 : Empty Filter}

\medskip

\begin{axdef}
  s6 : \seq \nat \\
  empty\_set6 : \power \nat \\
  disjoint\_set6 : \power \nat \\
  result6a : \seq \nat \\
  result6b : \seq \nat
  \where
  s6 = \langle 1, 2, 3 \rangle \land empty\_set6 = \{\} \land
  disjoint\_set6 = \{10, 20, 30\} \land result6a = s6 \filter
  empty\_set6 \land result6b = s6 \filter disjoint\_set6
\end{axdef}

\bigskip

Filtering by an empty set or disjoint set results in an empty sequence.

\bigskip

\bigskip

result1 = $\langle \rangle$ and result2 = $\langle \rangle$

\bigskip

\bigskip
\noindent
\textbf{Example 7 : Bag Union ( Related Phase 35 Feature )}

\medskip

\bigskip

Phase 35 also includes the bag union operator.

\bigskip

\bigskip

ASCII $notation : b1  \uplus  b2$

\bigskip

\bigskip

Unicode $alternative : b1 $$\uplus$ b2 (U+228E)

\bigskip

\begin{axdef}
  b1 : \bag \nat \\
  b2 : \bag \nat \\
  combined : \bag \nat
  \where
  b1 = \lbag 1, 1, 2, 3 \rbag \land b2 = \lbag 2, 3, 3, 4 \rbag \land
  combined = b1 \uplus b2
\end{axdef}

\bigskip

Bag union adds $multiplicities : combined $= [[1, 1, 2, 2, 3, 3, 3, 4]]

\bigskip

\bigskip

Element 1 appears 2 times (2 from b1), element 2 appears 2 times (1
from each), element 3 appears 3 times (1 from b1, 2 from b2), element
4 appears 1 time (1 from b2).

\bigskip

\end{document}

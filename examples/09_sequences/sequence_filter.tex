\documentclass[a4paper,10pt,fleqn]{article}
\usepackage[margin=1in]{geometry}
\usepackage{amssymb}
\usepackage{natbib}
\usepackage{zed-cm}
\usepackage{zed-maths}
\usepackage{zed-proof}
\begin{document}

\section*{Phase 35 : Sequence Filter Operator}

\bigskip

The sequence$ \filter $operator restricts a sequence to elements from
a specified set. It removes elements not in the$ \filter $set while
preserving order.

\bigskip

\bigskip

ASCII notation: s$ \filter $A

\bigskip

\bigskip

Unicode $alternative : s $$\filter$ A (U+21BE)

\bigskip

\bigskip
\noindent
\textbf{Example 1 : Basic Filter Operation}

\medskip

\bigskip

Filter a sequence to only include elements from a set.

\bigskip

\begin{axdef}
  s : \seq \mathbb{N} \\
  evens : \power \mathbb{N} \\
  filtered : \seq \mathbb{N}
  \where
  s = \langle 1, 2, 3, 4, 5, 6 \rangle \land evens = \{2, 4, 6, 8\}
  \land filtered = (s \filter evens)
\end{axdef}

\bigskip

The filtered sequence is $\langle 2, 4, 6 \rangle$, keeping only
elements that are in evens and preserving their original order.

\bigskip

\bigskip
\noindent
\textbf{Example 2 : Filter with Set Comprehension}

\medskip

\begin{zed}[Title, Length]
\end{zed}

\begin{axdef}
  \mathit{all\_titles} : \seq Title \\
  \mathit{long\_titles} : \power Title \\
  \mathit{filtered\_titles} : \seq Title
  \where
  \mathit{long\_titles} = \{ t \colon Title \mid true \} \land
  \mathit{filtered\_titles} = \mathit{all\_titles} \filter \mathit{long\_titles}
\end{axdef}

\bigskip

This filters a sequence of titles to include only titles with length
greater than 120 minutes.

\bigskip

\bigskip
\noindent
\textbf{Example 3 : Comparison with Range Restriction}

\medskip

\bigskip

Sequence$ \filter $(filter) is different from range restriction ($\rres$):

\bigskip

\bigskip

- filter: operates on sequences, preserves order

\bigskip

\bigskip

- $\rres$: operates on relations, restricts range

\bigskip

\begin{axdef}
  s : \seq \mathbb{N} \\
  R : \mathbb{N} \rel \mathbb{N} \\
  A : \power \mathbb{N} \\
  \mathit{filtered\_seq} : \seq \mathbb{N} \\
  \mathit{restricted\_rel} : \mathbb{N} \rel \mathbb{N}
  \where
  s = \langle 1, 2, 3, 4, 5 \rangle \land R = \{1 \mapsto 10, 2
  \mapsto 20, 3 \mapsto 30\} \land A = \{2, 3, 4\} \land
  \mathit{filtered\_seq} = s \filter A \land \mathit{restricted\_rel}
  = R \rres \{10, 20\}
\end{axdef}

\bigskip

filtered_seq is $\langle 2, 3, 4 \rangle$ (sequence $elements \in A$)

\bigskip

\bigskip

restricted_rel is $\{1 \mapsto 10, 2 \mapsto 20\}$ (relation pairs
with range in $\{10, 20\}$)

\bigskip

\bigskip
\noindent
\textbf{Example 4 : Practical Example - Video Database}

\medskip

\begin{zed}[Title, Length, Rating]
\end{zed}

\begin{axdef}
  catalog : \seq ((Title \cross Length) \cross Rating) \\
  \mathit{family\_friendly} : \power Rating
  \where
  \mathit{family\_friendly} = \{G, PG\}
\end{axdef}

\bigskip

The family_friendly set contains ratings G and PG. A$ \filter $could
be applied to the catalog sequence to only include entries with
ratings in this set.

\bigskip

\bigskip
\noindent
\textbf{Example 5 : Multiple Filters}

\medskip

\begin{axdef}
  s : \seq \mathbb{N} \\
  evens : \power \mathbb{N} \\
  large : \power \mathbb{N} \\
  result : \seq \mathbb{N}
  \where
  s = \langle 1, 2, 3, 4, 5, 6, 7, 8, 9, 10 \rangle \land evens =
  \{2, 4, 6, 8, 10, 12\} \land large = \{5, 6, 7, 8, 9, 10, 11\}
  \land result = (s \filter evens) \filter large
\end{axdef}

\bigskip

Multiple filters can be applied sequentially. First$ \filter $to
evens gives $\langle 2, 4, 6, 8, 10 \rangle$, then$ \filter $to large
gives $\langle 6, 8, 10 \rangle$.

\bigskip

\bigskip
\noindent
\textbf{Example 6 : Empty Filter}

\medskip

\begin{axdef}
  s : \seq \mathbb{N} \\
  \mathit{empty\_set} : \power \mathbb{N} \\
  \mathit{disjoint\_set} : \power \mathbb{N} \\
  result1 : \seq \mathbb{N} \\
  result2 : \seq \mathbb{N}
  \where
  s = \langle 1, 2, 3 \rangle \land \mathit{empty\_set} = \{\} \land
  \mathit{disjoint\_set} = \{10, 20, 30\} \land result1 = s \filter
  \mathit{empty\_set} \land result2 = s \filter \mathit{disjoint\_set}
\end{axdef}

\bigskip

Filtering by an empty set or disjoint set results in an empty sequence.

\bigskip

\bigskip

result1 = $\langle \rangle$ and result2 = $\langle \rangle$

\bigskip

\bigskip
\noindent
\textbf{Example 7 : Bag Union ( Related Phase 35 Feature )}

\medskip

\bigskip

Phase 35 also includes the bag union operator.

\bigskip

\bigskip

ASCII $notation : b1  \uplus  b2$

\bigskip

\bigskip

Unicode $alternative : b1 $$\uplus$ b2 (U+228E)

\bigskip

\begin{axdef}
  b1 : \bag \mathbb{N} \\
  b2 : \bag \mathbb{N} \\
  combined : \bag \mathbb{N}
  \where
  b1 = \lbag 1, 1, 2, 3 \rbag \land b2 = \lbag 2, 3, 3, 4 \rbag \land
  combined = b1 \uplus b2
\end{axdef}

\bigskip

Bag union adds $multiplicities : combined $= [[1, 1, 2, 2, 3, 3, 3, 4]]

\bigskip

\bigskip

Element 1 appears 2 times (2 from b1), element 2 appears 2 times (1
from each), element 3 appears 3 times (1 from b1, 2 from b2), element
4 appears 1 time (1 from b2).

\bigskip

\end{document}

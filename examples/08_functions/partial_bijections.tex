\documentclass[a4paper,10pt,fleqn]{article}
\usepackage[margin=1in]{geometry}
\usepackage{amssymb}
\usepackage{natbib}
\usepackage{zed-cm}
\usepackage{zed-maths}
\usepackage{zed-proof}
\begin{document}

\section*{Phase 33 : Partial Bijections}

\bigskip

Partial bijections are functions that are both injective (one-to-one)
and surjective (onto), but defined only on a subset of their domain.
The $operator \pbij denotes$ a partial bijection.

\bigskip

\bigskip
\noindent
\textbf{Example 1 : Basic Partial Bijection Type}

\medskip

\bigskip

A partial bijection from X to Y is a subset of X cross Y where each
element in the domain maps to a unique element in the range, and each
element in the range has at most one preimage.

\bigskip

\begin{axdef}
  partialMap : \mathbb{N} \pbij \mathbb{N}
  \where
  partialMap = \{1 \mapsto 10, 2 \mapsto 20, 3 \mapsto 30\}
\end{axdef}

\bigskip

This is a partial bijection because:

\bigskip

\bigskip

1. It is injective: different inputs map to different outputs

\bigskip

\bigskip

2. It is partial: not defined for all natural numbers

\bigskip

\bigskip

3. It is bijective on its domain: the restriction is a bijection

\bigskip

\bigskip
\noindent
\textbf{Example 2 : Partial Bijection in Type Signatures}

\medskip

\begin{zed}[Student, Course]
\end{zed}

\begin{axdef}
  enrollment : Student \pbij Course
  \where
  \# \dom enrollment \leq 100 \land (\forall s1, s2 \colon Student
    \bullet s1 \in \dom enrollment \land s2 \in \dom enrollment
  \Rightarrow (s1 \neq s2 \Rightarrow enrollment(s1) \neq enrollment(s2)))
\end{axdef}

\bigskip

The enrollment relation is a partial bijection because each enrolled
student is assigned to exactly one course, and each course has at
most one student assigned to it. Not all students need to be enrolled.

\bigskip

\bigskip
\noindent
\textbf{Example 3 : Comparison with Other Function Types}

\medskip

\bigskip

Comparison of function type operators:

\bigskip

\bigskip

- Total function ($\fun$): defined for all inputs, no uniqueness constraint

\bigskip

\bigskip

- Partial function ($\pfun$): may not be defined for all inputs

\bigskip

\bigskip

- Total bijection ($\bij$): defined everywhere, injective and surjective

\bigskip

\bigskip

- Partial bijection ($\pbij$): may not be defined everywhere, but
injective and surjective on domain

\bigskip

\begin{zed}[X, Y]
\end{zed}

\begin{axdef}
  totalFunc : X \fun Y \\
  partialFunc : X \pfun Y \\
  totalBij : X \bij Y \\
  partialBij : X \pbij Y
  \where
  \dom totalFunc = X \land \dom totalBij = X \land (\forall x1, x2
  \colon X \bullet totalBij(x1) = totalBij(x2) \Rightarrow x1 = x2)
  \land \ran totalBij = Y
\end{axdef}

\bigskip
\noindent
\textbf{Example 4 : Nested Partial Bijection Types}

\medskip

\bigskip

Function types are right-associative, so $X \pbij (Y \pbij
\mathbb{Z})$ means $X > 7$$\fun$ ($Y \pbij \mathbb{Z}$), which is a
partial bijection from X to partial bijections from Y to Z.

\bigskip

\begin{zed}[A, B, C]
\end{zed}

\begin{axdef}
  higherOrder : A \pbij (B \pbij C)
  \where
  \# \dom higherOrder > 0
\end{axdef}

\bigskip

This represents a partially defined function that, when given an
element of A, returns a partial bijection from B to C.

\bigskip

\bigskip
\noindent
\textbf{Example 5 : Operations on Partial Bijections}

\medskip

\begin{axdef}
  f : \mathbb{N} \pbij \mathbb{N} \\
  g : \mathbb{N} \pbij \mathbb{N}
  \where
  f = \{1 \mapsto 2, 2 \mapsto 4, 3 \mapsto 6\} \land g = \{10
  \mapsto 20, 20 \mapsto 40, 30 \mapsto 60\} \land \dom f \cap \dom g
  = \{\} \land \ran f \cap \ran g = \{\}
\end{axdef}

\bigskip

Since partial bijections are relations, we can use relational
operators like domain, range, inverse, and composition.

\bigskip

\begin{axdef}
  \mathit{f\_inv} : \mathbb{N} \pbij \mathbb{N} \\
  \mathit{f\_composed} : \mathbb{N} \pbij \mathbb{N}
  \where
  \mathit{f\_inv} = f^{-1} \land \mathit{f\_composed} = g \circ f
\end{axdef}

\bigskip

The inverse f~ is also a partial bijection because f is injective and
surjective on its domain. Composition of partial bijections may not
preserve the bijective property in general.

\bigskip

\end{document}

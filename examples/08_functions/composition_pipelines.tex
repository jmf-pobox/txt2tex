\documentclass[a4paper,10pt,fleqn]{article}
\usepackage[margin=1in]{geometry}
\usepackage{amssymb}
\usepackage{adjustbox}
\usepackage{natbib}
\usepackage[colorlinks=true,linkcolor=blue,citecolor=blue,urlcolor=blue]{hyperref}
\usepackage{fuzz}
\usepackage{zed-maths}
\usepackage{zed-proof}
\newdimen\savedleftskip
\begin{document}

\section*{Composition Pipelines}

\section*{Example 1 : Simple Three - Stage Pipeline}
\addcontentsline{toc}{section}{Example 1 : Simple Three - Stage Pipeline}

\noindent A three-stage numeric processing pipeline.

\bigskip

\begin{axdef}
  addOne : \nat \fun \nat \\
  doubleCP : \nat \fun \nat \\
  squareCP : \nat \fun \nat \\
  pipeline : \nat \fun \nat
  \where
  \forall n : \nat @ addOne(n) = n + 1 \\
  \forall n : \nat @ doubleCP(n) = 2 * n \\
  \forall n : \nat @ squareCP(n) = n * n \\
  pipeline = squareCP \comp doubleCP \comp addOne
\end{axdef}

\noindent pipeline(3) = squareCP(doubleCP(addOne(3))) =
squareCP(doubleCP(4)) = squareCP(8) = 64

\bigskip

\section*{Example 2 : String Pipeline}
\addcontentsline{toc}{section}{Example 2 : String Pipeline}

\begin{zed}
  [CharCP]
\end{zed}

\begin{axdef}
  str1 : \seq~CharCP \fun \seq~CharCP \\
  str2 : \seq~CharCP \fun \seq~CharCP \\
  str3 : \seq~CharCP \fun \seq~CharCP \\
  stringPipeline : \seq~CharCP \fun \seq~CharCP
  \where
  stringPipeline = str3 \comp str2 \comp str1
\end{axdef}

\noindent Demonstrates composition of string-processing functions.

\bigskip

\section*{Example 3 : Best Practices}
\addcontentsline{toc}{section}{Example 3 : Best Practices}

\noindent When building pipelines:

\bigskip

\noindent 1. Break complex operations into simple stages

\bigskip

\noindent 2. Each stage should have a single responsibility

\bigskip

\noindent 3. Test individual functions before composing

\bigskip

\noindent 4. Use meaningful names for pipeline stages

\bigskip

\noindent 5. Document the flow direction (comp applies right-to-left)

\bigskip

\end{document}

\documentclass[a4paper,10pt,fleqn]{article}
\usepackage[margin=1in]{geometry}
\usepackage{amssymb}
\usepackage{natbib}
\usepackage{fuzz}
\usepackage{zed-maths}
\usepackage{zed-proof}
\begin{document}

\section*{Simple Function Definitions}

\bigskip

This file demonstrates basic function $definitions \in Z$ notation that work well with fuzz type checking.

\bigskip

\bigskip
\noindent
\textbf{Example 1 : Simple Total Functions}

\medskip

\bigskip

Total functions are defined over their entire domain.

\bigskip

\begin{axdef}
square : \nat \fun \nat
\where
\forall n : \nat @ square(n) = n * n
\end{axdef}

\begin{axdef}
successor : \nat \fun \nat
\where
\forall n : \nat @ successor(n) = n + 1
\end{axdef}

\begin{axdef}
double : \nat \fun \nat
\where
\forall n : \nat @ double(n) = 2 * n
\end{axdef}

\bigskip
\noindent
\textbf{Example 2 : Partial Functions}

\medskip

\bigskip

Partial functions are not defined over their entire domain.

\bigskip

\begin{axdef}
predecessor : \nat \pfun \nat
\where
\forall n : \nat @ n > 0 \implies predecessor(n) = n - 1
\end{axdef}

\bigskip

predecessor is partial on natural numbers since 0 has no $predecessor \in N$.

\bigskip

\bigskip
\noindent
\textbf{Example 3 : Generic Functions}

\medskip

\bigskip

Generic functions work with any type parameter.

\bigskip

\begin{gendef}[X]
  identity: X \fun X
\where
  \forall x : X @ identity(x) = x
\end{gendef}

\begin{gendef}[X, Y]
  fst: X \cross Y \fun X
\where
  \forall x : X @ \forall y : Y @ fst(x, y) = x
\end{gendef}

\begin{gendef}[X, Y]
  snd: X \cross Y \fun Y
\where
  \forall x : X @ \forall y : Y @ snd(x, y) = y
\end{gendef}

\bigskip
\noindent
\textbf{Example 4 : Functions with Given Types}

\medskip

\begin{zed}[Person, Department]\end{zed}

\begin{axdef}
assignment : Person \pfun Department
\end{axdef}

\bigskip

assignment is a partial function from Person to Department.

\bigskip

\bigskip
\noindent
\textbf{Example 5 : Functions on Numbers}

\medskip

\begin{axdef}
triple : \nat \fun \nat
\where
\forall n : \nat @ triple(n) = 3 * n
\end{axdef}

\begin{axdef}
addOne : \num \fun \num
\where
\forall n : \num @ addOne(n) = n + 1
\end{axdef}

\bigskip
\noindent
\textbf{Example 6 : Function with Given Types}

\medskip

\begin{zed}[Student, Grade]\end{zed}

\begin{axdef}
grades : Student \pfun Grade
\end{axdef}

\bigskip

grades maps students to their grades.

\bigskip

\bigskip
\noindent
\textbf{Example 7 : Function Composition Example}

\medskip

\begin{axdef}
f : \nat \fun \nat \\
g : \nat \fun \nat \\
h : \nat \fun \nat
\where
\forall n : \nat @ f(n) = 2 * n \\
\forall n : \nat @ g(n) = n + 1 \\
h = f \circ g
\end{axdef}

\bigskip

f doubles its input, g adds one to its input. h is their forward $composition : h(n) $= f(g(n)) = 2 * (n + 1).

\bigskip

\bigskip
\noindent
\textbf{Example 8 : Modulo Function}

\medskip

\begin{axdef}
modulo3 : \nat \fun \nat
\where
\forall n : \nat @ modulo3(n) = n \mod 3
\end{axdef}

\bigskip

For comprehensive function examples including recursive functions, higher-order functions, and advanced patterns, see examples/09_sequences/pattern_matching.txt.

\bigskip

\end{document}
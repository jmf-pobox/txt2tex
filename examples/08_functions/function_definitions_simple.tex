\documentclass[a4paper,10pt,fleqn]{article}
\usepackage[margin=1in]{geometry}
\usepackage{amssymb}
\usepackage{adjustbox}
\usepackage{natbib}
\usepackage[colorlinks=true,linkcolor=blue,citecolor=blue,urlcolor=blue]{hyperref}
\usepackage{fuzz}
\usepackage{zed-maths}
\usepackage{zed-proof}
\newdimen\savedleftskip
\begin{document}

\section*{Simple Function Definitions}

\noindent This file demonstrates basic function definitions in Z
notation that work well with fuzz type checking.

\bigskip

\section*{Example 1 : Simple Total Functions}
\addcontentsline{toc}{section}{Example 1 : Simple Total Functions}

\noindent Total functions are defined over their entire domain.

\bigskip

\begin{axdef}
  square : \nat \fun \nat
  \where
  \forall n : \nat @ square(n) = n * n
\end{axdef}

\begin{axdef}
  successor : \nat \fun \nat
  \where
  \forall n : \nat @ successor(n) = n + 1
\end{axdef}

\begin{axdef}
  double : \nat \fun \nat
  \where
  \forall n : \nat @ double(n) = 2 * n
\end{axdef}

\section*{Example 2 : Partial Functions}
\addcontentsline{toc}{section}{Example 2 : Partial Functions}

\noindent Partial functions are not defined over their entire domain.

\bigskip

\begin{axdef}
  predecessor : \nat \pfun \nat
  \where
  \forall n : \nat | n > 0 @ predecessor(n) = n - 1
\end{axdef}

\noindent predecessor is partial on natural numbers since 0 has no
predecessor in N. The bullet separator filters to positive numbers
(where predecessor is defined), then specifies the function value.

\bigskip

\section*{Example 3 : Generic Functions}
\addcontentsline{toc}{section}{Example 3 : Generic Functions}

\noindent Generic functions work with any type parameter.

\bigskip

\begin{gendef}[X]
  identity: X \fun X
  \where
  \forall x : X @ identity(x) = x
\end{gendef}

\begin{gendef}[X, Y]
  fst: X \cross Y \fun X
  \where
  \forall x : X @ \forall y : Y @ fst(x, y) = x
\end{gendef}

\begin{gendef}[X, Y]
  snd: X \cross Y \fun Y
  \where
  \forall x : X @ \forall y : Y @ snd(x, y) = y
\end{gendef}

\section*{Example 4 : Functions with Given Types}
\addcontentsline{toc}{section}{Example 4 : Functions with Given Types}

\begin{zed}
  [Person, Department]
\end{zed}

\begin{axdef}
  assignment : Person \pfun Department
\end{axdef}

\noindent assignment is a partial function from Person to Department.

\bigskip

\section*{Example 5 : Functions on Numbers}
\addcontentsline{toc}{section}{Example 5 : Functions on Numbers}

\begin{axdef}
  triple : \nat \fun \nat
  \where
  \forall n : \nat @ triple(n) = 3 * n
\end{axdef}

\begin{axdef}
  addOne : \num \fun \num
  \where
  \forall n : \num @ addOne(n) = n + 1
\end{axdef}

\section*{Example 6 : Function with Given Types}
\addcontentsline{toc}{section}{Example 6 : Function with Given Types}

\begin{zed}
  [Student, Grade]
\end{zed}

\begin{axdef}
  grades : Student \pfun Grade
\end{axdef}

\noindent grades maps students to their grades.

\bigskip

\section*{Example 7 : Function Composition Example}
\addcontentsline{toc}{section}{Example 7 : Function Composition Example}

\begin{axdef}
  f : \nat \fun \nat \\
  g : \nat \fun \nat \\
  h : \nat \fun \nat
  \where
  \forall n : \nat @ f(n) = 2 * n \\
  \forall n : \nat @ g(n) = n + 1 \\
  h = f \circ g
\end{axdef}

\noindent f doubles its input, g adds one to its input. h is their
forward $composition : h(n)$ = f(g(n)) = 2 * (n + 1).

\bigskip

\section*{Example 8 : Modulo Function}
\addcontentsline{toc}{section}{Example 8 : Modulo Function}

\begin{axdef}
  modulo3 : \nat \fun \nat
  \where
  \forall n : \nat @ modulo3(n) = n \mod 3
\end{axdef}

\noindent For comprehensive function examples including recursive
functions, higher-order functions, and advanced patterns, see
examples/09\_sequences/pattern\_matching.txt.

\bigskip

\end{document}

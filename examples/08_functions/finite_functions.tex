\documentclass[a4paper,10pt,fleqn]{article}
\usepackage[margin=1in]{geometry}
\usepackage{amssymb}
\usepackage{natbib}
\usepackage{fuzz}
\usepackage{zed-maths}
\usepackage{zed-proof}
\begin{document}

\section*{Phase 34 : Finite Partial Functions}

\bigskip

Finite partial functions are partial functions with a finite domain.
The $operator \ffun denotes$ a finite partial function, meaning the
function may not be defined everywhere, and its domain must be finite.

\bigskip

\bigskip
\noindent
\textbf{Example 1 : Basic Finite Function Type}

\medskip

\bigskip

A finite partial function from X to Y is a partial function whose
domain is finite. This is useful when modeling resources with limited
capacity or databases with a bounded number of entries.

\bigskip

\begin{zed}[Year, Table]
\end{zed}

\begin{axdef}
  records : Year \ffun Table
  \where
  \# (\dom records) \leq 1000
\end{axdef}

\bigskip

The records relation is a finite partial function because:

\bigskip

\bigskip

1. It is partial: not all years need to have a table

\bigskip

\bigskip

2. Its domain is finite: at most 1000 years have records

\bigskip

\bigskip

3. It is functional: each year maps to at most one table

\bigskip

\bigskip
\noindent
\textbf{Example 2 : Comparison with Other Function Types}

\medskip

\bigskip

Comparison of function type operators:

\bigskip

\bigskip

- Total function ($\fun$): defined for all inputs, domain is entire source set

\bigskip

\bigskip

- Partial function ($\pfun$): may not be defined for all inputs,
domain may be infinite

\bigskip

\bigskip

- Finite partial function ($\ffun$): partial function with finite domain

\bigskip

\bigskip

- Total bijection ($\bij$): defined everywhere, injective and surjective

\bigskip

\begin{zed}[X, Y]
\end{zed}

\begin{axdef}
  totalFunc : X \fun Y \\
  partialFunc : X \pfun Y \\
  finitePartialFunc : X \ffun Y \\
  totalBij : X \bij Y
  \where
  \dom totalFunc = X \land \dom totalBij = X \land (\forall x1, x2 :
  X @ totalBij(x1) = totalBij(x2) \implies x1 = x2) \land \ran totalBij = Y
\end{axdef}

\bigskip
\noindent
\textbf{Example 3 : Finite Functions in Practice}

\medskip

\bigskip

Real-world example: A database table storing customer preferences.

\bigskip

\begin{zed}[CustomerID, Preference]
\end{zed}

\begin{axdef}
  preferences : CustomerID \ffun Preference
  \where
  \# (\dom preferences) \leq 10000
\end{axdef}

\bigskip

This models a preference database where:

\bigskip

\bigskip

- Not all customers have preferences recorded (partial)

\bigskip

\bigskip

- The database has a capacity limit (finite domain)

\bigskip

\bigskip

- Each customer maps to one preference value (functional)

\bigskip

\bigskip
\noindent
\textbf{Example 4 : Operations on Finite Functions}

\medskip

\begin{axdef}
  f : \nat \ffun \nat \\
  g : \nat \ffun \nat
  \where
  f = \{1 \mapsto 10, 2 \mapsto 20, 3 \mapsto 30\} \land g = \{10
  \mapsto 100, 20 \mapsto 200\} \land \# (\dom f) = 3 \land \# (\dom g) = 2
\end{axdef}

\bigskip

We can compose finite functions, but the result may not be finite unless proven.

\bigskip

\begin{axdef}
  f\_composed : \nat \ffun \nat
  \where
  f\_composed = g \circ f \land \# (\dom f\_composed) \leq \# (\dom f)
\end{axdef}

\bigskip

The composition has a domain size bounded by the domain of f, since
we can only compose where f's range intersects g's domain.

\bigskip

\bigskip
\noindent
\textbf{Example 5 : Domain Restrictions and Finiteness}

\medskip

\begin{zed}[Title, Length, ViewDate]
\end{zed}

\begin{axdef}
  viewed : Title \ffun ViewDate \\
  recent\_viewed : Title \ffun ViewDate \\
  recentSet : \power Title
  \where
  recentSet = \{ t : Title | t \in \dom viewed \} \land
  recent\_viewed = recentSet \dres viewed \land \# (\dom
  recent\_viewed) \leq \# (\dom viewed)
\end{axdef}

\bigskip

Restricting a finite partial function preserves finiteness. The
recent\_viewed function has a domain that is a subset of viewed's
domain, so it remains finite.

\bigskip

\end{document}

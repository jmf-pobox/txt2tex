\documentclass[a4paper,10pt,fleqn]{article}
\usepackage[margin=1in]{geometry}
\usepackage{amssymb}
\usepackage{adjustbox}
\usepackage{natbib}
\usepackage[colorlinks=true,linkcolor=blue,citecolor=blue,urlcolor=blue]{hyperref}
\usepackage{fuzz}
\usepackage{zed-maths}
\usepackage{zed-proof}
\newdimen\savedleftskip
\begin{document}

\section*{Phase 34 : Finite Partial Functions}

\noindent Finite partial functions are partial functions with a finite domain. The $operator \ffun denotes$ a finite partial function, meaning the function may not be defined everywhere, and its domain must be finite.

\bigskip

\section*{Example 1 : Basic Finite Function Type}
\addcontentsline{toc}{section}{Example 1 : Basic Finite Function Type}

\noindent A finite partial function from X to Y is a partial function whose domain is finite. This is useful when modeling resources with limited capacity or databases with a bounded number of entries.

\bigskip

\begin{zed}
[Year, TableFF]
\end{zed}

\begin{axdef}
records : Year \ffun TableFF
\where
\# (\dom records) \leq 1000
\end{axdef}

\noindent The records relation is a finite partial function because:

\bigskip

\noindent 1. It is partial: not all years need to have a table

\bigskip

\noindent 2. Its domain is finite: at most 1000 years have records

\bigskip

\noindent 3. It is functional: each year maps to at most one table

\bigskip

\section*{Example 2 : Comparison with Other Function Types}
\addcontentsline{toc}{section}{Example 2 : Comparison with Other Function Types}

\noindent Comparison of function type operators:

\bigskip

\noindent - Total function ($\fun$): defined for all inputs, domain is entire source set

\bigskip

\noindent - Partial function ($\pfun$): may not be defined for all inputs, domain may be infinite

\bigskip

\noindent - Finite partial function ($\ffun$): partial function with finite domain

\bigskip

\noindent - Total bijection ($\bij$): defined everywhere, injective and surjective

\bigskip

\begin{zed}
[X, Y]
\end{zed}

\begin{axdef}
totalFunc : X \fun Y \\
partialFunc : X \pfun Y \\
finitePartialFunc : X \ffun Y \\
totalBij : X \bij Y
\where
\dom totalFunc = X \land \dom totalBij = X \land (\forall x1, x2 : X | totalBij(x1) = totalBij(x2) @ x1 = x2) \land \ran totalBij = Y
\end{axdef}

\section*{Example 3 : Finite Functions elem Practice}
\addcontentsline{toc}{section}{Example 3 : Finite Functions elem Practice}

\noindent Real-world example: A database table storing customer preferences.

\bigskip

\begin{zed}
[CustomerID, Preference]
\end{zed}

\begin{axdef}
preferences : CustomerID \ffun Preference
\where
\# (\dom preferences) \leq 10000
\end{axdef}

\noindent This models a preference database where:

\bigskip

\noindent - Not all customers have preferences recorded (partial)

\bigskip

\noindent - The database has a capacity limit (finite domain)

\bigskip

\noindent - Each customer maps to one preference value (functional)

\bigskip

\section*{Example 4 : Operations on Finite Functions}
\addcontentsline{toc}{section}{Example 4 : Operations on Finite Functions}

\begin{axdef}
f : \nat \ffun \nat \\
g : \nat \ffun \nat
\where
f = \{1 \mapsto 10, 2 \mapsto 20, 3 \mapsto 30\} \land \\
\quad g = \{10 \mapsto 100, 20 \mapsto 200\} \land \\
\quad \# (\dom f) = 3 \land \\
\quad \# (\dom g) = 2
\end{axdef}

\noindent We can compose finite functions, but the result may not be finite unless proven.

\bigskip

\begin{axdef}
f\_composed : \nat \ffun \nat
\where
f\_composed = g \circ f \land \# (\dom f\_composed) \leq \# (\dom f)
\end{axdef}

\noindent The composition has a domain size bounded by the domain of f, since we can only compose where f's range intersects g's domain.

\bigskip

\section*{Example 5 : Domain Restrictions land Finiteness}
\addcontentsline{toc}{section}{Example 5 : Domain Restrictions land Finiteness}

\begin{zed}
[Title, Length, ViewDate]
\end{zed}

\begin{axdef}
viewed : Title \ffun ViewDate \\
recent\_viewed : Title \ffun ViewDate \\
recentSet : \power Title
\where
recentSet = \{~ t : Title | t \in \dom viewed ~\} \land \\
\quad recent\_viewed = recentSet \dres viewed \land \\
\quad \# (\dom recent\_viewed) \leq \# (\dom viewed)
\end{axdef}

\noindent Restricting a finite partial function preserves finiteness. The recent\_viewed function has a domain that is a subset of viewed's domain, so it remains finite.

\bigskip

\end{document}
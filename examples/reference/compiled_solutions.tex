\documentclass[a4paper,10pt,fleqn]{article}
\usepackage[margin=1in]{geometry}
\usepackage{amssymb}
\usepackage{natbib}
\usepackage{fuzz}
\usepackage{zed-maths}
\usepackage{zed-proof}
\begin{document}

\section*{Propositional logic}

\bigskip
\noindent
\textbf{Solution 1}

\medskip

\noindent
\hangindent=2em
(a) $false(as(true \implies false) \Leftrightarrow false)$

\medskip
\noindent
\hangindent=2em
(b) $true(as(false \implies false) \Leftrightarrow true)$

\medskip
\noindent
\hangindent=2em
(c) $true(as(false \implies true) \Leftrightarrow true)$

\medskip
\noindent
\hangindent=2em
(d) $true(as(false \implies false) \Leftrightarrow true)$

\bigskip

(Assuming that pigs can't fly . . . )

\bigskip

\medskip

\bigskip
\noindent
\textbf{Solution 2}

\medskip

(a)
\par
\vspace{11pt}
\begin{tabular}{c|c|c|c}
$p$ & $q$ & $p \land q$ & $\mathbf{( p \land q ) \Rightarrow p}$ \\
\hline
\textit{t} & \textit{t} & \textit{t} & \textbf{t} \\
\textit{t} & \textit{f} & \textit{f} & \textbf{t} \\
\textit{f} & \textit{t} & \textit{f} & \textbf{t} \\
\textit{f} & \textit{f} & \textit{f} & \textbf{t} \\
\end{tabular}

\medskip

(b)
\par
\vspace{11pt}
\begin{tabular}{c|c|c|c|c|c}
$p$ & $q$ & $p \land q$ & $\lnot p$ & $\lnot p \Rightarrow ( p \land q )$ & $\mathbf{( \lnot p \Rightarrow ( p \land q ) ) \Leftrightarrow p}$ \\
\hline
\textit{t} & \textit{t} & \textit{t} & \textit{f} & \textit{t} & \textbf{t} \\
\textit{t} & \textit{f} & \textit{f} & \textit{f} & \textit{t} & \textbf{t} \\
\textit{f} & \textit{t} & \textit{f} & \textit{t} & \textit{f} & \textbf{t} \\
\textit{f} & \textit{f} & \textit{f} & \textit{t} & \textit{f} & \textbf{t} \\
\end{tabular}

\medskip

(c)
\par
\vspace{11pt}
\begin{tabular}{c|c|c|c|c}
$p$ & $q$ & $p \Rightarrow q$ & $p \land ( p \Rightarrow q )$ & $\mathbf{( p \land ( p \Rightarrow q ) ) \Rightarrow q}$ \\
\hline
\textit{t} & \textit{t} & \textit{t} & \textit{t} & \textbf{t} \\
\textit{t} & \textit{f} & \textit{f} & \textit{f} & \textbf{t} \\
\textit{f} & \textit{t} & \textit{t} & \textit{f} & \textbf{t} \\
\textit{f} & \textit{f} & \textit{t} & \textit{f} & \textbf{t} \\
\end{tabular}

\medskip

\bigskip
\noindent
\textbf{Solution 3}

\medskip

(a)
\par
\vspace{11pt}
\[
\begin{array}{ll@{\hspace{2em}}l}
& p \implies \lnot p \\
&\Leftrightarrow \lnot p \lor \lnot p & [\mbox{$\Rightarrow$}] \\
&\Leftrightarrow \lnot p & [\mbox{idempotence}]
\end{array}
\]

\medskip

(b)
\par
\vspace{11pt}
\[
\begin{array}{ll@{\hspace{2em}}l}
& \lnot p \implies p \\
&\Leftrightarrow \lnot \lnot p \lor p & [\mbox{$\Rightarrow$}] \\
&\Leftrightarrow p \lor p & [\mbox{$\lnot$ $\lnot$}] \\
&\Leftrightarrow p & [\mbox{idempotence}]
\end{array}
\]

\medskip

(c)
\par
\vspace{11pt}
\[
\begin{array}{ll@{\hspace{2em}}l}
& p \implies (q \implies r) \\
&\Leftrightarrow \lnot p \lor (q \implies r) & [\mbox{$\Rightarrow$}] \\
&\Leftrightarrow \lnot p \lor \lnot q \lor r & [\mbox{$\Rightarrow$}] \\
&\Leftrightarrow \lnot p \lor \lnot q \lor r & [\mbox{associativity}] \\
&\Leftrightarrow \lnot (p \land q) \lor r & [\mbox{De Morgan}] \\
&\Leftrightarrow p \land q \implies r & [\mbox{$\Rightarrow$}]
\end{array}
\]

\medskip

(d)
\par
\vspace{11pt}
\[
\begin{array}{ll@{\hspace{2em}}l}
& q \implies (p \implies r) \\
&\Leftrightarrow \lnot q \lor (p \implies r) & [\mbox{$\Rightarrow$}] \\
&\Leftrightarrow \lnot q \lor \lnot p \lor r & [\mbox{$\Rightarrow$}] \\
&\Leftrightarrow \lnot p \lor \lnot q \lor r & [\mbox{associativity $\land$ commutativity}] \\
&\Leftrightarrow \lnot p \lor (q \implies r) & [\mbox{$\Rightarrow$}] \\
&\Leftrightarrow p \implies (q \implies r) & [\mbox{$\Rightarrow$}]
\end{array}
\]

\medskip

(e)
\par
\vspace{11pt}
\[
\begin{array}{ll@{\hspace{2em}}l}
& p \land q \Leftrightarrow p \\
&\Leftrightarrow (p \land q \implies p) \land (p \implies p \land q) & [\mbox{$\Leftrightarrow$}] \\
&\Leftrightarrow (\lnot (p \land q) \lor p) \land (\lnot p \lor p \land q) & [\mbox{$\Rightarrow$}] \\
&\Leftrightarrow (\lnot p \lor \lnot q \lor p) \land (\lnot p \lor p \land q) & [\mbox{De Morgan}] \\
&\Leftrightarrow (\lnot q \lor \lnot p \lor p) \land (\lnot p \lor p \land q) & [\mbox{associativity $\land$ comm .}] \\
&\Leftrightarrow (\lnot q \lor true) \land (\lnot p \lor p \land q) & [\mbox{excluded middle}] \\
&\Leftrightarrow true \land (\lnot p \lor p \land q) & [\mbox{$\lor$ $\land$ true}] \\
&\Leftrightarrow \lnot p \lor p \land q & [\mbox{$\land$ $\land$ true}] \\
&\Leftrightarrow (\lnot p \lor p) \land (\lnot p \lor q) & [\mbox{distribution}] \\
&\Leftrightarrow true \land (\lnot p \lor q) & [\mbox{excluded middle}] \\
&\Leftrightarrow \lnot p \lor q & [\mbox{$\land$ $\land$ true}] \\
&\Leftrightarrow p \implies q & [\mbox{$\Rightarrow$}]
\end{array}
\]

\medskip

(f)
\par
\vspace{11pt}
\[
\begin{array}{ll@{\hspace{2em}}l}
& p \lor q \Leftrightarrow p \\
&\Leftrightarrow (p \lor q \implies p) \land (p \implies p \lor q) & [\mbox{$\Leftrightarrow$}] \\
&\Leftrightarrow (\lnot (p \lor q) \lor p) \land (\lnot p \lor p \lor q) & [\mbox{$\Rightarrow$}] \\
&\Leftrightarrow (\lnot p \land \lnot q \lor p) \land (\lnot p \lor p \lor q) & [\mbox{De Morgan}] \\
&\Leftrightarrow (\lnot p \lor p) \land (\lnot q \lor p) \land (\lnot p \lor p \lor q) & [\mbox{distribution}] \\
&\Leftrightarrow true \land (\lnot q \lor p) \land (\lnot p \lor p \lor q) & [\mbox{excluded middle}] \\
&\Leftrightarrow (\lnot q \lor p) \land (\lnot p \lor p \lor q) & [\mbox{$\land$ $\land$ true}] \\
&\Leftrightarrow (\lnot q \lor p) \land (\lnot p \lor p \lor q) & [\mbox{associativity}] \\
&\Leftrightarrow (\lnot q \lor p) \land (true \lor q) & [\mbox{excluded middle}] \\
&\Leftrightarrow (\lnot q \lor p) \land true & [\mbox{$\lor$ $\land$ true}] \\
&\Leftrightarrow \lnot q \lor p & [\mbox{$\land$ $\land$ true}] \\
&\Leftrightarrow q \implies p & [\mbox{$\Rightarrow$}]
\end{array}
\]

\medskip

\bigskip
\noindent
\textbf{Solution 4}

\medskip

\noindent
\hangindent=2em
(a) $p \lor q$ $\Leftrightarrow$ $(\lnot p \lor \lnot q) \land q$ is $\lnot a$ tautology. You might illustrate this via a truth table or via a chain of equivalences, showing that the proposition is not equivalent to true. Alternatively, you might try and find a combination of values for which the proposition is false. (In this case, the proposition is false when p and q are both true.)

\bigskip


\medskip
\noindent
\hangindent=2em
(b) $p \lor q$ $\Leftrightarrow$ $\lnot p \land \lnot q \lor q$ is $\lnot a$ tautology. In this case, the proposition is false when p is true and q is false.

\bigskip


\medskip
\bigskip
\noindent
\textbf{Solution 5}

\medskip

\noindent
\hangindent=2em
(a) $\exists d : Dog @ gentle(d) \land well\_trained(d)$

\medskip
\noindent
\hangindent=2em
(b) $\forall d : Dog @ neat(d) \land well\_trained(d) \implies attractive(d)$

\medskip
\noindent
\hangindent=2em
(c) $\exists d : Dog @ gentle(d) \implies (\forall t : Trainer @ groomed(d, t))$

\medskip
\bigskip
\noindent
\textbf{Solution 6}

\medskip

\noindent
\hangindent=2em
(a) This is a true proposition: whatever the value of x, the expression $x^2$ - x + 1 denotes a natural number. If we choose y to be this natural number, we will find that p is true.

\bigskip


\medskip
\noindent
\hangindent=2em
(b) This is a false proposition. We cannot choose a large enough value for y such that p will hold for any value of x.

\bigskip


\medskip
\noindent
\hangindent=2em
(c) This is a false proposition. It is an implication whose antecedent part is true and whose consequent part is false.

\bigskip


\medskip
\noindent
\hangindent=2em
(d) This is a true proposition. It is an implication whose antecedent part is false and whose consequent part is true.

\bigskip


\medskip
\bigskip
\noindent
\textbf{Solution 7}

\medskip

\noindent
\hangindent=2em
(a) We must define a predicate p that is false for at least one value of x, and is true for at least one other value. A suitable solution would be $p \Leftrightarrow x > 1$.

\bigskip


\medskip
\noindent
\hangindent=2em
(b) With the above choice of p, we require only that q is sometimes false when p is true (for else the universal quantification would hold). A suitable solution would be $q \Leftrightarrow x > 3$.

\bigskip


\medskip
\bigskip
\noindent
\textbf{Solution 8}

\medskip

\noindent
\hangindent=2em
(a) $\forall x : \nat @ x \geq z$

\medskip
\noindent
\hangindent=2em
(b) $\forall z : \nat @ z \geq x + y$

\medskip
\noindent
\hangindent=2em
(c) $x + 3 > 0 \land (\forall z : \nat @ z \geq x + 3)$

\medskip
\section*{Equality}

\bigskip
\noindent
\textbf{Solution 9}

\medskip

(a)
\par
\vspace{11pt}
\[
\begin{array}{ll@{\hspace{2em}}l}
& \exists y : \nat @ y \in \{0, 1\} \land y \neq 1 \land x \neq y \\
&\Leftrightarrow \exists y : \nat @ y = 0 \land x \neq y & [\mbox{arithmetic}] \\
&\Leftrightarrow 0 \in \nat \land x \neq 0 & [\mbox{one - point rule}] \\
&\Leftrightarrow x \neq 0
\end{array}
\]

\medskip

(b)
\par
\vspace{11pt}
\[
\begin{array}{ll@{\hspace{2em}}l}
& \exists x, y : \nat @ x + y = 4 \land x < y \\
&\Leftrightarrow \exists x, y : \nat @ y = 4 - x \land x < y \\
&\Leftrightarrow \exists x : \nat @ 4 - x \in \nat \land x < 4 - x \\
&\Leftrightarrow true
\end{array}
\]

\bigskip

The final equivalence holds because $0 \in N$, $4 - 0 \in N$, and 0 $<$ 4.

\bigskip

\medskip

(c)
\par
\vspace{11pt}
\[
\begin{array}{ll@{\hspace{2em}}l}
& \forall x : \nat @ \exists y : \nat @ x = y + 1 \\
&\Leftrightarrow \forall x : \nat @ \exists y : \nat @ y = x - 1 \\
&\Leftrightarrow \forall x : \nat @ x - 1 \in \nat
\end{array}
\]

\bigskip

The final equivalence holds because $0 \in N$ and yet $0 - 1 \notin N$. We may assume that the subtraction operator is defined for all integers.

\bigskip

\medskip

(d)
\par
\vspace{11pt}
\[
\begin{array}{ll@{\hspace{2em}}l}
& \exists x : \nat @ x = 1 \land x > y \lor x = 2 \land x > z \\
&\Leftrightarrow (\exists x : \nat @ x = 1 \land x > y) \lor (\exists x : \nat @ x = 2 \land x > z) \\
&\Leftrightarrow 1 \in \nat \land 1 > y \lor (\exists x : \nat @ x = 2 \land x > z) \\
&\Leftrightarrow 1 \in \nat \land 1 > y \lor 2 \in \nat \land 2 > z \\
&\Leftrightarrow 1 > y \lor 2 > z
\end{array}
\]

\medskip

\bigskip
\noindent
\textbf{Solution 10}

\medskip

\bigskip

As discussed, the quantifier $\exists_1$ can help give rise to a 'test' or 'precondition' to ensure that an application of mu will work.

\bigskip

\bigskip

So, as a simple example, as the proposition

\bigskip

\noindent
$\exists_1 n : \nat @ \forall m : \nat @ n \leq m$


\bigskip

is equivalent to true, we can be certain that the statement

\bigskip

\noindent
$(\mu n : \nat | \forall m : \nat @ n \leq m)$


\bigskip

will return a result (which happens to be 0).

\bigskip

\bigskip
\noindent
\textbf{Solution 11}

\medskip

\noindent
\hangindent=2em
(a) $(\mu a : \nat | a = a) = 0$

\bigskip

is a provable statement, since 0 is the only natural number with the specified property.

\bigskip

\medskip

\noindent
\hangindent=2em
(b) $(\mu b : \nat | b = b) = 1$

\bigskip

is not provable. The specified property is true of both 0 and 1, and thus the value of the mu-expression is undefined.

\bigskip

\medskip

\noindent
\hangindent=2em
(c) $(\mu c : \nat | c > c) = (\mu c : \nat | c > c)$

\bigskip

is a provable statement. Neither expression is properly defined, but we may conclude that they are equal; there is little else that we can prove about them.

\bigskip

\medskip

\noindent
\hangindent=2em
(d) $(\mu d : \nat | d = d) = 1$

\bigskip

is $\lnot a$ provable statement. We cannot confirm that 1 is the only natural number with the specified property; we do not know what value is taken by undefined operations.

\bigskip

\medskip

\bigskip
\noindent
\textbf{Solution 12}

\medskip

\bigskip

(Requires mu-operator with expression part - not yet implemented)

\bigskip

\noindent
\hangindent=2em
(a) $(\mu m : Mountain | \forall n : Mountain @ height(n) \leq height(m) @ height(m))$

\medskip
\noindent
\hangindent=2em
(b) $(\mu c : Chapter | \exists_1 d : Chapter @ length(d) > length(c) @ length(c))$

\medskip
\noindent
\hangindent=2em
(c) Assuming the existence of a suitable function, max: ($(\mu n : \nat | n = max)$($\{ m : \nat | 8 * m < 100.8 * m \}$) . 100 - n)

\bigskip


\medskip
\section*{Deductive proofs}

\bigskip
\noindent
\textbf{Solution 13}

\medskip

\noindent
$\displaystyle
\infer[\Rightarrow\textrm{-intro}^{[1]}]{p \land (p \implies q) \implies p \land q}{
  \ulcorner p \land (p \implies q) \urcorner^{[1]}
&
\infer[\land \mathrm{intro}]{p \land q}{
  \infer[\land\textrm{-elim}^{[1]}]{p}{
  \infer[\mathrm{from} above]{p \land (p \implies q)}{}
}
&
\infer[\Rightarrow \mathrm{elim}]{q}{
  \infer[\land\textrm{-elim}^{[1]}]{p \implies q}{
  \infer[\mathrm{from} above]{p \land (p \implies q)}{}
}
&
\infer[\land\textrm{-elim}^{[1]}]{p}{
  \infer[\mathrm{from} above]{p \land (p \implies q)}{}
}
}
}
}
$

\bigskip
\noindent
\textbf{Solution 14}

\medskip

\bigskip

In one direction:

\bigskip

\noindent
$\displaystyle
\infer[\Rightarrow\textrm{-intro}^{[1]}]{(p \land q \Leftrightarrow p) \implies (p \implies q)}{
  \ulcorner p \land q \Leftrightarrow p \urcorner^{[1]}
&
\infer[\Rightarrow\textrm{-intro}^{[2]}]{p \implies q}{
  \ulcorner p \urcorner^{[2]}
&
\infer[\land\textrm{-elim}^{[3]}]{q}{
  \infer[\Rightarrow \mathrm{elim} \mathrm{from} 1 \land 2]{p \land q}{
  \infer[derived]{p \land q}{}
}
}
}
}
$

\bigskip

and the other:

\bigskip

\noindent
$\displaystyle
\infer[\Rightarrow\textrm{-intro}^{[1]}]{(p \implies q) \implies (p \land q \Leftrightarrow p)}{
  \ulcorner p \implies q \urcorner^{[1]}
&
\infer[\Leftrightarrow \mathrm{intro}]{p \land q \Leftrightarrow p}{
  \infer[\Rightarrow\textrm{-intro}^{[2]}]{p \land q \implies p}{
  \ulcorner p \land q \urcorner^{[2]}
&
\infer[\land\textrm{-elim}^{[2]}]{p}{}
}
&
\infer[\Rightarrow\textrm{-intro}^{[3]}]{p \implies p \land q}{
  \ulcorner p \urcorner^{[3]}
&
\infer[\land \mathrm{intro} \mathrm{from} 1 \land 3]{p \land q}{}
}
}
}
$

\bigskip

We can then combine these two proofs $with \Leftrightarrow intro$.

\bigskip

\bigskip
\noindent
\textbf{Solution 15}

\medskip

\noindent
$\displaystyle
\infer[\Rightarrow\textrm{-intro}^{[1]}]{(p \implies q) \land \lnot q \implies \lnot p}{
  \ulcorner (p \implies q) \land \lnot q \urcorner^{[1]}
&
\infer[false\textrm{-elim}^{[2]}]{\lnot p}{
  \ulcorner p \urcorner^{[2]}
&
\infer[\mathrm{false} \mathrm{intro}]{false}{
  \infer[\Rightarrow \mathrm{elim}]{q}{
  \infer[\land\textrm{-elim}^{[1]}]{p \implies q}{}
&
\ulcorner p \urcorner^{[2]}
}
&
\infer[\land\textrm{-elim}^{[1]}]{\lnot q}{}
}
}
}
$

\bigskip
\noindent
\textbf{Solution 16}

\medskip

\bigskip

In one direction:

\bigskip

\noindent
$\displaystyle
\infer[\Rightarrow\textrm{-intro}^{[1]}]{p \land (q \lor r) \implies p \land q \lor p \land r}{
  \ulcorner p \land (q \lor r) \urcorner^{[1]}
&
\infer[\lor\textrm{-elim}^{[2]}]{p \land q \lor p \land r}{
  \ulcorner q \lor r \urcorner^{[1]}
&
\raiseproof{10ex}{\infer[\lor \mathrm{intro}]{p \land q \lor p \land r}{\infer[\land \mathrm{intro}]{p \land q}{
  \infer[\land\textrm{-elim}^{[1]}]{p}{}
&
\infer[\mathrm{case} \mathrm{assumption}]{q}{}
}}}
&
\hskip 6em \raiseproof{26ex}{\infer[\lor \mathrm{intro}]{p \land q \lor p \land r}{\infer[\land \mathrm{intro}]{p \land r}{
  \infer[\land\textrm{-elim}^{[1]}]{p}{}
&
\infer[\mathrm{case} \mathrm{assumption}]{r}{}
}}}
}
}
$

\bigskip

In the other:

\bigskip

\noindent
$\displaystyle
\infer[\Rightarrow\textrm{-intro}^{[3]}]{p \land q \lor p \land r \implies p \land (q \lor r)}{
  \ulcorner p \land q \lor p \land r \urcorner^{[3]}
&
\infer[\lor\textrm{-elim}^{[4]}]{p \land (q \lor r)}{
  \ulcorner case1 \lor case2 \urcorner^{[3]}
&
\raiseproof{8ex}{\infer[\land \mathrm{intro}]{p \land (q \lor r)}{\infer[\land \mathrm{elim}]{p}{} & \infer[\lor \mathrm{intro}]{q \lor r}{}}}
&
\hskip 6em \raiseproof{22ex}{\infer[\land \mathrm{intro}]{p \land (q \lor r)}{\infer[\land \mathrm{elim}]{p}{} & \infer[\lor \mathrm{intro}]{q \lor r}{}}}
}
}
$

\bigskip
\noindent
\textbf{Solution 17}

\medskip

\bigskip

In one direction:

\bigskip

\noindent
$\displaystyle
\infer[\Rightarrow\textrm{-intro}^{[3]}]{p \lor q \land r \implies (p \lor q) \land (p \lor r)}{
  \ulcorner p \lor q \land r \urcorner^{[3]}
&
\infer[\lor \mathrm{elim} \land \land \mathrm{intro}]{(p \lor q) \land (p \lor r)}{}
}
$

\bigskip

and the other:

\bigskip

\noindent
$\displaystyle
\infer[\Rightarrow\textrm{-intro}^{[1]}]{(p \lor q) \land (p \lor r) \implies p \lor q \land r}{
  \ulcorner (p \lor q) \land (p \lor r) \urcorner^{[1]}
&
\infer[\lor \mathrm{elim} \mathrm{from} 2 \land 3]{p \lor q \land r}{}
}
$

\bigskip
\noindent
\textbf{Solution 18}

\medskip

\bigskip

In one direction:

\bigskip

\noindent
$\displaystyle
\infer[\Rightarrow\textrm{-intro}^{[1]}]{(p \implies q) \implies \lnot p \lor q}{
  \ulcorner p \implies q \urcorner^{[1]}
&
\infer[\lor \mathrm{elim} \mathrm{from} \mathrm{excluded} \mathrm{middle}]{\lnot p \lor q}{}
}
$

\bigskip

and the other:

\bigskip

\noindent
$\displaystyle
\infer[\Rightarrow\textrm{-intro}^{[3]}]{\lnot p \lor q \implies (p \implies q)}{
  \ulcorner \lnot p \lor q \urcorner^{[3]}
&
\infer[\Rightarrow\textrm{-intro}^{[4]}]{p \implies q}{
  \ulcorner p \urcorner^{[4]}
&
\infer[\lor elim \land false\textrm{-elim}^{[3]}]{q}{}
}
}
$

\section*{Sets and types}

\bigskip
\noindent
\textbf{Solution 19}

\medskip

\noindent
\hangindent=2em
(a) 1 in $\{4, 3, 2, 1\}$ is true.

\bigskip


\medskip
\noindent
\hangindent=2em
(b) $\{1\}$ in $\{1, 2, 3, 4\}$ is undefined.

\bigskip


\medskip
\noindent
\hangindent=2em
(c) $\{1\}$ in $\{\{1\}, \{2\}, \{3\}, \{4\}\}$ is true.

\bigskip


\medskip
\noindent
\hangindent=2em
(d) The empty set in $\{1, 2, 3, 4\}$ is undefined.

\bigskip


\medskip
\bigskip
\noindent
\textbf{Solution 20}

\medskip

\noindent
\hangindent=2em
(a) $\{1\} \cross \{2, 3\}$

\bigskip

is the set $\{(1, 2), (1, 3)\}$

\bigskip

\medskip

\noindent
\hangindent=2em
(b) The empty set cross $\{2, 3\}$ is the empty set

\bigskip


\medskip
\noindent
\hangindent=2em
(c) $\power \emptyset \cross \{1\}$

\bigskip

is the set $\{(\emptyset, 1)\}$

\bigskip

\medskip

\noindent
\hangindent=2em
(d) $\{(1, 2)\}$ cross $\{3, 4\}$ is the set $\{((1, 2), 3), ((1, 2), 4)\}$

\bigskip


\medskip
\bigskip
\noindent
\textbf{Solution 21}

\medskip

\bigskip

There are various ways of describing these sets via set comprehensions. Examples are given below.

\bigskip

\noindent
\hangindent=2em
(a) $\{ z : \num | 0 \leq z \land z \leq 100 \}$

\medskip
\noindent
\hangindent=2em
(b) $\{ z : \num | z = 10 \}$

\medskip
\noindent
\hangindent=2em
(c) $\{ z : \num | z \mod 2 = 0 \lor z \mod 3 = 0 \lor z \mod 5 = 0 \}$

\medskip
\bigskip
\noindent
\textbf{Solution 22}

\medskip

\noindent
\hangindent=2em
(a) $\{ n : \nat | n \leq 4 @ n^2 \}$

\medskip
\noindent
\hangindent=2em
(b) $\{ n : \nat | n \leq 4 @ (n, n^2) \}$

\medskip
\noindent
\hangindent=2em
(c) $\{ n : \power \{0, 1\} \}$

\medskip
\noindent
\hangindent=2em
(d) $\{ n : \power \{0, 1\} | true @ (n, \# n) \}$

\medskip
\bigskip
\noindent
\textbf{Solution 23}

\medskip

(a)
\par
\vspace{11pt}
\[
\begin{array}{ll@{\hspace{2em}}l}
& x \in a \cap a \\
&\Leftrightarrow x \in a \land x \in a \\
&\Leftrightarrow x \in a
\end{array}
\]

\medskip

(b)
\par
\vspace{11pt}
\[
\begin{array}{ll@{\hspace{2em}}l}
& x \in a \cup a \\
&\Leftrightarrow x \in a \lor x \in a \\
&\Leftrightarrow x \in a
\end{array}
\]

\medskip

\bigskip
\noindent
\textbf{Solution 24}

\medskip

(a)
\par
\vspace{11pt}
\bigskip

The set of all pairs of integers is Z cross Z. To give it a name, we could write:

\bigskip

\begin{zed}
Pairs == \num \cross \num
\end{zed}

\medskip

(b)
\par
\vspace{11pt}
\bigskip

The set of all integer pairs in which each element is strictly greater than zero could be defined by:

\bigskip

\begin{zed}
StrictlyPositivePairs == \{ m, n : \num | m > 0 \land n > 0 @ (m, n) \}
\end{zed}

\medskip

(c)
\par
\vspace{11pt}
\bigskip

It is intuitive to use a singular noun for the name of a basic type; we define the set of all people by writing:

\bigskip

\begin{zed}[Person]\end{zed}

\medskip

(d)
\par
\vspace{11pt}
\bigskip

The set of all couples could be defined by:

\bigskip

\begin{zed}
Couples == \{ s : \power Person | \# s = 2 \}
\end{zed}

\medskip

\bigskip
\noindent
\textbf{Solution 25}

\medskip

\bigskip

$Requires(generic)(set)(notation) \land Cartesian(product)$

\bigskip

\bigskip
\noindent
\textbf{Solution 26}

\medskip

\bigskip

$Requires(generic)(parameters) \land relation(type)(notation)$

\bigskip

\section*{Relations}

\bigskip
\noindent
\textbf{Solution 27}

\medskip

(a)
\par
\vspace{11pt}
\bigskip

The power set of $\{(0, 0), (0, 1), (1, 0), (1, 1)\}$ is:

\bigskip

\noindent
$\{\emptyset, \{(0, 0)\}, \{(0, 1)\}, \{(1, 0)\}, \{(1, 1)\}, \{(1, 0), (1, 1)\}, \{(0, 0), (0, 1)\}, \{(0, 1), (1, 1)\}, \{(0, 1), (1, 0)\}, \{(0, 0), (1, 1)\}, \{(0, 0), (1, 0)\}, \{(0, 0), (1, 0), (1, 1)\}, \{(0, 0), (0, 1), (1, 1)\}, \{(0, 0), (0, 1), (1, 0)\}, \{(0, 1), (1, 0), (1, 1)\}, \{(0, 0), (0, 1), (1, 0), (1, 1)\}\}$


\medskip

\noindent
\hangindent=2em
(b) $\{\emptyset, \{(0, 0)\}, \{(0, 1)\}, \{(0, 0), (0, 1)\}\}$

\medskip
\noindent
\hangindent=2em
(c) $\{\emptyset\}$

\medskip
\noindent
\hangindent=2em
(d) $\{\emptyset\}$

\medskip
\bigskip
\noindent
\textbf{Solution 28}

\medskip

\noindent
\hangindent=2em
(a) $\dom R = \{0, 1, 2\}$

\medskip
\noindent
\hangindent=2em
(b) $\ran R = \{1, 2, 3\}$

\medskip
\noindent
\hangindent=2em
(c) $\{1, 2\} \dres R = \{1 \mapsto 2, 1 \mapsto 3, 2 \mapsto 3\}$

\medskip
\bigskip
\noindent
\textbf{Solution 29}

\medskip

\noindent
\hangindent=2em
(a) $\{2 \mapsto 4, 3 \mapsto 3, 3 \mapsto 4, 4 \mapsto 2\}$

\medskip
\noindent
\hangindent=2em
(b) $\{1 \mapsto 3, 2 \mapsto 2, 2 \mapsto 3, 3 \mapsto 1\}$

\medskip
\noindent
\hangindent=2em
(c) $\{1 \mapsto 1, 2 \mapsto 2, 2 \mapsto 3, 3 \mapsto 2, 3 \mapsto 3, 4 \mapsto 4\}$

\medskip
\noindent
\hangindent=2em
(d) $\{1 \mapsto 4, 2 \mapsto 2, 2 \mapsto 3, 3 \mapsto 2, 3 \mapsto 3, 4 \mapsto 1\}$

\medskip
\bigskip
\noindent
\textbf{Solution 30}

\medskip

\begin{axdef}
childOf : Person \rel Person
\end{axdef}

(a)
\par
\vspace{11pt}
\begin{zed}
parentOf == childOf^{-1}
\end{zed}

\bigskip

This is a good example of how there are many different ways of writing the same thing. An alternative abbreviation would $be : parentOf $== $\{ x, y : Person | x \mapsto y \in childOf @ y \mapsto x \}$. Or via an axiomatic definition with $parentOf : Person <\fun Person and where clause parentOf $= childOf~.

\bigskip

\medskip

(b)
\par
\vspace{11pt}
\begin{zed}
siblingOf == (childOf \circ parentOf) \setminus \id[Person]
\end{zed}

\medskip

(c)
\par
\vspace{11pt}
\begin{zed}
cousinOf == childOf \circ siblingOf \circ parentOf
\end{zed}

\medskip

(d)
\par
\vspace{11pt}
\begin{zed}
ancestorOf == parentOf^+
\end{zed}

\medskip

\bigskip
\noindent
\textbf{Solution 31}

\medskip

\bigskip

(Requires compound identifiers with operators - R+, R*)

\bigskip

(a)
\par
\vspace{11pt}
\begin{zed}
R == \{ a, b : \nat | b = a \lor b = a \}
\end{zed}

\medskip

(b)
\par
\vspace{11pt}
\begin{zed}
S == \{ a, b : \nat | b = a \lor b = a \}
\end{zed}

\medskip

\noindent
\hangindent=2em
(c) R+ == $\{ a, b : \nat | b > a \}$

\bigskip


\medskip
\noindent
\hangindent=2em
(d) R* == $\{ a, b : \nat | b \geq a \}$

\bigskip


\medskip
\bigskip
\noindent
\textbf{Solution 32}

\medskip

(a)
\par
\vspace{11pt}
\[
\begin{array}{ll@{\hspace{2em}}l}
& x \mapsto y \in A \dres B \dres R \\
&\Leftrightarrow x \in A \land x \mapsto y \in (B \dres R) \\
&\Leftrightarrow x \in A \land x \in B \land x \mapsto y \in R \\
&\Leftrightarrow x \in A \cap B \land x \mapsto y \in R \\
&\Leftrightarrow x \mapsto y \in A \cap B \dres R
\end{array}
\]

\medskip

(b)
\par
\vspace{11pt}
\[
\begin{array}{ll@{\hspace{2em}}l}
& x \mapsto y \in R \cup S \rres C \\
&\Leftrightarrow x \mapsto y \in R \cup S \land y \in C \\
&\Leftrightarrow (x \mapsto y \in R \lor x \mapsto y \in S) \land y \in C \\
&\Leftrightarrow x \mapsto y \in R \land y \in C \lor x \mapsto y \in S \land y \in C \\
&\Leftrightarrow x \mapsto y \in R \rres C \lor x \mapsto y \in S \rres C \\
&\Leftrightarrow x \mapsto y \in (R \rres C) \cup (S \rres C)
\end{array}
\]

\medskip

\section*{Functions}

\bigskip
\noindent
\textbf{Solution 33}

\medskip

\bigskip

The set of 9 functions:

\bigskip

\noindent
$\{\emptyset, \{(0, 0)\}, \{(0, 1)\}, \{(1, 1)\}, \{(1, 0)\}, \{(0, 0), (1, 1)\}, \{(0, 1), (1, 1)\}, \{(1, 0), (0, 0)\}, \{(0, 1), (1, 0)\}\}$


(a)
\par
\vspace{11pt}
\bigskip

The set of total functions:

\bigskip

\noindent
$\{\{(0, 0), (1, 1)\}, \{(0, 1), (1, 1)\}, \{(1, 0), (0, 0)\}, \{(0, 1), (1, 0)\}\}$


\medskip

(b)
\par
\vspace{11pt}
\bigskip

The set of functions which are neither injective nor surjective:

\bigskip

\noindent
$\{\{(0, 1), (1, 1)\}, \{(0, 0), (1, 0)\}\}$


\medskip

(c)
\par
\vspace{11pt}
\bigskip

The set of functions which are injective but not surjective:

\bigskip

\noindent
$\{\emptyset, \{(0, 0)\}, \{(0, 1)\}, \{(1, 0)\}, \{(1, 1)\}\}$


\medskip

\noindent
\hangindent=2em
(d) There are no functions (of this type) which are surjective but not injective.

\bigskip


\medskip
(e)
\par
\vspace{11pt}
\bigskip

The set of bijective functions:

\bigskip

\noindent
$\{\{(0, 0), (1, 1)\}, \{(0, 1), (1, 0)\}\}$


\medskip

\bigskip
\noindent
\textbf{Solution 34}

\medskip

\noindent
\hangindent=2em
(a) $\{1 \mapsto a, 2 \mapsto b, 3 \mapsto c, 4 \mapsto b\}$

\medskip
\noindent
\hangindent=2em
(b) $\{1 \mapsto c, 2 \mapsto b, 3 \mapsto c, 4 \mapsto d\}$

\medskip
\noindent
\hangindent=2em
(c) $\{1 \mapsto c, 2 \mapsto b, 3 \mapsto c, 4 \mapsto b\}$

\medskip
\noindent
\hangindent=2em
(d) $\{1 \mapsto c, 2 \mapsto b, 3 \mapsto c, 4 \mapsto b\}$

\medskip
\bigskip
\noindent
\textbf{Solution 35}

\medskip

\bigskip

$Requires(power)(set)(notation)(\power) \land relational(image)$

\bigskip

(a)
\par
\vspace{11pt}
\begin{axdef}
children : Person \fun \power Person
\where
children = \{ p : Person @ p \mapsto (parentOf \limg \{p\} \rimg) \}
\end{axdef}

\medskip

(b)
\par
\vspace{11pt}
\begin{axdef}
number\_of\_grandchildren : Person \fun \nat
\where
number\_of\_grandchildren = \{ p : Person @ p \mapsto \# (parentOf \circ parentOf \limg \{p\} \rimg) \}
\end{axdef}

\medskip

\bigskip
\noindent
\textbf{Solution 36}

\medskip

\bigskip

(Note: This solution demonstrates relation types in quantifier domains)

\bigskip

\begin{axdef}
number\_of\_drivers : Drivers \rel Cars \fun (Cars \fun \nat)
\where
number\_of\_drivers = \lambda r : Drivers \rel Cars \bullet \{ c : \ran r @ c \mapsto \# \{ d : Drivers | d \mapsto c \in r \} \}
\end{axdef}

\section*{Sequences}

\bigskip
\noindent
\textbf{Solution 37}

\medskip

\noindent
\hangindent=2em
(a) $\langle a \rangle$

\medskip
\noindent
\hangindent=2em
(b) $\{1 \mapsto a, 2 \mapsto b, 2 \mapsto a, 3 \mapsto c, 3 \mapsto b, 4 \mapsto d\}$

\medskip
\noindent
\hangindent=2em
(c) $\{2 \mapsto b, 3 \mapsto c, 4 \mapsto d\}$

\medskip
\noindent
\hangindent=2em
(d) $\{1, 2, 3, 4\}$

\medskip
\noindent
\hangindent=2em
(e) $\{a, b\}$

\medskip
\noindent
\hangindent=2em
(f) $\{a \mapsto 1, b \mapsto 2, c \mapsto 3, d \mapsto 4\}$

\medskip
\noindent
\hangindent=2em
(g) $\langle a, b \rangle$

\medskip
\noindent
\hangindent=2em
(h) $\{3 \mapsto b\}$

\medskip
\noindent
\hangindent=2em
(i) $\{a\}$

\medskip
\noindent
\hangindent=2em
(j) $c$

\medskip
\bigskip
\noindent
\textbf{Solution 38}

\medskip

(a)
\par
\vspace{11pt}
\begin{axdef}
f : Place \fun \power Place
\where
\forall p : Place @ f(p) = \{ q : Place | p \mapsto q \in \ran trains \}
\end{axdef}

\medskip

\noindent
\hangindent=2em
(b) $\{ p : Place | \exists_1 x : \dom trains @ trains(x).2 = p \}$

\medskip
\noindent
\hangindent=2em
(c) $(\mu p : Place | \forall q : Place @ p \neq q \land \# \{ x : \dom trains | trains(x).2 = p \} > \# \{ x : \dom trains | trains(x).2 = q \})$

\medskip
\bigskip
\noindent
\textbf{Solution 39}

\medskip

(a)
\par
\vspace{11pt}
\bigskip

$large\_coins : Collection \fun N$

\bigskip

\bigskip

$\forall c : Collection @ large\_coins(c) = c(large)$

\bigskip

\bigskip

(Blocked by: underscore in identifier for fuzz compatibility)

\bigskip

\medskip

(b)
\par
\vspace{11pt}
\bigskip

$add\_coin : Collection * Coin \fun Collection$

\bigskip

\bigskip

$\forall c : Collection @ \forall d : Coin @ add\_coin(c, d) = c \cup \lbag d \rbag$

\bigskip

\bigskip

(Blocked by: underscore in identifier and bag union)

\bigskip

\medskip

\section*{Modelling}

\bigskip

Solutions 40-52 are work in progress - many require features not yet implemented

\bigskip

\bigskip
\noindent
\textbf{Solution 40}

\medskip

\bigskip

(Work in progress - requires semicolon-separated bindings in set comprehensions)

\bigskip

(a)
\par
\vspace{11pt}
\bigskip

$hd : seq(Title * Length * Viewed)$

\bigskip

\bigskip

$cumulative\_total(hd) \leq 12000$

\bigskip

\bigskip

$\forall p : \ran hd @ p.2 \leq 360$

\bigskip

\bigskip

Note that cumulative_total is defined in part (d).

\bigskip

\medskip

\noindent
\hangindent=2em
(b) $\{ p : \ran hd | p.2 > 120 @ p.1 \}$

\medskip
(c)
\par
\vspace{11pt}
\bigskip

These can be defined recursively:

\bigskip

\begin{axdef}
viewed : \seq Programme \fun \seq Programme
\where
viewed(\langle \rangle) = \langle \rangle \land (\forall x : Programme @ \forall s : \seq Programme @ viewed(\langle x \rangle \cat s) = \IF x.3 = yes \THEN \langle x \rangle \cat viewed(s) \ELSE viewed(s))
\end{axdef}

\bigskip

or otherwise (omitted - requires semicolon-separated bindings in set comprehension)

\bigskip

\medskip

(d)
\par
\vspace{11pt}
\begin{axdef}
cumulative\_total : \seq Title * Length * Viewed \fun \nat
\where
cumulative\_total(\langle \rangle) = 0 \\
\forall x : Title * Length * Viewed @ \forall s : \seq Title * Length * Viewed @ cumulative\_total(\langle x \rangle \cat s) = x.2 + cumulative\_total(s)
\end{axdef}

\medskip

(e)
\par
\vspace{11pt}
\bigskip

(mu $p : ran hd | $$\forall q : \ran hd @ p \neq q \land p.2 > q.2$$\mid$ p.1)

\bigskip

\bigskip

(This, of course, assumes that there is a unique element with this property.)

\bigskip

\medskip

(f)
\par
\vspace{11pt}
\noindent
\hangindent=2em
(f) Omitted - requires semicolon-separated bindings in nested set comprehension

\bigskip


\medskip
\medskip

(g)
\par
\vspace{11pt}
\bigskip

axdef

\bigskip

\bigskip

$g : seq(Title * Length * Viewed) \fun seq(Title * Length * Viewed)$

\bigskip

\bigskip

where

\bigskip

\bigskip

$\forall s : \seq Title * Length * Viewed @ g(s) = s$$\rres$ $\{ x : \ran s | x \neq longest\_viewed(s) \}$

\bigskip

\bigskip

end

\bigskip

\bigskip



\bigskip

\bigskip

Where longest_viewed is defined as

\bigskip

\bigskip



\bigskip

\bigskip

axdef

\bigskip

\bigskip

$longest\_viewed : seq(Title * Length * Viewed) +\fun Title * Length * Viewed$

\bigskip

\bigskip

where

\bigskip

\bigskip

$\forall s : \seq Title * Length * Viewed @ longest\_viewed(s)$= $(\mu p : \ran s | p.3 = yes \land (\forall q : \ran s @ p \neq q \land q.3 = yes \land p.2 > q.2))$

\bigskip

\bigskip

end

\bigskip

\bigskip

This, of course, assumes that there is at least one viewed programme (and one of a unique maximum length).

\bigskip

\medskip

(h)
\par
\vspace{11pt}
\begin{axdef}
s : \seq Title * Length * Viewed \fun \seq Title * Length * Viewed
\where
\forall x : \seq Title * Length * Viewed @ items(s(x)) = items(x) \land (\forall i, j : \dom (s(x)) @ i < j \implies s(x)(i).2 \geq s(x)(j).2)
\end{axdef}

\medskip

\bigskip
\noindent
\textbf{Solution 41}

\medskip

(a)
\par
\vspace{11pt}
\bigskip

axdef

\bigskip

\bigskip

$records : Year $\pinj$ Table$

\bigskip

\bigskip

where

\bigskip

\bigskip

dom(records) = 1993..current

\bigskip

\bigskip

$\forall y : \dom records @ \# (records(y)) \leq 50$

\bigskip

\bigskip

$\forall$ $y : dom(records) | $$\forall e : \ran (records(y)) @ year(e.1) = y$

\bigskip

\bigskip

$\forall$ $r : ran(records) | $$\forall i1, i2 : \dom r @ i1 \neq i2 \land r(i1).1 = r(i2).1 \implies r(i1).3 \neq r(i2).3$

\bigskip

\bigskip

end

\bigskip

\medskip

(b)
\par
\vspace{11pt}
\medskip

\noindent
\hangindent=2em
(i) $\{ e : Entry | \exists r : \ran records @ e \in \ran r \land e.3 = 479 \}$

\noindent
$ii$


\noindent
$\{ e : Entry | \exists r : \ran records @ e \in \ran r \land e.6 > e.5 \}$


\noindent
$iii$


\noindent
$\{ e : Entry | \exists r : \ran records @ e \in \ran r \land e.7 \geq 70 \}$


\noindent
$iv$


\noindent
$\{ c : Course | \forall r : \ran records @ \forall e : \ran r @ e.2 = c \implies e.7 \geq 70 \}$


\noindent
$v$


\bigskip

$\{ y : Year | y \in \dom records @ y \mapsto \{ l : Lecturer | \# \{ c : \ran (records(y)) | c.4 = l \} > 6 \} \}$

\bigskip

\medskip

(c)
\par
\vspace{11pt}
\bigskip

axdef

\bigskip

\bigskip

where

\bigskip

\bigskip

$\forall x : Entry @ \forall s : \seq Entry @ 479\_courses$($\langle \rangle$) = $\langle \rangle$ and 479_courses ($\langle x \rangle$ ^ s) = if x.3 = 479 then $\langle x \rangle$ ^ 479_courses(s) else 479_courses(s)

\bigskip

\bigskip

end

\bigskip

\bigskip

(Blocked by: underscore in identifier - use camelCase for fuzz compatibility)

\bigskip

\medskip

(d)
\par
\vspace{11pt}
\begin{axdef}
\where
\forall x : Entry @ \forall s : \seq Entry @ total(\langle \rangle) = 0 \land total(\langle x \rangle \cat s) = x.5 + total(s)
\end{axdef}

\medskip

\bigskip
\noindent
\textbf{Solution 42}

\medskip

\begin{zed}[Person]\end{zed}

\bigskip

axdef

\bigskip

\bigskip

$State : P(seq(iseq(Person)))$

\bigskip

\bigskip

where

\bigskip

\bigskip

$\forall$ $s : State | $$\forall i, j : \dom s @ i \neq j \land \ran (s(i)) \cap \ran (s(j))$= $\{\}$

\bigskip

\bigskip

end

\bigskip

(b)
\par
\vspace{11pt}
\bigskip

axdef

\bigskip

\bigskip

$add : N * Person * State $\pinj$ State$

\bigskip

\bigskip

where

\bigskip

\bigskip

$\forall n : \nat @ \forall p : Person @ \forall s : State @ n \in \dom s \land p \notin \bigcup \ran \ran s$$\mid$

\bigskip

\bigskip

add(n, p, s) = s ++ {n $\mapsto$ s(n) ^ $\langle p \rangle$}

\bigskip

\bigskip

end

\bigskip

\bigskip

(Blocked by: $\pinj$ operator not implemented)

\bigskip

\medskip

\bigskip
\noindent
\textbf{Solution 43}

\medskip

(a)
\par
\vspace{11pt}
\noindent
\hangindent=2em
(i) $\forall$ $i : dom bookings | $$\forall x, y : bookings(i) @ x \neq y \land x.2 \upto x.3 \cap y.2 \upto y.3$= $\{\}$

\bigskip


\medskip
\bigskip



\bigskip

\noindent
\hangindent=2em
(ii) $\forall$ $i : dom bookings | $\forall$ x $: bookings(i) $\mid$ $\{x.2, x.3\}$ subseteq 1..max(i.1)

\bigskip


\medskip
\bigskip



\bigskip

\noindent
\hangindent=2em
(iii) $\forall$ $i : dom bookings | $$\forall b : bookings(i) @ b.2 \leq b.3$

\bigskip


\medskip
\bigskip



\bigskip

\noindent
\hangindent=2em
(iv) This is enforced by the constraint for part (i).

\bigskip


\medskip
\bigskip



\bigskip

\medskip

(b)
\par
\vspace{11pt}
\noindent
\hangindent=2em
(i) $\{ i : \dom bookings | i.1 = Banbury @ i.2 \}$

\bigskip


\medskip
\bigskip



\bigskip

\noindent
\hangindent=2em
(ii) $\{ i : \dom bookings | i.1 = Banbury \land (\exists b : bookings(i) @ 50 \in b.2 \upto b.3) \}$

\bigskip


\medskip
\bigskip



\bigskip

\noindent
\hangindent=2em
(iii) {$r : Room$; $s : N | $$\exists i : \dom bookings @ i.1 = r \land i.2 = s$. (r, s)}

\bigskip


\medskip
\bigskip



\bigskip

\noindent
\hangindent=2em
(iv) $\{ r : Room | \exists i : \dom bookings @ i.1 = r \land \# (bookings(i)) \geq 10 \}$

\bigskip


\medskip
\bigskip



\bigskip

\medskip

\section*{Free types and induction}

\begin{zed}[N]\end{zed}

\begin{zed}Tree ::= stalk | leaf \ldata \nat \rdata | branch \ldata Tree \cross Tree \rdata\end{zed}

\bigskip
\noindent
\textbf{Solution 44}

\medskip

\bigskip

The two cases of the proof are established by equational reasoning: the first by

\bigskip

\bigskip



\bigskip

\bigskip

reverse ($\langle \rangle$ ^ t) = reverse t [cat.1a] = (reverse t) ^ $\langle \rangle$ [cat.1b]

\bigskip

\bigskip



\bigskip

\bigskip

where cat.1a is $\langle \rangle$ ^ $s = s$ and cat.1b is s ^ $\langle \rangle$ = s

\bigskip

\bigskip



\bigskip

\bigskip

and the second by

\bigskip

\bigskip



\bigskip

\bigskip

reverse (($\langle x \rangle$ ^ u) ^ t) = reverse ($\langle x \rangle$ ^ (u ^ t)) [cat.2]

\bigskip

\bigskip

= reverse (u ^ t) ^ $\langle x \rangle$    [reverse.2]

\bigskip

\bigskip

= (reverse t ^ reverse u) ^ $\langle x \rangle$  [anti-distributive]

\bigskip

\bigskip

= reverse t ^ (reverse u ^ $\langle x \rangle$)  [cat.2]

\bigskip

\bigskip

= reverse t ^ reverse ($\langle x \rangle$ ^ u)  [reverse.2]

\bigskip

\bigskip
\noindent
\textbf{Solution 45}

\medskip

\bigskip

The base case:

\bigskip

\bigskip



\bigskip

\bigskip

reverse (reverse $\langle \rangle$) = reverse $\langle \rangle$ [reverse.1] = $\langle \rangle$ [reverse.1]

\bigskip

\bigskip



\bigskip

\bigskip

The inductive step:

\bigskip

\bigskip



\bigskip

\bigskip

reverse (reverse ($\langle x \rangle$ ^ t))

\bigskip

\bigskip

= reverse ((reverse t) ^ $\langle x \rangle$)  [reverse.2]

\bigskip

\bigskip

= reverse ($\langle x \rangle$) ^ reverse (reverse t)  [anti-distributive]

\bigskip

\bigskip

= reverse ($\langle x \rangle$ ^ $\langle \rangle$) ^ reverse (reverse t)  [cat.1]

\bigskip

\bigskip

= ((reverse $\langle \rangle$) ^ $\langle x \rangle$) ^ reverse (reverse t)  [reverse.2]

\bigskip

\bigskip

= ($\langle \rangle$ ^ $\langle x \rangle$) ^ reverse (reverse t)  [reverse.1]

\bigskip

\bigskip

= $\langle x \rangle$ ^ reverse (reverse t)  [cat.1]

\bigskip

\bigskip

= $\langle x \rangle$ ^ t  [reverse (reverse t) = t]

\bigskip

\bigskip
\noindent
\textbf{Solution 46}

\medskip

(a)
\par
\vspace{11pt}
\bigskip

$count : Tree \fun N$

\bigskip

\bigskip

count $stalk = 0$

\bigskip

\bigskip

$\forall n : \nat @ count(leaf(n)) = 1$

\bigskip

\bigskip

$\forall t1, t2 : Tree @ count(branch(t1, t2)) = count(t1) + count(t2)$

\bigskip

\bigskip

(Blocked $by : recursive free types and pattern matching)$

\bigskip

\medskip

(b)
\par
\vspace{11pt}
\bigskip

$flatten : Tree \fun seq N$

\bigskip

\bigskip

flatten stalk = $\langle \rangle$

\bigskip

\bigskip

$\forall n : \nat @ flatten(leaf(n))$= $\langle n \rangle$

\bigskip

\bigskip

$\forall t1, t2 : Tree @ flatten(branch(t1, t2)) = flatten(t1^{flatten})(t2)$

\bigskip

\bigskip

(Blocked $by : recursive free types and pattern matching)$

\bigskip

\medskip

\bigskip
\noindent
\textbf{Solution 47}

\medskip

\bigskip

First, exhibit the induction principle for the free type:

\bigskip

\bigskip



\bigskip

\bigskip

P stalk and ($\forall n : \nat @ \power (leaf(n))$) and $\forall t1, t2 : Tree @ \power t1 \land \power t2 \implies \power (branch(t1, t2))$

\bigskip

\bigskip

implies $\forall t : Tree @ \power t$

\bigskip

\bigskip



\bigskip

\bigskip

This gives three cases for the proof:

\bigskip

\bigskip



\bigskip

\bigskip

# (flatten stalk) = # $\langle \rangle$ [flatten] = 0 [#] = count stalk [count]

\bigskip

\bigskip



\bigskip

\bigskip

(Remaining cases omitted - require equational reasoning with recursive functions)

\bigskip

\section*{Supplementary material : assignment practice}

\bigskip
\noindent
\textbf{Solution 48}

\medskip

\begin{zed}[SongId, UserId, PlaylistId, Playlist]\end{zed}

\begin{axdef}
songs : \finset SongId \\
users : \finset UserId \\
playlists : PlaylistId \pfun Playlist \\
playlistOwner : PlaylistId \pfun UserId \\
playlistSubscribers : PlaylistId \pfun \finset_1 UserId
\where
\forall i : \dom playlists @ \ran (playlists(i))(subseteq)(songs) \\
\dom playlistOwner(subseteq)(\dom playlists) \\
\ran playlistOwner(subseteq)(users) \\
\dom playlistSubscribers(subseteq)(\dom playlists) \\
\forall i : \dom playlistSubscribers @ playlistSubscribers(i)(subseteq)(users) \\
\forall i : \dom playlists @ playlistOwner(i) \in playlistSubscribers(i)
\end{axdef}

\bigskip
\noindent
\textbf{Solution 49}

\medskip

\begin{axdef}
hated : UserId \pfun \finset SongId \\
loved : UserId \pfun \finset SongId
\where
\dom hated(subseteq)(users) \\
\forall i : \dom hated @ hated(i)(subseteq)(songs) \\
\dom loved(subseteq)(users) \\
\forall i : \dom loved @ loved(i)(subseteq)(songs) \\
\forall i : \dom hated \cup \dom loved @ hated(i) \cap loved(i) = \emptyset
\end{axdef}

\bigskip
\noindent
\textbf{Solution 50}

\medskip

(a)
\par
\vspace{11pt}
\begin{zed}
A == users \setminus \bigcup \ran playlistSubscribers
\end{zed}

\medskip

(b)
\par
\vspace{11pt}
\begin{zed}
B == \{ p : \dom playlistSubscribers | \# (playlistSubscribers(p)) \geq 100 \}
\end{zed}

\medskip

(c)
\par
\vspace{11pt}
\begin{zed}
C == (\mu u : \dom loved | \forall v : \dom loved @ u \neq v \land \# (loved(u)) > \# (loved(v)))
\end{zed}

\medskip

(d)
\par
\vspace{11pt}
\begin{zed}
D == (\mu s : songs | \forall t : songs @ s \neq t \land \# \{ u : UserId | s \in loved(u) \} > \# \{ u : UserId | t \in loved(u) \})
\end{zed}

\medskip

\bigskip
\noindent
\textbf{Solution 51}

\medskip

(a)
\par
\vspace{11pt}
\bigskip

Let's first define two helper functions:

\bigskip

\bigskip



\bigskip

\bigskip

$loveHateScore : SongId +\fun N$

\bigskip

\bigskip

$\forall$ $i : songs | $# $\{ u : UserId | i \in loved(u) \}$ $\geq$ # $\{ u : UserId | i \in hated(u) \}$ $\Rightarrow$

\bigskip

\bigskip

loveHateScore(i) = # $\{ u : UserId | i \in loved(u) \}$ - # $\{ u : UserId | i \in hated(u) \}$

\bigskip

\bigskip

and

\bigskip

\bigskip

$\forall$ $i : songs | $# $\{ u : UserId | i \in loved(u) \}$ $<$ # $\{ u : UserId | i \in hated(u) \}$ $\Rightarrow$

\bigskip

\bigskip

loveHateScore(i) = 0

\bigskip

\bigskip



\bigskip

\begin{axdef}
playlistCount : SongId \pfun \nat
\where
\forall i : songs @ playlistCount(i) = \# \{ p : \dom playlist | i \in \ran (playlist(p)) \}
\end{axdef}

\bigskip

We then have:

\bigskip

\begin{axdef}
length : SongId \pfun \nat \\
popularity : SongId \pfun \nat
\where
\dom length(subseteq)(songs) \\
\dom popularity(subseteq)(songs) \\
\forall i : songs @ popularity(i) = loveHateScore(i) + playlistCount(i)
\end{axdef}

\medskip

(b)
\par
\vspace{11pt}
\bigskip

$mostPopular : SongId$

\bigskip

\bigskip

($\exists_1$ $i : songs | $$\forall j : songs @ i \neq j \land popularity(i) > popularity(j)$) $\Rightarrow$

\bigskip

\bigskip

mostPopular = (mu $i : songs | $$\forall j : songs @ i \neq j \land popularity(i) > popularity(j)$)

\bigskip

\bigskip

and

\bigskip

\noindent
$\lnot \exists_1 i : songs @ \forall j : songs @ i \neq j \land popularity(i) > popularity(j) \implies mostPopular = nullSong$


\medskip

\noindent
\hangindent=2em
(c) playlistsContainingMostPopularSong == $\{ i : \dom playlists | mostPopular \in \ran (playlists(i)) \}$

\bigskip


\medskip
\bigskip
\noindent
\textbf{Solution 52}

\medskip

(a)
\par
\vspace{11pt}
\bigskip

$premiumPlays : seq(Play) \fun seq(Play)$

\bigskip

\bigskip

premiumPlays($\langle \rangle$) = $\langle \rangle$

\bigskip

\bigskip

$\forall$ $x : Play$; $s : seq(Play) |$

\bigskip

\bigskip

premiumPlays($\langle x \rangle$ ^ s) = $\langle x \rangle$ ^ premiumPlays(s)   if userStatus(x.2) = premium

\bigskip

\bigskip

premiumPlays(s)          if userStatus(x.2) = standard

\bigskip

\bigskip

(Note: Uses camelCase for fuzz compatibility)

\bigskip

\medskip

(b)
\par
\vspace{11pt}
\bigskip

$standardPlays : seq(Play) \fun seq(Play)$

\bigskip

\bigskip

standardPlays($\langle \rangle$) = $\langle \rangle$

\bigskip

\bigskip

$\forall$ $x : Play$; $s : seq(Play) |$

\bigskip

\bigskip

standardPlays($\langle x \rangle$ ^ s) = $\langle x \rangle$ ^ standardPlays(s)  if userStatus(x.2) = standard

\bigskip

\bigskip

standardPlays(s)        if userStatus(x.2) = premium

\bigskip

\bigskip

(Note: Uses camelCase for fuzz compatibility)

\bigskip

\medskip

(c)
\par
\vspace{11pt}
\bigskip

$cumulativeLength : seq(Play) \fun N$

\bigskip

\bigskip

cumulativeLength($\langle \rangle$) = 0

\bigskip

\bigskip

$\forall$ $x : Play$; $s : seq(Play) |$

\bigskip

\bigskip

cumulativeLength($\langle x \rangle$ ^ s) = length(x.1) + cumulativeLength(s)

\bigskip

\bigskip

(Note: Uses camelCase for fuzz compatibility)

\bigskip

\medskip

\end{document}
\documentclass[a4paper,10pt,fleqn]{article}
\usepackage[margin=1in]{geometry}
\usepackage{amssymb}
\usepackage{adjustbox}
\usepackage{natbib}
\usepackage[colorlinks=true,linkcolor=blue,citecolor=blue,urlcolor=blue]{hyperref}
\usepackage{fuzz}
\usepackage{zed-maths}
\usepackage{zed-proof}
\newdimen\savedleftskip
\begin{document}

\section*{PAGEBREAK Examples}

\section*{Example 1 : Basic Page Break}
\addcontentsline{toc}{section}{Example 1 : Basic Page Break}

\noindent This content appears on the first page.

\bigskip

\noindent
$PAGEBREAK$


\noindent This content appears on the next page. The PAGEBREAK directive forces a page break at that point in the document.

\bigskip

\section*{Example 2 : Separating Solutions}
\addcontentsline{toc}{section}{Example 2 : Separating Solutions}

\noindent When preparing solutions for submission, you may want each solution on a separate page.

\bigskip

\section*{Solution 1}
\addcontentsline{toc}{section}{Solution 1}

\noindent First solution content here.

\bigskip

\noindent
$\forall x : \nat @ x \geq 0$


\noindent Explanation of solution 1.

\bigskip

\noindent
$PAGEBREAK$


\section*{Solution 2}
\addcontentsline{toc}{section}{Solution 2}

\noindent Second solution content here.

\bigskip

\noindent
$\exists y : \nat @ y > 10$


\noindent Explanation of solution 2.

\bigskip

\noindent
$PAGEBREAK$


\section*{Solution 3}
\addcontentsline{toc}{section}{Solution 3}

\noindent Third solution content here.

\bigskip

\section*{Example 3 : Section Separation}
\addcontentsline{toc}{section}{Example 3 : Section Separation}

\section*{Section 1 : Propositional Logic}

\noindent Content for section 1 covering propositional logic operators and truth tables.

\bigskip

\noindent
$p(and)(q) \implies p$


\noindent More content for section 1.

\bigskip

\noindent
$PAGEBREAK$


\section*{Section 2 : Predicate Logic}

\noindent Content for section 2 covering quantifiers and predicates.

\bigskip

\noindent
$\forall x : \nat @ \exists y : \nat @ x + y = 10$


\noindent More content for section 2.

\bigskip

\section*{Example 4 : Long Proofs}
\addcontentsline{toc}{section}{Example 4 : Long Proofs}

\noindent For lengthy proofs, you might want to start each major proof on a new page:

\bigskip

\section*{Theorem 1}
\addcontentsline{toc}{section}{Theorem 1}

\begin{center}
\adjustbox{max width=\textwidth}{%
$\displaystyle
\infer[\Rightarrow\textrm{-intro}^{[1]}]{p(and(p \implies q)) \implies q}{
  \infer[\Rightarrow \mbox{elim}]{q}{\infer[\land\textrm{-elim-1}]{p}{\ulcorner p(and(p \implies q)) \urcorner^{[1]}} & \infer[\land\textrm{-elim-2}]{p \implies q}{\ulcorner p(and(p \implies q)) \urcorner^{[1]}}}
}
$%
}
\end{center}
\bigskip

\noindent End of proof for Theorem 1.

\bigskip

\noindent
$PAGEBREAK$


\section*{Theorem 2}
\addcontentsline{toc}{section}{Theorem 2}

\begin{center}
\adjustbox{max width=\textwidth}{%
$\displaystyle
\infer[\Rightarrow\textrm{-intro}^{[1]}]{p(or)(q)(and(not(q))) \implies p}{
  \infer[\lor \mbox{elim}]{p}{\infer[\land\textrm{-elim-2}]{not(q)}{\infer[\land\textrm{-elim-1}]{p(or)(q)}{\ulcorner p(or)(q)(and(not(q))) \urcorner^{[1]}}}}
}
$%
}
\end{center}
\bigskip

\noindent End of proof for Theorem 2.

\bigskip

\section*{Example 5 : Strategic Page Breaks}
\addcontentsline{toc}{section}{Example 5 : Strategic Page Breaks}

\noindent Use PAGEBREAK strategically to improve document readability. Don't overuse it—let LaTeX handle most page breaking automatically.

\bigskip

\noindent Good uses of PAGEBREAK include:

\bigskip

\noindent - Starting new major sections

\bigskip

\noindent - Separating independent solutions

\bigskip

\noindent - Ensuring related content stays together

\bigskip

\section*{Example 6 : Avoiding Orphans}
\addcontentsline{toc}{section}{Example 6 : Avoiding Orphans}

\noindent When you have a section title or definition that would appear at the bottom of a page with its content on the next page, you might insert a PAGEBREAK before the section.

\bigskip

\noindent
$PAGEBREAK$


\section*{Important Definition}

\begin{axdef}
critical : \nat \fun \nat
\where
\forall n : \nat @ critical(n) = n * n
\end{axdef}

\noindent This ensures the definition stays with its title.

\bigskip

\section*{Example 7 : Submission Formatting}
\addcontentsline{toc}{section}{Example 7 : Submission Formatting}

\noindent For coursework submissions, instructors may require one solution per page:

\bigskip

\section*{Problem 1}
\addcontentsline{toc}{section}{Problem 1}

\noindent Solution to problem 1 goes here.

\bigskip

\noindent
$PAGEBREAK$


\section*{Problem 2}
\addcontentsline{toc}{section}{Problem 2}

\noindent Solution to problem 2 goes here.

\bigskip

\noindent
$PAGEBREAK$


\section*{Problem 3}
\addcontentsline{toc}{section}{Problem 3}

\noindent Solution to problem 3 goes here.

\bigskip

\noindent This format makes it easy for instructors to review and grade individual problems.

\bigskip

\end{document}
\documentclass[a4paper,10pt,fleqn]{article}
\usepackage[margin=1in]{geometry}
\usepackage{amssymb}
\usepackage{adjustbox}
\usepackage{natbib}
\usepackage[colorlinks=true,linkcolor=blue,citecolor=blue,urlcolor=blue]{hyperref}
\usepackage{fuzz}
\usepackage{zed-maths}
\usepackage{zed-proof}
\newdimen\savedleftskip
\title{Bibliography Example}
\author{Example Author}
\date{2025}
\begin{document}

\maketitle

\section*{Introduction}

\noindent This example demonstrates bibliography file support $\in$ txt2tex.

\bigskip

\noindent The formal methods community has produced many influential
works on Z notation. The foundational work by \citep{spivey92}
introduced the Z notation, while \citep{woodcock96} provides a
comprehensive guide to using Z in practice.

\bigskip

\noindent This example uses a separate .bib file for bibliography
management. The build process automatically runs BibTeX to process citations.

\bigskip

\section*{Solution 1}
\addcontentsline{toc}{section}{Solution 1}

\noindent When citing multiple sources, the bibliography is
automatically generated from the .bib file. For example,
\citep{spivey92} and \citep{woodcock96} are both included in the references.

\bigskip

\setlength{\leftskip}{0pt}

\bibliographystyle{plainnat}

\bibliography{references}

\end{document}

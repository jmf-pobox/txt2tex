\documentclass[a4paper,10pt,fleqn]{article}
\usepackage[margin=1in]{geometry}
\usepackage{amssymb}
\usepackage{adjustbox}
\usepackage{natbib}
\usepackage[colorlinks=true,linkcolor=blue,citecolor=blue,urlcolor=blue]{hyperref}
\usepackage{fuzz}
\usepackage{zed-maths}
\usepackage{zed-proof}
\newdimen\savedleftskip
\begin{document}

\section*{LATEX Passthrough Examples}

\section*{Example 1 : Basic LaTeX Commands}
\addcontentsline{toc}{section}{Example 1 : Basic LaTeX Commands}

\noindent You can insert custom LaTeX commands using the LATEX directive:

\bigskip

\noindent This paragraph has no indentation.

\noindent The LATEX block passes its content directly to the LaTeX output.

\bigskip

\section*{Example 2 : Custom Spacing}
\addcontentsline{toc}{section}{Example 2 : Custom Spacing}

\noindent Control vertical spacing with LaTeX commands:

\bigskip

\vspace{1cm}

\noindent The above adds 1cm of vertical space.

\bigskip

\vspace{0.5cm}

\noindent You can add as much or as little space as needed.

\bigskip

\section*{Example 3 : Custom Formatting}
\addcontentsline{toc}{section}{Example 3 : Custom Formatting}

\noindent Apply special formatting to text:

\bigskip

\begin{center}

\textbf{\Large This text is centered, bold, and large}

\end{center}

\noindent The LATEX blocks allow full control over typography.

\bigskip

\section*{Example 4 : Boxes and Frames}
\addcontentsline{toc}{section}{Example 4 : Boxes and Frames}

\noindent Create boxed text with LaTeX environments:

\bigskip

\begin{center}

\fbox{\parbox{0.8\textwidth}{

This text appears in a framed box. You can use this to highlight important definitions or theorems.

}}

\end{center}

\section*{Example 5 : Custom Lists}
\addcontentsline{toc}{section}{Example 5 : Custom Lists}

\noindent Create specialized list formats:

\bigskip

\begin{description}

\item[Precondition:] The input must be a natural number

\item[Postcondition:] The output is the square of the input

\item[Invariant:] The result is always non-negative

\end{description}

\section*{Example 6 : Theorems and Definitions}
\addcontentsline{toc}{section}{Example 6 : Theorems and Definitions}

\noindent If you've loaded theorem packages, you can use them:

\bigskip

\begin{quotation}

\textbf{Theorem 1.} For all natural numbers $n$, we have $n \geq 0$.

\end{quotation}

\noindent The LATEX directive lets you use any LaTeX environment or command.

\bigskip

\section*{Example 7 : Page Layout Control}
\addcontentsline{toc}{section}{Example 7 : Page Layout Control}

\noindent Control page layout with LaTeX commands:

\bigskip

\clearpage

\noindent The clearpage command forces a page break and flushes all pending floats.

\bigskip

\section*{Example 8 : Mixed Z Notation and LaTeX}
\addcontentsline{toc}{section}{Example 8 : Mixed Z Notation and LaTeX}

\noindent You can combine Z notation with custom LaTeX formatting:

\bigskip

\begin{axdef}
example : \nat
\where
example = 42
\end{axdef}

\medskip

\noindent\textit{Note:} The above defines a constant with value 42.

\noindent This demonstrates the flexibility of mixing txt2tex notation with raw LaTeX.

\bigskip

\section*{Example 9 : Custom Macros}
\addcontentsline{toc}{section}{Example 9 : Custom Macros}

\noindent If you define custom LaTeX macros in your preamble, you can use them:

\bigskip

% Assuming you've defined \newcommand{\spec}[1]{\textsc{#1}}

% You could write:

% \spec{Specification} is shown in small caps.

\noindent The LATEX directive is your escape hatch for any LaTeX feature not directly supported by txt2tex.

\bigskip

\end{document}
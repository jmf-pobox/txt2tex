\documentclass[a4paper,10pt,fleqn]{article}
\usepackage[margin=1in]{geometry}
\usepackage{amssymb}
\usepackage{adjustbox}
\usepackage{natbib}
\usepackage[colorlinks=true,linkcolor=blue,citecolor=blue,urlcolor=blue]{hyperref}
\usepackage{fuzz}
\usepackage{zed-maths}
\usepackage{zed-proof}
\newdimen\savedleftskip
\begin{document}

\section*{Combined Text Directives}

\section*{Example 1 : Documentation Style}
\addcontentsline{toc}{section}{Example 1 : Documentation Style}

\noindent This example demonstrates how to combine different text directives effectively. Use TEXT for normal explanatory prose with proper typography.

\bigskip

\bigskip

Use PURETEXT when showing literal syntax or code examples:

\bigskip

\bigskip

txt2tex examples/hello\_world.txt

\bigskip

\noindent And use LATEX when you need specific LaTeX formatting not available otherwise.

\bigskip

\section*{Example 2 : Tutorial Format}
\addcontentsline{toc}{section}{Example 2 : Tutorial Format}

\noindent Let's explore the natural numbers. In Z notation, we write N for the set of natural numbers: {0, 1, 2, 3, ...}.

\bigskip

\begin{axdef}
first\_five : \power \nat
\where
first\_five = \{0, 1, 2, 3, 4\}
\end{axdef}

\medskip

\noindent The above defines a set containing the first five natural numbers. Note how TEXT blocks provide context before and after formal definitions.

\bigskip

\section*{Example 3 : Showing Syntax vs Using Syntax}
\addcontentsline{toc}{section}{Example 3 : Showing Syntax vs Using Syntax}

\noindent To demonstrate the difference between showing syntax and using it, consider quantifiers.

\bigskip

\noindent When we want to use a quantifier in our specification:

\bigskip

\noindent
$\forall x : \nat @ x \geq 0$


\noindent But when we want to show the syntax literally for teaching purposes:

\bigskip

\bigskip

forall x : T | predicate

\bigskip

\noindent The PURETEXT block shows the syntax pattern without actually creating a formal quantified expression.

\bigskip

\section*{Example 4 : Code Examples with Explanations}
\addcontentsline{toc}{section}{Example 4 : Code Examples with Explanations}

\noindent The txt2tex tool supports several commands. Here are the most common:

\bigskip

\bigskip

\# Convert to LaTeX only

\bigskip

\bigskip

txt2tex examples/file.txt --tex-only

\bigskip

\bigskip



\bigskip

\bigskip

\# Convert to PDF (full pipeline)

\bigskip

\bigskip

txt2tex examples/file.txt

\bigskip

\noindent The first command generates only the LaTeX file, while the second runs the complete pipeline including PDF generation.

\bigskip

\medskip\noindent\textit{Tip:} Always use the convert command for final output.

\section*{Example 5 : Mathematical Explanation}
\addcontentsline{toc}{section}{Example 5 : Mathematical Explanation}

\noindent Consider the function f(n) = n\textasciicircum$\{\}$2. We can define this formally in Z notation:

\bigskip

\begin{axdef}
square : \nat \fun \nat
\where
\forall n : \nat @ square(n) = n * n
\end{axdef}

\noindent This states that square is a total function from natural numbers to natural numbers.

\bigskip

\begin{center}

\begin{tabular}{c|c}

$n$ & $f(n)$ \\ \hline

0 & 0 \\

1 & 1 \\

2 & 4 \\

3 & 9 \\

4 & 16

\end{tabular}

\end{center}

\noindent The table above shows the first few values of the square function.

\bigskip

\section*{Example 6 : Syntax Reference}
\addcontentsline{toc}{section}{Example 6 : Syntax Reference}

\noindent Here's a quick reference for text directives:

\bigskip

\begin{description}

\item[TEXT:] Normal prose with smart quotes and inline math

\item[PURETEXT:] Verbatim text without processing

\item[LATEX:] Raw LaTeX commands

\item[PAGEBREAK:] Force a page break

\end{description}

\bigskip

Example usage:

\bigskip

\bigskip

TEXT: This is a paragraph.

\bigskip

\bigskip

PURETEXT: This is verbatim.

\bigskip

\bigskip

LATEX: \textbackslash\{\}textbf\{Bold text\}

\bigskip

\bigskip

PAGEBREAK

\bigskip

\section*{Example 7 : Best Practices}
\addcontentsline{toc}{section}{Example 7 : Best Practices}

\noindent When writing specifications, follow these guidelines:

\bigskip

\noindent 1. Use TEXT for all explanatory prose—it provides the best typography

\bigskip

\noindent 2. Use PURETEXT sparingly, only when you need to show literal syntax or preserve special characters

\bigskip

\noindent 3. Use LATEX as an escape hatch for formatting not supported by txt2tex

\bigskip

\noindent 4. Use PAGEBREAK only when you need explicit control over page breaks

\bigskip

\vspace{1cm}

\noindent By following these practices, your documents will be both readable and maintainable.

\bigskip

\section*{Example 8 : Real - World Document Structure}
\addcontentsline{toc}{section}{Example 8 : Real - World Document Structure}

\noindent A typical specification document might look like this:

\bigskip

\section*{Section 1 : Introduction}

\noindent Prose introduction explaining the problem domain.

\bigskip

\section*{Section 2 : Data Types}

\begin{zed}
[User, Document, Permission]
\end{zed}

\noindent The system has three basic $types : users$, documents, and permissions.

\bigskip

\section*{Section 3 : State Schema}

\begin{schema}{FileSystem}
docs : User \pfun \power Document \\
perms : Document \pfun \power Permission \\
all\_docs : \power Document
\where
all\_docs = \bigcup (\ran docs) \\
all\_docs \subseteq \dom perms
\end{schema}

\noindent The state schema defines the relationships between users, documents, and permissions.

\bigskip

\noindent
$PAGEBREAK$


\section*{Section 4 : Operations}

\noindent Now we define the operations on the file system...

\bigskip

\noindent This structure—sections with TEXT introductions, formal definitions, and strategic page breaks—creates clear, professional specifications.

\bigskip

\end{document}
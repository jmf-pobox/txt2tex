\documentclass[a4paper,10pt,fleqn]{article}
\usepackage[margin=1in]{geometry}
\usepackage{amssymb}
\usepackage{adjustbox}
\usepackage{natbib}
\usepackage{fuzz}
\usepackage{zed-maths}
\usepackage{zed-proof}
\usepackage[colorlinks=true,linkcolor=blue,citecolor=blue,urlcolor=blue]{hyperref}
\newdimen\savedleftskip
\begin{document}

\section*{TEXT Block Examples}

\section*{Example 1 : Basic TEXT Blocks}
\addcontentsline{toc}{section}{Example 1 : Basic TEXT Blocks}

\noindent This is a simple prose paragraph. The TEXT directive allows
you to write explanatory text that will be typeset as regular
paragraphs in the output document.

\bigskip

\noindent You can have multiple TEXT blocks in sequence. Each TEXT
block becomes a separate paragraph with proper spacing.

\bigskip

\section*{Example 2 : Smart Quotes}
\addcontentsline{toc}{section}{Example 2 : Smart Quotes}

\noindent The TEXT directive automatically converts straight quotes
to "smart quotes" for proper typography. This includes both "double
quotes" and 'single quotes'.

\bigskip

\noindent For example: "The quick brown fox" becomes properly
formatted with opening and closing quotes.

\bigskip

\section*{Example 3 : Inline Mathematical Notation}
\addcontentsline{toc}{section}{Example 3 : Inline Mathematical Notation}

\noindent You can include inline mathematical $expressions \in TEXT$
blocks. For example, the function f(n) = $n^2$ computes the square of n.

\bigskip

\noindent More complex expressions work too: The formula for the sum
of squares is sum($i = 1$ to n) of $i^2$ = n(n+1)(2n+1)/6.

\bigskip

\section*{Example 4 : Mixed Content}
\addcontentsline{toc}{section}{Example 4 : Mixed Content}

\noindent Z notation provides powerful tools for formal
specification. The $\forall$ quantifier allows us to express
universal properties.

\bigskip

\noindent
$\forall x : \nat @ x \geq 0$

\noindent The above predicate states that all natural numbers are
non-negative, which is true by definition of N.

\bigskip

\section*{Example 5 : Explaining Z Notation}
\addcontentsline{toc}{section}{Example 5 : Explaining Z Notation}

\noindent When defining a function like $square : N$ $\fun$ N, we're
declaring a total function from natural numbers to natural numbers.

\bigskip

\begin{axdef}
  square : \nat \fun \nat
  \where
  \forall n : \nat @ square(n) = n * n
\end{axdef}

\noindent The axdef block gives the signature and the constraint that
defines the function's behavior.

\bigskip

\section*{Example 6 : Long Explanations}
\addcontentsline{toc}{section}{Example 6 : Long Explanations}

\noindent The TEXT directive is particularly useful for providing
detailed explanations of formal specifications. It allows you to
write at length about the design decisions, invariants, and
assumptions that underlie your formal models.

\bigskip

\noindent By interspersing TEXT blocks with formal notation, you
create documents that are both mathematically precise and
human-readable. This is essential for effective formal methods practice.

\bigskip

\section*{Example 7 : Technical Writing}
\addcontentsline{toc}{section}{Example 7 : Technical Writing}

\noindent In practice, specifications should include both formal
definitions and prose explanations. The formal notation provides
precision and enables automated checking with tools like fuzz. The
prose provides context and motivation that help readers understand
the purpose and design of the specification.

\bigskip

\end{document}

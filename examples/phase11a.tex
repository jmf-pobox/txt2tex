\documentclass[fleqn]{article}
\usepackage[left=1.55in,right=1.55in,top=1in,bottom=1in]{geometry}
\usepackage{fuzz}
\usepackage{amsmath}
\usepackage{amssymb}
\begin{document}

\section*{Phase 11 a : Function Types}

\bigskip

Total functions (every element in domain has a mapping):

\bigskip

\begin{axdef}
f : X \fun Y
g : \mathbb{N} \fun \mathbb{N}
\end{axdef}

\bigskip

Partial functions (some elements may not have a mapping):

\bigskip

\begin{axdef}
f : X \pfun Y
\end{axdef}

\bigskip

Injections (one-to-one mappings):

\bigskip

\begin{axdef}
ftotal : X \inj Y
fpartial : X \pinj Y
\end{axdef}

\bigskip

Surjections (onto mappings):

\bigskip

\begin{axdef}
ftotal : X \surj Y
fpartial : X \psurj Y
\end{axdef}

\bigskip

Bijections (one-to-one and onto):

\bigskip

\begin{axdef}
f : X \bij Y
\end{axdef}

\bigskip

Complex function types (nested and higher-order):

\bigskip

\begin{axdef}
f : X \fun Y \fun \mathbb{Z}
g : X \fun (Y \pfun \mathbb{Z})
h : \mathbb{N} \fun \mathbb{N} \fun (\mathbb{N} \fun \mathbb{N})
\end{axdef}

\bigskip

Mixed operators (relations and function types work together):

\bigskip

\begin{axdef}
relation : X \rel Y
function : X \fun Y
\where
function \subseteq relation
\end{axdef}

\end{document}
\documentclass[a4paper,10pt,fleqn]{article}
\usepackage[margin=1in]{geometry}
\usepackage{amssymb}
\usepackage{adjustbox}
\usepackage{natbib}
\usepackage{fuzz}
\usepackage{zed-maths}
\usepackage{zed-proof}
\usepackage[colorlinks=true,linkcolor=blue,citecolor=blue,urlcolor=blue]{hyperref}
\newdimen\savedleftskip
\begin{document}

\section*{Law of Excluded Middle Examples}

\section*{Example 1 : Statement of LEM}
\addcontentsline{toc}{section}{Example 1 : Statement of LEM}

\noindent The law of excluded middle states: for any proposition p,
either p is true or $\lnot p$ is true.

\bigskip

\noindent
$p \lor \lnot p$

\noindent This is an axiom of classical logic but not of constructive logic.

\bigskip

\section*{Example 2 : Proof by Cases Using LEM}
\addcontentsline{toc}{section}{Example 2 : Proof by Cases Using LEM}

\noindent Prove: $(p \implies q) \implies ((\lnot p \implies q) \implies q)$

\bigskip

\begin{center}
  \adjustbox{max width=\textwidth}{%
    $\displaystyle
    \infer[\lor \mathrm{elim}]{q}{
      \infer[\mathrm{assumption}]{\lnot p \implies q}{\infer[LEM]{p
        \lor \lnot p}{\ulcorner p \implies q \urcorner^{[1]}} &
        \infer[\Rightarrow \mathrm{elim} with p\Rightarrow q]{q}{} &
        \infer[\Rightarrow \mathrm{elim} with \lnot p\Rightarrow
        q]{q}{} & \infer[\lor \mathrm{elim}]{q}{\ulcorner p \implies q
      \urcorner^{[1]}}}
    }
    $%
  }
\end{center}
\bigskip

\noindent By LEM, we can case-analyze on p or $\lnot p$. In both
cases we derive q.

\bigskip

\section*{Example 3 : Double Negation Elimination Using LEM}
\addcontentsline{toc}{section}{Example 3 : Double Negation
Elimination Using LEM}

\noindent Prove: not $\lnot p \implies p$

\bigskip

\begin{center}
  \adjustbox{max width=\textwidth}{%
    $\displaystyle
    \infer[\lor \mathrm{elim}]{p}{
      \infer[\lor \mathrm{elim}]{p}{\infer[\mathrm{false}
        \mathrm{elim}]{p}{\infer[\mathrm{contradiction} with \lnot
      \lnot p]{false}{}}}
    }
    $%
  }
\end{center}
\bigskip

\noindent LEM enables double negation elimination in classical logic.

\bigskip

\section*{Example 4 : De Morgan's Law Using LEM}
\addcontentsline{toc}{section}{Example 4 : De Morgan's Law Using LEM}

\noindent Prove: not $(p \land q)$ $\Rightarrow$ $(\lnot p \lor \lnot q)$

\bigskip

\begin{center}
  \adjustbox{max width=\textwidth}{%
    $\displaystyle
    \infer[\lor \mathrm{elim}]{\lnot p \lor \lnot q}{
      \infer[\lor \mathrm{elim}]{\lnot p \lor \lnot q}{\infer[\lor
      \mathrm{intro} left]{\lnot p \lor \lnot q}{\infer[identity]{\lnot p}{}}}
    }
    $%
  }
\end{center}
\bigskip

\noindent This proof requires LEM to case-analyze on both p and q.

\bigskip

\section*{Example 5 : Decidability}
\addcontentsline{toc}{section}{Example 5 : Decidability}

\noindent A proposition is decidable if we can prove p or $\lnot p$.
In logical notation, we would $write : decidable(p) means$ $(p \lor \lnot p)$.

\bigskip

\noindent LEM states that all propositions are decidable in classical logic.

\bigskip

\section*{Example 6 : Proof by Contradiction via LEM}
\addcontentsline{toc}{section}{Example 6 : Proof by Contradiction via LEM}

\noindent Prove: $(\lnot p \implies false) \implies p$

\bigskip

\begin{center}
  \adjustbox{max width=\textwidth}{%
    $\displaystyle
    \infer[\lor \mathrm{elim}]{p}{
      \infer[\lor \mathrm{elim}]{p}{\infer[\mathrm{false}
      \mathrm{elim}]{p}{\infer[\Rightarrow \mathrm{elim}]{false}{}}}
    }
    $%
  }
\end{center}
\bigskip

\noindent This shows that proof by contradiction follows from LEM.

\bigskip

\section*{Example 7 : Material Implication}
\addcontentsline{toc}{section}{Example 7 : Material Implication}

\noindent Prove: $(p \implies q) \iff (\lnot p \lor q)$

\bigskip

\noindent Left-to-right $(p \implies (q \implies \lnot p \lor q))$:

\bigskip

\begin{center}
  \adjustbox{max width=\textwidth}{%
    $\displaystyle
    \infer[\lor \mathrm{elim}]{\lnot p \lor q}{
      \infer[\lor \mathrm{elim}]{\lnot p \lor q}{\infer[\lor
      \mathrm{intro} left]{\lnot p \lor q}{}}
    }
    $%
  }
\end{center}
\bigskip

\noindent Right-to-left $(\lnot p \lor q \implies (p \implies q))$:

\bigskip

\begin{center}
  \adjustbox{max width=\textwidth}{%
    $\displaystyle
    \infer[\Rightarrow\textrm{-intro}^{[1]}]{p \implies q}{
      \infer[\mathrm{assumption}]{p}{\infer[\mathrm{false}
        \mathrm{elim}]{q}{\infer[\mathrm{contradiction}]{false}{}} &
        \infer[identity]{q}{} & \infer[\lor \mathrm{elim}]{q}{\ulcorner
      \lnot p \lor q \urcorner^{[1]}}}
    }
    $%
  }
\end{center}
\bigskip

\noindent Material implication equivalence uses LEM in the forward direction.

\bigskip

\section*{Example 8 : Pierce's Law}
\addcontentsline{toc}{section}{Example 8 : Pierce's Law}

\noindent Prove: $((p \implies q) \implies p)$ $\Rightarrow$ p

\bigskip

\begin{center}
  \adjustbox{max width=\textwidth}{%
    $\displaystyle
    \infer[\lor \mathrm{elim}]{p}{
      \infer[\lor \mathrm{elim}]{p}{\infer[\Rightarrow \mathrm{elim}
          with
        \mathrm{premise}]{p}{\infer[\Rightarrow\textrm{-intro}^{[2]}]{p
          \implies q}{\infer[\mathrm{false}
            \mathrm{elim}]{q}{\infer[\mathrm{contradiction}]{false}{\ulcorner
      p \urcorner^{[2]}}}}}}
    }
    $%
  }
\end{center}
\bigskip

\noindent Pierce's law is equivalent to LEM in classical logic.

\bigskip

\section*{Example 9 : Tertium Non Datur}
\addcontentsline{toc}{section}{Example 9 : Tertium Non Datur}

\noindent Tertium non datur (Latin: "no third possibility") is
another name for LEM.

\bigskip

\noindent For any proposition p, there is no third option besides p
and $\lnot p$.

\bigskip

\noindent
$\forall p @ p \lor \lnot p$

\noindent This principle distinguishes classical from intuitionistic logic.

\bigskip

\section*{Example 10 : Constructive vs Classical}
\addcontentsline{toc}{section}{Example 10 : Constructive vs Classical}

\noindent In constructive (intuitionistic) logic, LEM is not provable
as a general principle.

\bigskip

\noindent Constructive: To prove p or $\lnot p$, you must either:

\bigskip

\noindent - Construct a proof of p, or

\bigskip

\noindent - Construct a proof of $\lnot p$ (a proof that p leads to
contradiction)

\bigskip

\noindent $Classical : LEM$ is an axiom—every proposition is assumed
decidable without construction.

\bigskip

\section*{Example 11 : Example Where LEM Is Not Needed}
\addcontentsline{toc}{section}{Example 11 : Example Where LEM Is Not Needed}

\noindent Some propositions are constructively decidable:

\bigskip

\noindent
$\forall n : \nat @ (n = 0) \lor \lnot (n = 0)$

\noindent We can decide this by checking if n equals 0.

\bigskip

\noindent But general propositions may not be decidable constructively:

\bigskip

\noindent
$\forall p @ p \lor \lnot (p - -(requires(LEM))) \in general(case)$

\section*{Example 12 : Proof Strategy with LEM}
\addcontentsline{toc}{section}{Example 12 : Proof Strategy with LEM}

\noindent Common proof strategy using LEM:

\bigskip

\noindent To prove Q:

\bigskip

\noindent 1. Invoke p or $\lnot p$ for some relevant p

\bigskip

\noindent 2. [case] p prove Q assuming p

\bigskip

\noindent 3. [case] $\lnot p$ prove Q assuming $\lnot p$

\bigskip

\noindent 4. Conclude Q by or elim

\bigskip

\noindent This technique is called "proof by cases" or "case analysis."

\bigskip

\section*{Example 13 : LEM and Indirect Proof}
\addcontentsline{toc}{section}{Example 13 : LEM and Indirect Proof}

\noindent Indirect proof pattern using LEM:

\bigskip

\noindent To prove p:

\bigskip

\noindent 1. Assume $\lnot p$ [assumption 1]

\bigskip

\noindent 2. Derive false

\bigskip

\noindent 3. Conclude not $\lnot p$ [not intro from 1]

\bigskip

\noindent 4. Apply $LEM : p$ or $\lnot p$

\bigskip

\noindent 5. [case] p p

\bigskip

\noindent 6. [case] $\lnot p$ contradicts not $\lnot p$, derive p by false elim

\bigskip

\noindent 7. Conclude p [or elim]

\bigskip

\noindent LEM enables the final step (double negation elimination).

\bigskip

\section*{Example 14 : Drinker Paradox}
\addcontentsline{toc}{section}{Example 14 : Drinker Paradox}

\noindent Prove: There $\exists$ a person x such that if x drinks,
then everyone drinks.

\bigskip

\begin{center}
  \adjustbox{max width=\textwidth}{%
    $\displaystyle
    \infer[\lor \mathrm{elim}]{\exists x @ x(drinks) \implies
    (\forall y @ y(drinks))}{
      \infer[\lor \mathrm{elim}]{\exists x @ x(drinks) \implies
      (\forall y @ y(drinks))}{\infer[\exists \mathrm{intro} with
        x=a]{\exists x @ x(drinks) \implies (\forall y @
        y(drinks))}{\infer[\Rightarrow \mathrm{intro},trivially
          true]{a(drinks) \implies (\forall y @
          y(drinks))}{\infer[\mathrm{from} p]{\forall y @
            y(drinks)}{\infer[\mathrm{from}
              p]{a(drinks)}{\infer{Pick(any)(person)}{
                  a
        }}}}} & \infer[\exists \mathrm{intro} with x=b]{\exists x @
          x(drinks) \implies (\forall y @
        y(drinks))}{\infer[\Rightarrow\textrm{-intro}^{[2]}]{b(drinks)
            \implies (\forall y @
          y(drinks))}{\infer[\mathrm{assumption}]{b(drinks)}{\infer[\exists
              \mathrm{elim}]{\lnot (b(drinks))}{\infer[negation of
      \forall]{\exists b @ \lnot (b(drinks))}{}}}}}}
    }
    $%
  }
\end{center}
\bigskip

\noindent This classical proof uses LEM crucially.

\bigskip

\section*{Example 15 : When to Use LEM}
\addcontentsline{toc}{section}{Example 15 : When to Use LEM}

\noindent Use LEM when:

\bigskip

\noindent 1. You need to prove something by case analysis on an
arbitrary proposition

\bigskip

\noindent 2. You're working in classical logic $\lnot constructive$

\bigskip

\noindent 3. You need double negation elimination

\bigskip

\noindent 4. Your proof strategy requires considering both p and $\lnot p$

\bigskip

\noindent Avoid LEM when:

\bigskip

\noindent 1. Working in constructive/intuitionistic logic

\bigskip

\noindent 2. The proposition has a decidable, constructive proof

\bigskip

\noindent 3. You want algorithms that compute witnesses (constructive proofs)

\bigskip

\section*{Example 16 : Relationship to Other Principles}
\addcontentsline{toc}{section}{Example 16 : Relationship to Other Principles}

\noindent In classical logic, these are equivalent:

\bigskip

\noindent - Law of excluded $middle : p$ or $\lnot p$

\bigskip

\noindent - Double negation $elimination : not$ $\lnot p \implies p$

\bigskip

\noindent - Pierce's law: $((p \implies q) \implies p)$ $\Rightarrow$ p

\bigskip

\noindent - Reductio ad absurdum: $(\lnot p \implies false) \implies p$

\bigskip

\noindent Accepting any one of these principles gives you classical logic.

\bigskip

\noindent In intuitionistic logic, none of these hold in general—you
must construct proofs explicitly.

\bigskip

\end{document}

\documentclass[a4paper,10pt,fleqn]{article}
\usepackage[margin=1in]{geometry}
\usepackage{amssymb}
\usepackage{adjustbox}
\usepackage{natbib}
\usepackage[colorlinks=true,linkcolor=blue,citecolor=blue,urlcolor=blue]{hyperref}
\usepackage{fuzz}
\usepackage{zed-maths}
\usepackage{zed-proof}
\newdimen\savedleftskip
\begin{document}

\section*{Proof Tree Nesting Depth Test}

\section*{Test 1 : Simple ( depth 1 )}
\addcontentsline{toc}{section}{Test 1 : Simple ( depth 1 )}

\begin{center}
  \adjustbox{max width=\textwidth}{%
    $\displaystyle
    \infer[\Rightarrow\textrm{-intro}^{[1]}]{p \implies q}{
      \ulcorner p \urcorner^{[1]}
      &
      \infer[axiom]{q}{}
    }
    $%
  }
\end{center}
\bigskip

\section*{Test 2 : Moderate ( depth 2 )}
\addcontentsline{toc}{section}{Test 2 : Moderate ( depth 2 )}

\begin{center}
  \adjustbox{max width=\textwidth}{%
    $\displaystyle
    \infer[\Rightarrow\textrm{-intro}^{[1]}]{p \implies (q \implies r)}{
      \ulcorner p \urcorner^{[1]}
      &
      \infer[\Rightarrow\textrm{-intro}^{[2]}]{q \implies r}{
        \ulcorner q \urcorner^{[2]}
        &
        \infer[axiom]{r}{}
      }
    }
    $%
  }
\end{center}
\bigskip

\section*{Test 3 : Deep ( depth 3 )}
\addcontentsline{toc}{section}{Test 3 : Deep ( depth 3 )}

\begin{center}
  \adjustbox{max width=\textwidth}{%
    $\displaystyle
    \infer[\Rightarrow\textrm{-intro}^{[1]}]{p \implies (q \implies
    (r \implies s))}{
      \ulcorner p \urcorner^{[1]}
      &
      \infer[\Rightarrow\textrm{-intro}^{[2]}]{q \implies (r \implies s)}{
        \ulcorner q \urcorner^{[2]}
        &
        \infer[\Rightarrow\textrm{-intro}^{[3]}]{r \implies s}{
          \ulcorner r \urcorner^{[3]}
          &
          \infer[axiom]{s}{}
        }
      }
    }
    $%
  }
\end{center}
\bigskip

\section*{Test 4 : Very Deep ( depth 4 )}
\addcontentsline{toc}{section}{Test 4 : Very Deep ( depth 4 )}

\begin{center}
  \adjustbox{max width=\textwidth}{%
    $\displaystyle
    \infer[\Rightarrow\textrm{-intro}^{[1]}]{p \implies (q \implies
    (r \implies (s \implies t)))}{
      \ulcorner p \urcorner^{[1]}
      &
      \infer[\Rightarrow\textrm{-intro}^{[2]}]{q \implies (r \implies
      (s \implies t))}{
        \ulcorner q \urcorner^{[2]}
        &
        \infer[\Rightarrow\textrm{-intro}^{[3]}]{r \implies (s \implies t)}{
          \ulcorner r \urcorner^{[3]}
          &
          \infer[\Rightarrow\textrm{-intro}^{[4]}]{s \implies t}{
            \ulcorner s \urcorner^{[4]}
            &
            \infer[axiom]{t}{}
          }
        }
      }
    }
    $%
  }
\end{center}
\bigskip

\section*{Test 5 : Horizontal siblings ( depth 2 )}
\addcontentsline{toc}{section}{Test 5 : Horizontal siblings ( depth 2 )}

\begin{center}
  \adjustbox{max width=\textwidth}{%
    $\displaystyle
    \infer[\Rightarrow\textrm{-intro}^{[1]}]{(p \land q) \implies r}{
      \ulcorner p \land q \urcorner^{[1]}
      &
      \infer[\land \mathrm{elim}]{r}{
        \ulcorner p \urcorner^{[1]}
        &
        \ulcorner q \urcorner^{[1]}
      }
    }
    $%
  }
\end{center}
\bigskip

\end{document}

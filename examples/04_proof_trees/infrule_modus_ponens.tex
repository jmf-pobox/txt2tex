\documentclass[a4paper,10pt,fleqn]{article}
\usepackage[margin=1in]{geometry}
\usepackage{amssymb}
\usepackage{adjustbox}
\usepackage{natbib}
\usepackage[colorlinks=true,linkcolor=blue,citecolor=blue,urlcolor=blue]{hyperref}
\usepackage{fuzz}
\usepackage{zed-maths}
\usepackage{zed-proof}
\newdimen\savedleftskip
\begin{document}

\noindent Modus ponens is a fundamental inference rule in propositional logic.

\bigskip

\noindent If we know P is true and P implies Q, we can conclude Q is true.

\bigskip

\begin{infrule}
  \power & \\
  \power \implies Q & \\
  \derive \\
  Q & \mbox{modus ponens}
\end{infrule}

\noindent The horizontal line (generated by ---) separates premises
from conclusion.

\bigskip

\noindent The label in brackets appears to the right of the conclusion.

\bigskip

\noindent Here's a more complex example with multiple premises:

\bigskip

\begin{infrule}
  \power \implies Q & \\
  Q \implies R & \\
  \power & \\
  \derive \\
  R & \mbox{chain of implications}
\end{infrule}

\noindent We can also use INFRULE without labels:

\bigskip

\begin{infrule}
  A \land B & \\
  \derive \\
  A & \mbox{conjunction elimination}
\end{infrule}

\end{document}

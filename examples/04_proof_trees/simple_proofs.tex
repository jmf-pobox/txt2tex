\documentclass[a4paper,10pt,fleqn]{article}
\usepackage[margin=1in]{geometry}
\usepackage{amssymb}
\usepackage{adjustbox}
\usepackage{natbib}
\usepackage[colorlinks=true,linkcolor=blue,citecolor=blue,urlcolor=blue]{hyperref}
\usepackage{fuzz}
\usepackage{zed-maths}
\usepackage{zed-proof}
\newdimen\savedleftskip
\begin{document}

\section*{Phase 5 : Proof Trees}

\section*{Example 1 : Simple Implication}
\addcontentsline{toc}{section}{Example 1 : Simple Implication}

\begin{center}
  \adjustbox{max width=\textwidth}{%
    $\displaystyle
    \infer[\Rightarrow\textrm{-intro}^{[1]}]{p \land q \implies q}{
      \infer[\land\textrm{-elim-2}]{q}{\ulcorner p \land q \urcorner^{[1]}}
    }
    $%
  }
\end{center}
\bigskip

\section*{Example 2 : With Sibling Premises}
\addcontentsline{toc}{section}{Example 2 : With Sibling Premises}

\begin{center}
  \adjustbox{max width=\textwidth}{%
    $\displaystyle
    \infer[\Rightarrow\textrm{-intro}^{[1]}]{p \land (p \implies q)
    \implies (p \land q)}{
      \infer[\land \mbox{intro}]{p \land q}{\infer[\Rightarrow
        \mbox{elim}]{q}{\infer[\land\textrm{-elim-1}]{p}{\ulcorner p
          \land (p \implies q) \urcorner^{[1]}} &
          \infer[\land\textrm{-elim-2}]{p \implies q}{\ulcorner p \land
      (p \implies q) \urcorner^{[1]}}}}
    }
    $%
  }
\end{center}
\bigskip

\section*{Example 3 : Distribution with Cases}
\addcontentsline{toc}{section}{Example 3 : Distribution with Cases}

\begin{center}
  \adjustbox{max width=\textwidth}{%
    $\displaystyle
    \infer[\Rightarrow\textrm{-intro}^{[1]}]{p \land (q \lor r)
    \implies (p \land q) \lor (p \land r)}{
      \infer[\lor \mbox{elim}]{(p \land q) \lor (p \land
      r)}{\infer[\lor\textrm{-intro-1}]{(p \land q) \lor (p \land
        r)}{\infer[\land \mbox{intro}]{p \land q}{\infer[\mbox{from
        case}]{q}{\infer[\mbox{from above}]{p}{}}}} &
        \infer[\lor\textrm{-intro-2}]{(p \land q) \lor (p \land
        r)}{\infer[\land \mbox{intro}]{p \land r}{\infer[\mbox{from
      case}]{r}{\infer[\mbox{from above}]{p}{}}}}}
    }
    $%
  }
\end{center}
\bigskip

\section*{Example 4 : Modus Tollens}
\addcontentsline{toc}{section}{Example 4 : Modus Tollens}

\begin{center}
  \adjustbox{max width=\textwidth}{%
    $\displaystyle
    \infer[\Rightarrow\textrm{-intro}^{[1]}]{(p \implies q) \land
    \lnot q \implies \lnot p}{
      \infer[negation\textrm{-intro}^{[2]}]{\lnot
      p}{\infer[\land\textrm{-elim-1}]{p \implies q}{\ulcorner (p
        \implies q) \land \lnot q \urcorner^{[1]}} &
        \infer[\land\textrm{-elim-2}]{\lnot q}{\ulcorner (p \implies q)
        \land \lnot q \urcorner^{[1]}} &
        \infer[\mbox{contradiction}]{false}{\infer[\Rightarrow
      \mbox{elim}]{q}{\ulcorner p \urcorner^{[2]}}}}
    }
    $%
  }
\end{center}
\bigskip

\section*{Example 5 : Solution 18 Implication to Disjunction}
\addcontentsline{toc}{section}{Example 5 : Solution 18 Implication to
Disjunction}

\begin{center}
  \adjustbox{max width=\textwidth}{%
    $\displaystyle
    \infer[\Rightarrow\textrm{-intro}^{[1]}]{(p \implies q) \implies
    (\lnot p \lor q)}{
      \infer[\mbox{excluded middle}]{p \lor \lnot p}{\ulcorner p
      \implies q \urcorner^{[1]}} &
      \infer[\lor\textrm{-elim}^{[2]}]{\lnot p \lor
      q}{\infer[\lor\textrm{-intro-2}]{\lnot p \lor
        q}{\infer[\Rightarrow \mbox{elim}]{q}{\ulcorner p
        \urcorner^{[2]}}} & \infer[\lor\textrm{-intro-1}]{\lnot p \lor
      q}{\ulcorner \lnot p \urcorner^{[2]}}}
    }
    $%
  }
\end{center}
\bigskip

\end{document}

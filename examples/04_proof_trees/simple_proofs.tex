\documentclass[a4paper,10pt,fleqn]{article}
\usepackage[margin=1in]{geometry}
\usepackage{amssymb}
\usepackage{zed-cm}
\usepackage{zed-maths}
\usepackage{zed-proof}
\begin{document}

\section*{Phase 5 : Proof Trees}

\bigskip
\noindent
\textbf{Example 1 : Simple Implication}

\medskip

\noindent
$\displaystyle
\infer[\Rightarrow\textrm{-intro}^{[1]}]{p \land q \Rightarrow q}{
  \infer[\land\textrm{-elim-2}]{q}{\ulcorner p \land q \urcorner^{[1]}}
}
$

\bigskip
\noindent
\textbf{Example 2 : With Sibling Premises}

\medskip

\noindent
$\displaystyle
\infer[\Rightarrow\textrm{-intro}^{[1]}]{p \land (p \Rightarrow q) \Rightarrow p \land q}{
  \infer[\land \mathrm{intro}]{p \land q}{\infer[\Rightarrow \mathrm{elim}]{q}{\infer[\land\textrm{-elim-1}]{p}{\ulcorner p \land (p \Rightarrow q) \urcorner^{[1]}} & \infer[\land\textrm{-elim-2}]{p \Rightarrow q}{\ulcorner p \land (p \Rightarrow q) \urcorner^{[1]}}}}
}
$

\bigskip
\noindent
\textbf{Example 3 : Distribution with Cases}

\medskip

\noindent
$\displaystyle
\infer[\Rightarrow\textrm{-intro}^{[1]}]{p \land (q \lor r) \Rightarrow p \land q \lor p \land r}{
  \infer[\lor \mathrm{elim}]{p \land q \lor p \land r}{\infer[\lor\textrm{-intro-1}]{p \land q \lor p \land r}{\infer[\land \mathrm{intro}]{p \land q}{\infer[\mathrm{from} \mathrm{case}]{q}{p}}} & \infer[\lor\textrm{-intro-2}]{p \land q \lor p \land r}{\infer[\land \mathrm{intro}]{p \land r}{\infer[\mathrm{from} \mathrm{case}]{r}{p}}}}
}
$

\bigskip
\noindent
\textbf{Example 4 : Modus Tollens}

\medskip

\noindent
$\displaystyle
\infer[\Rightarrow\textrm{-intro}^{[1]}]{(p \Rightarrow q) \land \lnot q \Rightarrow \lnot p}{
  \infer[negation\textrm{-intro}^{[2]}]{\lnot p}{\infer[\land\textrm{-elim-1}]{p \Rightarrow q}{\ulcorner (p \Rightarrow q) \land \lnot q \urcorner^{[1]}} & \infer[\land\textrm{-elim-2}]{\lnot q}{\ulcorner (p \Rightarrow q) \land \lnot q \urcorner^{[1]}} & \infer[\mathrm{contradiction}]{false}{\infer[\Rightarrow \mathrm{elim}]{q}{\ulcorner p \urcorner^{[2]}}}}
}
$

\bigskip
\noindent
\textbf{Example 5 : Solution 18 Implication to Disjunction}

\medskip

\noindent
$\displaystyle
\infer[\Rightarrow\textrm{-intro}^{[1]}]{(p \Rightarrow q) \Rightarrow \lnot p \lor q}{
  \infer[\mathrm{excluded} \mathrm{middle}]{p \lor \lnot p}{\ulcorner p \Rightarrow q \urcorner^{[1]}} & \infer[\lor\textrm{-elim}^{[2]}]{\lnot p \lor q}{\infer[\lor\textrm{-intro-2}]{\lnot p \lor q}{\infer[\Rightarrow \mathrm{elim}]{q}{\ulcorner p \urcorner^{[2]}}} & \infer[\lor\textrm{-intro-1}]{\lnot p \lor q}{\ulcorner \lnot p \urcorner^{[2]}}}
}
$

\end{document}
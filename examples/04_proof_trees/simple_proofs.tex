\documentclass[a4paper,10pt,fleqn]{article}
\usepackage[margin=1in]{geometry}
\usepackage{amssymb}
\usepackage{adjustbox}
\usepackage{natbib}
\usepackage[colorlinks=true,linkcolor=blue,citecolor=blue,urlcolor=blue]{hyperref}
\usepackage{fuzz}
\usepackage{zed-maths}
\usepackage{zed-proof}
\newdimen\savedleftskip
\begin{document}

\section*{Phase 5 : Proof Trees}

\section*{Example 1 : Simple Implication}
\addcontentsline{toc}{section}{Example 1 : Simple Implication}

\begin{center}
\adjustbox{max width=\textwidth}{%
$\displaystyle
\infer[\Rightarrow\textrm{-intro}^{[1]}]{p(and)(q) \implies q}{
  \infer[\land\textrm{-elim-2}]{q}{\ulcorner p(and)(q) \urcorner^{[1]}}
}
$%
}
\end{center}
\bigskip

\section*{Example 2 : With Sibling Premises}
\addcontentsline{toc}{section}{Example 2 : With Sibling Premises}

\begin{center}
\adjustbox{max width=\textwidth}{%
$\displaystyle
\infer[\Rightarrow\textrm{-intro}^{[1]}]{p(and(p \implies q)) \implies p(and)(q)}{
  \infer[\land \mbox{intro}]{p(and)(q)}{\infer[\Rightarrow \mbox{elim}]{q}{\infer[\land\textrm{-elim-1}]{p}{\ulcorner p(and(p \implies q)) \urcorner^{[1]}} & \infer[\land\textrm{-elim-2}]{p \implies q}{\ulcorner p(and(p \implies q)) \urcorner^{[1]}}}}
}
$%
}
\end{center}
\bigskip

\section*{Example 3 : Distribution with Cases}
\addcontentsline{toc}{section}{Example 3 : Distribution with Cases}

\begin{center}
\adjustbox{max width=\textwidth}{%
$\displaystyle
\infer[\Rightarrow\textrm{-intro}^{[1]}]{p(and(q(or)(r))) \implies p(and)(q)(or(p(and)(r)))}{
  \infer[\lor \mbox{elim}]{p(and)(q)(or(p(and)(r)))}{\infer[\lor\textrm{-intro-1}]{p(and)(q)(or(p(and)(r)))}{\infer[\land \mbox{intro}]{p(and)(q)}{\infer[\mbox{from case}]{q}{\infer[\mbox{from above}]{p}{}}}} & \infer[\lor\textrm{-intro-2}]{p(and)(q)(or(p(and)(r)))}{\infer[\land \mbox{intro}]{p(and)(r)}{\infer[\mbox{from case}]{r}{\infer[\mbox{from above}]{p}{}}}}}
}
$%
}
\end{center}
\bigskip

\section*{Example 4 : Modus Tollens}
\addcontentsline{toc}{section}{Example 4 : Modus Tollens}

\begin{center}
\adjustbox{max width=\textwidth}{%
$\displaystyle
\infer[\Rightarrow\textrm{-intro}^{[1]}]{((p \implies q))(and)(not)(q) \implies not(p)}{
  \infer[negation\textrm{-intro}^{[2]}]{not(p)}{\infer[\land\textrm{-elim-1}]{p \implies q}{\ulcorner ((p \implies q))(and)(not)(q) \urcorner^{[1]}} & \infer[\land\textrm{-elim-2}]{not(q)}{\ulcorner ((p \implies q))(and)(not)(q) \urcorner^{[1]}} & \infer[\mbox{contradiction}]{false}{\infer[\Rightarrow \mbox{elim}]{q}{\ulcorner p \urcorner^{[2]}}}}
}
$%
}
\end{center}
\bigskip

\section*{Example 5 : Solution 18 Implication to Disjunction}
\addcontentsline{toc}{section}{Example 5 : Solution 18 Implication to Disjunction}

\begin{center}
\adjustbox{max width=\textwidth}{%
$\displaystyle
\infer[\Rightarrow\textrm{-intro}^{[1]}]{(p \implies q) \implies not(p)(or)(q)}{
  \infer[\mbox{excluded middle}]{p(or)(not)(p)}{\ulcorner p \implies q \urcorner^{[1]}} & \infer[\lor\textrm{-elim}^{[2]}]{not(p)(or)(q)}{\infer[\lor\textrm{-intro-2}]{not(p)(or)(q)}{\infer[\Rightarrow \mbox{elim}]{q}{\ulcorner p \urcorner^{[2]}}} & \infer[\lor\textrm{-intro-1}]{not(p)(or)(q)}{\ulcorner not(p) \urcorner^{[2]}}}
}
$%
}
\end{center}
\bigskip

\end{document}
\documentclass[a4paper,10pt,fleqn]{article}
\usepackage[margin=1in]{geometry}
\usepackage{amssymb}
\usepackage{adjustbox}
\usepackage{natbib}
\usepackage[colorlinks=true,linkcolor=blue,citecolor=blue,urlcolor=blue]{hyperref}
\usepackage{fuzz}
\usepackage{zed-maths}
\usepackage{zed-proof}
\newdimen\savedleftskip
\begin{document}

\section*{Advanced Proof Patterns}

\section*{Example 1 : Proof by Cases ( Three - Way Split )}
\addcontentsline{toc}{section}{Example 1 : Proof by Cases ( Three - Way Split )}

\noindent Prove: For all integers n, n $\langle 0(or)(n) = 0(or)(n) \rangle$ 0

\bigskip

\begin{center}
\adjustbox{max width=\textwidth}{%
$\displaystyle
\infer[\mbox{trichotomy of integers}, \mbox{axiom}]{n < 0 \lor n = 0 \lor n > 0}{}
$%
}
\end{center}
\bigskip

\noindent This uses the trichotomy property as an axiom. To prove something about all integers, we can case-analyze on these three possibilities.

\bigskip

\section*{Example 2 : Multi - Level Case Analysis}
\addcontentsline{toc}{section}{Example 2 : Multi - Level Case Analysis}

\noindent Prove: $(p \lor q)$ $\land$ $(r \lor s)$ $\Rightarrow$ $(p \land r)$ $\lor$ $(p \land s)$ $\lor$ $(q \land r)$ $\lor$ $(q \land s)$

\bigskip

\begin{center}
\adjustbox{max width=\textwidth}{%
$\displaystyle
\infer[\Rightarrow\textrm{-intro}^{[1]}]{((p \lor q) \land (r \lor s)) \implies (p \land r) \lor (p \land s) \lor (q \land r) \lor (q \land s)}{
  \infer[\lor \mbox{elim on p} \lor q]{(p \land r) \lor (p \land s) \lor (q \land r) \lor (q \land s)}{\infer[\lor \mbox{elim on r} \lor s]{(p \land r) \lor (p \land s) \lor (q \land r) \lor (q \land s)}{
  \raiseproof{10ex}{\infer[\lor \mbox{intro}]{(p \land r) \lor (p \land s) \lor (q \land r) \lor (q \land s)}{\infer[\land \mbox{intro}]{p \land r}{}}}
&
\hskip 6em \raiseproof{26ex}{\infer[\lor \mbox{intro}]{(p \land r) \lor (p \land s) \lor (q \land r) \lor (q \land s)}{\infer[\land \mbox{intro}]{p \land s}{}}}
} & \infer[\lor \mbox{elim on r} \lor s]{(p \land r) \lor (p \land s) \lor (q \land r) \lor (q \land s)}{
  \raiseproof{10ex}{\infer[\lor \mbox{intro}]{(p \land r) \lor (p \land s) \lor (q \land r) \lor (q \land s)}{\infer[\land \mbox{intro}]{q \land r}{}}}
&
\hskip 6em \raiseproof{26ex}{\infer[\lor \mbox{intro}]{(p \land r) \lor (p \land s) \lor (q \land r) \lor (q \land s)}{\infer[\land \mbox{intro}]{q \land s}{}}}
}}
}
$%
}
\end{center}
\bigskip

\noindent Nested case analysis on two disjunctions, exploring all four combinations.

\bigskip

\section*{Example 3 : Proof by Mathematical Induction ( Base land Step )}
\addcontentsline{toc}{section}{Example 3 : Proof by Mathematical Induction ( Base land Step )}

\noindent Prove: For all n ∈ ℕ, sum(1 to n) = n(n+1)/2

\bigskip

\noindent Base case ($n = 0$):

\bigskip

\begin{center}
\adjustbox{max width=\textwidth}{%
$\displaystyle
\infer[\Rightarrow\textrm{-intro}^{[1]}]{true \implies (sum\_{to}(0) = 0 * ((0 + 1))(div)(2))}{
  \infer[\mbox{equality}]{sum\_{to}(0) = 0 * ((0 + 1))(div)(2)}{\infer[\mbox{arithmetic}]{0 * ((0 + 1))(div)(2) = 0}{\infer[\mbox{definition}]{sum\_{to}(0) = 0}{\ulcorner true \urcorner^{[1]}}}}
}
$%
}
\end{center}
\bigskip

\noindent Inductive step (assume for n, prove for n+1):

\bigskip

\begin{center}
\adjustbox{max width=\textwidth}{%
$\displaystyle
\infer[\Rightarrow\textrm{-intro}^{[1]}]{(sum\_{to}(n) = n * ((n + 1))(div)(2)) \implies (sum\_{to}(n + 1) = (n + 1) * ((n + 2))(div)(2))}{
  \infer[\mbox{factoring}]{sum\_{to}(n + 1) = (n + 1) * ((n + 2))(div)(2)}{\infer[\mbox{algebra}]{sum\_{to}(n + 1) = ((n * (n + 1) + 2 * (n + 1)))(div)(2)}{\infer[\mbox{substitution}]{sum\_{to}(n + 1) = n * ((n + 1))(div)(2) + (n + 1)}{\infer[\mbox{definition}]{sum\_{to}(n + 1) = sum\_{to}(n) + (n + 1)}{\ulcorner sum\_{to}(n) = n * ((n + 1))(div)(2) \urcorner^{[1]}}}}}
}
$%
}
\end{center}
\bigskip

\noindent By induction, the formula holds for all natural numbers.

\bigskip

\section*{Example 4 : Structural Induction on Lists}
\addcontentsline{toc}{section}{Example 4 : Structural Induction on Lists}

\noindent Prove: For all sequences s, reverse(reverse(s)) = s

\bigskip

\noindent Base case (empty sequence):

\bigskip

\begin{center}
\adjustbox{max width=\textwidth}{%
$\displaystyle
\infer[\Rightarrow\textrm{-intro}^{[1]}]{true \implies (reverse(reverse(emptyseq)) = emptyseq)}{
  \infer[\mbox{definition}]{reverse(reverse(emptyseq)) = emptyseq}{\infer[\mbox{substitution}]{reverse(reverse(emptyseq)) = reverse(emptyseq)}{\infer[\mbox{definition}]{reverse(emptyseq) = emptyseq}{\ulcorner true \urcorner^{[1]}}}}
}
$%
}
\end{center}
\bigskip

\noindent Inductive step (assume for s, prove for cons(x, s)):

\bigskip

\begin{center}
\adjustbox{max width=\textwidth}{%
$\displaystyle
\infer[\Rightarrow\textrm{-intro}^{[1]}]{(reverse(reverse(s)) = s) \implies (reverse(reverse(cons(x, s))) = cons(x, s))}{
  \infer[\mbox{inductive hypothesis}]{cons(x, reverse(reverse(s))) = cons(x, s)}{\infer[\mbox{definition}]{reverse(append(reverse(s), x)) = cons(x, reverse(reverse(s)))}{\infer[\mbox{substitution}]{reverse(reverse(cons(x, s))) = reverse(append(reverse(s), x))}{\infer[\mbox{definition}]{reverse(cons(x, s)) = append(reverse(s), x)}{\ulcorner reverse(reverse(s)) = s \urcorner^{[1]}}}}}
}
$%
}
\end{center}
\bigskip

\noindent By structural induction, reverse(reverse(s)) = s for all sequences.

\bigskip

\section*{Example 5 : Constructive Existence Proof}
\addcontentsline{toc}{section}{Example 5 : Constructive Existence Proof}

\noindent Prove: There $\exists$ an even number greater than 10

\bigskip

\begin{center}
\adjustbox{max width=\textwidth}{%
$\displaystyle
\infer[\Rightarrow\textrm{-intro}^{[1]}]{true \implies (\exists n : \nat @ even(n) \land n > 10)}{
  \infer[\exists \mbox{intro with n}=12]{\exists n : \nat @ even(n) \land n > 10}{\infer[\land \mbox{intro}]{even(12) \land 12 > 10}{\infer[\mbox{arithmetic}]{12 > 10}{\infer[\mbox{definition of even}, 12=2 * 6]{even(12)}{\infer[\mbox{arithmetic}]{12 = 2 * 6}{\ulcorner true \urcorner^{[1]}}}}}}
}
$%
}
\end{center}
\bigskip

\noindent Constructive proof: we exhibit a specific witness (12).

\bigskip

\section*{Example 6 : Non - Constructive Existence Proof}
\addcontentsline{toc}{section}{Example 6 : Non - Constructive Existence Proof}

\noindent Prove: There exist irrational numbers a and b such that $a^b$ is rational

\bigskip

\begin{center}
\adjustbox{max width=\textwidth}{%
$\displaystyle
\infer[\Rightarrow\textrm{-intro}^{[1]}]{irrational(sqrt(2)) \implies (\exists a, b @ irrational(a) \land irrational(b) \land rational(power(a, b)))}{
  \infer[\lor \mbox{elim}]{\exists a, b @ irrational(a) \land irrational(b) \land rational(power(a, b))}{\infer[\exists \mbox{intro with a}=b=sqrt(2)]{\exists a, b @ irrational(a) \land irrational(b) \land rational(power(a, b))}{\infer[\land \mbox{intro}]{irrational(sqrt(2)) \land irrational(sqrt(2)) \land rational(power(sqrt(2), sqrt(2)))}{}} & \infer[\exists \mbox{intro}]{\exists a, b @ irrational(a) \land irrational(b) \land rational(power(a, b))}{\infer[\land \mbox{intro}]{irrational(power(sqrt(2), sqrt(2))) \land irrational(sqrt(2)) \land rational(power(power(sqrt(2), sqrt(2)), sqrt(2)))}{\infer[\mbox{known}]{rational(2)}{\infer[\mbox{simplification}]{power(sqrt(2), 2) = 2}{\infer[\mbox{arithmetic}]{power(sqrt(2), sqrt(2) * sqrt(2)) = power(sqrt(2), 2)}{\infer[\mbox{exponent law}]{power(power(sqrt(2), sqrt(2)), sqrt(2)) = power(sqrt(2), sqrt(2) * sqrt(2))}{}}}}}}}
}
$%
}
\end{center}
\bigskip

\noindent Non-constructive: we don't know which case is true, but both lead to the conclusion.

\bigskip

\section*{Example 7 : Proof by Strong Induction}
\addcontentsline{toc}{section}{Example 7 : Proof by Strong Induction}

\noindent Prove: Every natural number n ≥ 2 has a prime factorization

\bigskip

\noindent Base case ($n = 2$):

\bigskip

\begin{center}
\adjustbox{max width=\textwidth}{%
$\displaystyle
\infer[\Rightarrow\textrm{-intro}^{[1]}]{true \implies prime\_factorization(2)}{
  \infer[\mbox{trivial}, \mbox{singleton factorization}]{prime\_factorization(2)}{\infer[\mbox{definition}]{prime(2)}{\ulcorner true \urcorner^{[1]}}}
}
$%
}
\end{center}
\bigskip

\noindent Inductive step (assume for all $k < n$, prove for n):

\bigskip

\begin{center}
\adjustbox{max width=\textwidth}{%
$\displaystyle
\infer[\Rightarrow\textrm{-intro}^{[1]}]{(n \geq 2) \implies prime\_factorization(n)}{
  \infer[\lor \mbox{elim}]{prime\_factorization(n)}{\infer[\mbox{trivial}, \mbox{singleton factorization}]{prime\_factorization(n)}{} & \infer[n=a * b]{prime\_factorization(n)}{\infer[\mbox{multiplication of factorizations}]{prime\_factorization(a * b)}{\infer[\mbox{strong IH}, b < n]{prime\_factorization(b)}{\infer[\mbox{strong IH}, a < n]{prime\_factorization(a)}{\infer[\mbox{definition of composite}]{\exists a, b @ 2 \leq a \land a < n \land 2 \leq b \land b < n \land n = a * b}{}}}}}}
}
$%
}
\end{center}
\bigskip

\noindent Strong induction: we assume the property for all smaller values, not just n-1.

\bigskip

\section*{Example 8 : Proof Using Lemmas}
\addcontentsline{toc}{section}{Example 8 : Proof Using Lemmas}

\noindent Lemma 1: If n is even, then $n^2$ is even

\bigskip

\begin{center}
\adjustbox{max width=\textwidth}{%
$\displaystyle
\infer[\Rightarrow\textrm{-intro}^{[1]}]{even(n) \implies even(power(n, 2))}{
  \infer[\mbox{definition of even}]{even(power(n, 2))}{\infer[\exists \mbox{intro with m}=2 * power(k, 2)]{\exists m @ power(n, 2) = 2 * m}{\infer[\mbox{factoring}]{4 * power(k, 2) = 2 * (2 * power(k, 2))}{\infer[\mbox{algebra}]{power(2 * k, 2) = 4 * power(k, 2)}{\infer[\mbox{substitution}]{power(n, 2) = power(2 * k, 2)}{\infer[\mbox{definition of even}]{\exists k @ n = 2 * k}{\ulcorner even(n) \urcorner^{[1]}}}}}}}
}
$%
}
\end{center}
\bigskip

\noindent Main theorem: If $n^2$ is odd, then n is odd

\bigskip

\begin{center}
\adjustbox{max width=\textwidth}{%
$\displaystyle
\infer[\Rightarrow\textrm{-intro}^{[1]}]{odd(power(n, 2)) \implies odd(n)}{
  \infer[\lor \mbox{elim}]{odd(n)}{\infer[\mbox{false elim}]{odd(n)}{\infer[\mbox{contradiction}]{false}{\infer[\mbox{contradiction}]{odd(power(n, 2)) \land even(power(n, 2))}{\infer[Lemma 1]{even(power(n, 2))}{}}}} & \infer[\mbox{identity}]{odd(n)}{}}
}
$%
}
\end{center}
\bigskip

\noindent Proof by contrapositive using lemma.

\bigskip

\section*{Example 9 : Proof by Minimal Counterexample}
\addcontentsline{toc}{section}{Example 9 : Proof by Minimal Counterexample}

\noindent Prove: All natural numbers n ≥ 1 satisfy P(n)

\bigskip

\begin{center}
\adjustbox{max width=\textwidth}{%
$\displaystyle
\infer[\Rightarrow\textrm{-intro}^{[1]}]{true \implies (\forall n @ n \geq 1 \implies \power n)}{
  \infer[\lnot\textrm{-intro}^{[2]}]{\forall n @ n \geq 1 \implies \power n}{\infer[\lor \mbox{elim}]{false}{\infer[\mbox{contradiction}]{false}{\infer[\mbox{contradiction}]{\lnot \power m \land \power 1}{\infer[\mbox{base \mbox{case} \mbox{proved} separately}]{\power 1}{}}} & \infer[\mbox{contradiction}]{false}{\infer[\mbox{contradiction}]{\lnot \power m \land \power m}{\infer[\mbox{by \mbox{inductive} \mbox{step} \mbox{from} P}(m - 1)]{\power m}{\infer[\mbox{since m} - 1 >=1 \land m - 1 < m]{\power (m - 1)}{\infer[\mbox{minimality of m}]{\forall k @ k \geq 1 \land k < m \implies \power k}{}}}}}}}
}
$%
}
\end{center}
\bigskip

\noindent Minimal counterexample combines well-ordering with contradiction.

\bigskip

\section*{Example 10 : Proof by Invariant}
\addcontentsline{toc}{section}{Example 10 : Proof by Invariant}

\noindent Prove: A loop maintains invariant I

\bigskip

\noindent Initialization:

\bigskip

\begin{center}
\adjustbox{max width=\textwidth}{%
$\displaystyle
\infer[\Rightarrow\textrm{-intro}^{[1]}]{initial\_state \implies invariant(initial\_state)}{
  \infer[\mbox{verification}]{invariant(initial\_state)}{\ulcorner initial\_state \urcorner^{[1]}}
}
$%
}
\end{center}
\bigskip

\noindent Preservation:

\bigskip

\begin{center}
\adjustbox{max width=\textwidth}{%
$\displaystyle
\infer[\Rightarrow\textrm{-intro}^{[1]}]{(invariant(before\_state) \land executes\_loop\_body) \implies invariant(after\_state)}{
  \infer[\mbox{verification}]{invariant(after\_state)}{\infer[\land\textrm{-elim-2}]{executes\_loop\_body}{\infer[\land\textrm{-elim-1}]{invariant(before\_state)}{\ulcorner invariant(before\_state) \land executes\_loop\_body \urcorner^{[1]}}}}
}
$%
}
\end{center}
\bigskip

\noindent Termination:

\bigskip

\begin{center}
\adjustbox{max width=\textwidth}{%
$\displaystyle
\infer[\Rightarrow\textrm{-intro}^{[1]}]{loop\_terminates \implies desired\_property}{
  \infer[\mbox{logic}]{desired\_property}{\infer[\land \mbox{intro}]{invariant(termination\_state) \land termination\_condition}{\infer[\mbox{by preservation}]{invariant(termination\_state)}{\ulcorner loop\_terminates \urcorner^{[1]}}}}
}
$%
}
\end{center}
\bigskip

\section*{Example 11 : Proof by Diagonalization}
\addcontentsline{toc}{section}{Example 11 : Proof by Diagonalization}

\noindent Prove: The set of real numbers is uncountable

\bigskip

\begin{center}
\adjustbox{max width=\textwidth}{%
$\displaystyle
\infer[\Rightarrow\textrm{-intro}^{[1]}]{true \implies uncountable(reals)}{
  \infer[\lnot\textrm{-intro}^{[2]}]{uncountable(reals)}{\infer[\mbox{contradiction}]{false}{\infer[\mbox{contradiction}]{not\_in\_range(r, f) \land enumeration(f, reals)}{\infer[\mbox{previous line}]{not\_in\_range(r, f)}{\infer[\mbox{by construction}, \mbox{differs at nth digit}]{\forall n @ r \neq apply(f, n)}{\infer[\mbox{diagonal method}]{diagonal\_construction(r)}{\infer[\mbox{definition of countable}]{\exists f @ enumeration(f, reals)}{\ulcorner countable(reals) \urcorner^{[2]}}}}}}}}
}
$%
}
\end{center}
\bigskip

\noindent Cantor's diagonal argument (outline).

\bigskip

\section*{Example 12 : Constructive Proof Pattern}
\addcontentsline{toc}{section}{Example 12 : Constructive Proof Pattern}

\noindent To constructively prove: $\exists x @ \power x$

\bigskip

\noindent Strategy:

\bigskip

\noindent 1. Explicitly construct a witness w

\bigskip

\noindent 2. Verify P(w) holds

\bigskip

\noindent 3. Conclude $\exists x @ \power x$with $x = w$

\bigskip

\noindent Example: Prove $\exists n : \nat @ n > 100$$\land$ n is even

\bigskip

\begin{center}
\adjustbox{max width=\textwidth}{%
$\displaystyle
\infer[\Rightarrow\textrm{-intro}^{[1]}]{true \implies (\exists n @ n > 100 \land even(n))}{
  \infer[\exists \mbox{intro with n}=102]{\exists n @ n > 100 \land even(n)}{\infer[\land \mbox{intro}]{102 > 100 \land even(102)}{\infer[\mbox{definition}]{even(102)}{\infer[\mbox{arithmetic}]{102 = 2 * 51}{\infer[\mbox{arithmetic}]{102 > 100}{\infer[\mbox{construction}]{witness\_construction(102)}{\ulcorner true \urcorner^{[1]}}}}}}}
}
$%
}
\end{center}
\bigskip

\section*{Example 13 : Proof Composition}
\addcontentsline{toc}{section}{Example 13 : Proof Composition}

\noindent Combine multiple proof techniques:

\bigskip

\noindent Theorem: Property P holds for all cases

\bigskip

\noindent Overall strategy: Case analysis + Induction + Contradiction

\bigskip

\begin{center}
\adjustbox{max width=\textwidth}{%
$\displaystyle
\infer[\Rightarrow\textrm{-intro}^{[1]}]{true \implies (\forall n @ \power n)}{
  \infer[\lor \mbox{elim over cases}]{\forall n @ \power n}{\infer[\mbox{direct proof}]{\power base}{} & \infer[\lor \mbox{elim over subcases}]{\power n}{
  \raiseproof{12ex}{\infer[\lnot\textrm{-intro}^{[2]}]{\power n}{
  \infer[\mbox{contradiction}]{false}{\infer[\mbox{proof steps}]{contradiction\_derived}{\ulcorner \lnot \power n \urcorner^{[2]}}}
}}
&
\hskip 6em \raiseproof{30ex}{\infer[\Rightarrow\textrm{-intro}^{[3]}]{\power n}{
  \infer[\mbox{inductive step}]{\power n}{\ulcorner \power (n - 1) \urcorner^{[3]}}
}}
}}
}
$%
}
\end{center}
\bigskip

\section*{Example 14 : Best Practices for Complex Proofs}
\addcontentsline{toc}{section}{Example 14 : Best Practices for Complex Proofs}

\noindent Guidelines for writing advanced proofs:

\bigskip

\noindent 1. State strategy at the beginning

\bigskip

\noindent 2. Label cases clearly

\bigskip

\noindent 3. Discharge assumptions promptly

\bigskip

\noindent 4. Reference lemmas explicitly

\bigskip

\noindent 5. Show key algebraic steps

\bigskip

\noindent 6. Justify non-obvious steps

\bigskip

\noindent 7. Use proof by cases when structure suggests it

\bigskip

\noindent 8. Use induction for recursive definitions

\bigskip

\noindent 9. Use contradiction for negative conclusions

\bigskip

\noindent 10. Verify base cases thoroughly

\bigskip

\section*{Example 15 : Proof Documentation}
\addcontentsline{toc}{section}{Example 15 : Proof Documentation}

\noindent Document complex proofs:

\bigskip

\noindent - **Goal**: State what you're proving

\bigskip

\noindent - **Strategy**: Explain the proof approach

\bigskip

\noindent - **Lemmas needed**: List dependencies

\bigskip

\noindent - **Key insights**: Highlight non-obvious steps

\bigskip

\noindent - **Pitfalls**: Note where proof could go wrong

\bigskip

\noindent - **Generalization**: Explain how proof extends

\bigskip

\noindent Well-documented proofs are maintainable and reusable.

\bigskip

\end{document}
\documentclass[a4paper,10pt,fleqn]{article}
\usepackage[margin=1in]{geometry}
\usepackage{amssymb}
\usepackage{adjustbox}
\usepackage{natbib}
\usepackage[colorlinks=true,linkcolor=blue,citecolor=blue,urlcolor=blue]{hyperref}
\usepackage{fuzz}
\usepackage{zed-maths}
\usepackage{zed-proof}
\newdimen\savedleftskip
\begin{document}

\section*{Proof by Contradiction Examples}

\section*{Example 1 : Basic Contradiction}
\addcontentsline{toc}{section}{Example 1 : Basic Contradiction}

\noindent To prove p, assume $\lnot p$ and derive a contradiction (false).

\bigskip

\begin{center}
  \adjustbox{max width=\textwidth}{%
    $\displaystyle
    \infer[\Rightarrow\textrm{-intro}^{[1]}]{(\lnot p \land (q \land
    \lnot q)) \implies p}{
      \infer[\mbox{false
      elim}]{p}{\infer[\mbox{contradiction}]{false}{\infer[\land\textrm{-elim-2}]{\lnot
          q}{\infer[\land\textrm{-elim-1}]{q}{\infer[\land\textrm{-elim-2}]{q
              \land \lnot q}{\ulcorner \lnot p \land (q \land \lnot q)
      \urcorner^{[1]}}}}}}
    }
    $%
  }
\end{center}
\bigskip

\noindent Once we derive false, we can conclude anything—here we
conclude p by discharging assumption 1.

\bigskip

\section*{Example 2 : Simple Example - Proving q}
\addcontentsline{toc}{section}{Example 2 : Simple Example - Proving q}

\noindent Prove: $(p \land \lnot p)$ $\Rightarrow$ q

\bigskip

\begin{center}
  \adjustbox{max width=\textwidth}{%
    $\displaystyle
    \infer[\Rightarrow\textrm{-intro}^{[1]}]{(p \land \lnot p) \implies q}{
      \infer[\mbox{false
      elim}]{q}{\infer[\mbox{contradiction}]{false}{\infer[\land\textrm{-elim-2}]{\lnot
          p}{\infer[\land\textrm{-elim-1}]{p}{\ulcorner p \land \lnot p
      \urcorner^{[1]}}}}}
    }
    $%
  }
\end{center}
\bigskip

\noindent From a contradiction, we can derive anything (ex falso quodlibet).

\bigskip

\section*{Example 3 : Law of Non - Contradiction}
\addcontentsline{toc}{section}{Example 3 : Law of Non - Contradiction}

\noindent Prove that p $\land$ $\lnot$ p leads to a contradiction.

\bigskip

\begin{center}
  \adjustbox{max width=\textwidth}{%
    $\displaystyle
    \infer[\Rightarrow\textrm{-intro}^{[1]}]{p \land \lnot p \implies false}{
      \infer[\mbox{contradiction}]{false}{\ulcorner p \land \lnot p
      \urcorner^{[1]}}
    }
    $%
  }
\end{center}
\bigskip

\noindent This proves the law of non-contradiction - no proposition
can be both true and false.

\bigskip

\section*{Example 4 : Indirect Proof}
\addcontentsline{toc}{section}{Example 4 : Indirect Proof}

\noindent Prove: $(\lnot p \implies false)$ $\Rightarrow$ $\lnot$ $\lnot$ p

\bigskip

\begin{center}
  \adjustbox{max width=\textwidth}{%
    $\displaystyle
    \infer[\Rightarrow\textrm{-intro}^{[1]}]{(\lnot p \implies false)
    \implies \lnot \lnot p}{
      \infer[\lnot\textrm{-intro}^{[2]}]{\lnot \lnot
      p}{\infer[\mbox{assumption}]{\lnot p}{\infer[\Rightarrow
      \mbox{elim}]{false}{}}}
    }
    $%
  }
\end{center}
\bigskip

\noindent This is the pattern for indirect proof: if assuming $\lnot$
p leads to false, then $\lnot$ $\lnot p$ holds.

\bigskip

\section*{Example 5 : Modus Tollens via Contradiction}
\addcontentsline{toc}{section}{Example 5 : Modus Tollens via Contradiction}

\noindent Prove: $(p \implies q) \land \lnot q \implies \lnot p$

\bigskip

\begin{center}
  \adjustbox{max width=\textwidth}{%
    $\displaystyle
    \infer[\Rightarrow\textrm{-intro}^{[1]}]{((p \implies q) \land
    \lnot q) \implies \lnot p}{
      \infer[\lnot\textrm{-intro}^{[2]}]{\lnot
      p}{\infer[\mbox{assumption}]{p}{\infer[\Rightarrow
      \mbox{elim}]{q}{} & \infer[\mbox{contradiction}]{false}{}}}
    }
    $%
  }
\end{center}
\bigskip

\noindent Classic modus tollens pattern using contradiction.

\bigskip

\section*{Example 6 : Disjunction from Contradiction}
\addcontentsline{toc}{section}{Example 6 : Disjunction from Contradiction}

\noindent Prove: l$\lnot p \implies (p \implies q)$

\bigskip

\begin{center}
  \adjustbox{max width=\textwidth}{%
    $\displaystyle
    \infer[\Rightarrow\textrm{-intro}^{[1]}]{\lnot p \implies (p \implies q)}{
      \infer[\Rightarrow\textrm{-intro}^{[2]}]{(p \implies
      q)}{\infer[\mbox{false
        elim}]{q}{\infer[\mbox{contradiction}]{false}{\ulcorner p
      \urcorner^{[2]}}}}
    }
    $%
  }
\end{center}
\bigskip

\noindent From $\lnot$ p, we can prove $p \implies q$ for any q.

\bigskip

\section*{Example 7 : Double Negation Elimination}
\addcontentsline{toc}{section}{Example 7 : Double Negation Elimination}

\noindent Prove: $\lnot$ l$\lnot p \implies p(requires(classical)(logic))$

\bigskip

\begin{center}
  \adjustbox{max width=\textwidth}{%
    $\displaystyle
    \infer[\Rightarrow\textrm{-intro}^{[1]}]{\lnot \lnot p \implies
    \lnot \lnot \lnot p}{
      \infer[\lnot\textrm{-intro}^{[2]}]{\lnot \lnot \lnot
      p}{\infer[\mbox{false
        elim}]{p}{\infer[\mbox{contradiction}]{false}{\ulcorner \lnot p
      \urcorner^{[2]}}}}
    }
    $%
  }
\end{center}
\bigskip

\noindent In classical logic, not $\lnot$ p implies p. This requires
excluded middle or equivalent axiom.

\bigskip

\section*{Example 8 : Reductio ad Absurdum}
\addcontentsline{toc}{section}{Example 8 : Reductio ad Absurdum}

\noindent Prove: $(\lnot p \implies (q \land \lnot q))$ $\Rightarrow$ p

\bigskip

\begin{center}
  \adjustbox{max width=\textwidth}{%
    $\displaystyle
    \infer[\Rightarrow\textrm{-intro}^{[1]}]{(\lnot p \implies (q
    \land \lnot q)) \implies \lnot \lnot p}{
      \infer[\lnot\textrm{-intro}^{[2]}]{\lnot \lnot
      p}{\infer[\mbox{false
        elim}]{p}{\infer[\mbox{contradiction}]{false}{\infer[\land\textrm{-elim-2}]{\lnot
            q}{\infer[\land\textrm{-elim-1}]{q}{\infer[\Rightarrow
      \mbox{elim}]{q \land \lnot q}{\ulcorner \lnot p \urcorner^{[2]}}}}}}}
    }
    $%
  }
\end{center}
\bigskip

\noindent Reductio ad absurdum: if assuming $\lnot p$ leads to
absurdity, then p holds.

\bigskip

\section*{Example 9 : Contradiction with Universal Quantifier}
\addcontentsline{toc}{section}{Example 9 : Contradiction with
Universal Quantifier}

\noindent Prove: ($\forall x @ \power x$) $\land$ $\exists x @ \lnot
\power x$ is contradictory.

\bigskip

\begin{center}
  \adjustbox{max width=\textwidth}{%
    $\displaystyle
    \infer[\Rightarrow\textrm{-intro}^{[1]}]{((\forall x @ \power x)
    \land (\exists x @ \lnot \power x)) \implies false}{
      \infer[\mbox{contradiction}]{false}{\infer[\forall
        \mbox{elim}]{\power a}{\infer[\exists \mbox{elim}, \mbox{fresh
          a}]{\lnot \power a}{\infer[\land\textrm{-elim-2}]{\exists x @
            \lnot \power x}{\infer[\land\textrm{-elim-1}]{\forall x @
              \power x}{\ulcorner (\forall x @ \power x) \land (\exists x @
      \lnot \power x) \urcorner^{[1]}}}}}}
    }
    $%
  }
\end{center}
\bigskip

\noindent If something holds for all x, it cannot fail for some x.

\bigskip

\section*{Example 10 : Proving Uniqueness by Contradiction}
\addcontentsline{toc}{section}{Example 10 : Proving Uniqueness by Contradiction}

\noindent Prove: if f is injective, then f(x) = f(y) $\Rightarrow$ $x = y$.

\bigskip

\begin{center}
  \adjustbox{max width=\textwidth}{%
    $\displaystyle
    \infer[\Rightarrow\textrm{-intro}^{[1]}]{(injective(f) \land
    (f(x) = f(y))) \implies x = y}{
      \infer[\lnot\textrm{-intro}^{[2]}]{x =
      y}{\infer[\mbox{contradiction with
        f}(x)=f(y)]{false}{\infer[\mbox{injective property}]{f(x) \neq
      f(y)}{\ulcorner x \neq y \urcorner^{[2]}}}}
    }
    $%
  }
\end{center}
\bigskip

\noindent Uses contradiction to prove equality.

\bigskip

\section*{Example 11 : Case Analysis Leading to Contradiction}
\addcontentsline{toc}{section}{Example 11 : Case Analysis Leading to
Contradiction}

\noindent Prove: p $\lor$ q, $\lnot$ p, l$\lnot q \implies false$

\bigskip

\begin{center}
  \adjustbox{max width=\textwidth}{%
    $\displaystyle
    \infer[\Rightarrow\textrm{-intro}^{[1]}]{((p \lor q) \land \lnot
    p \land \lnot q) \implies false}{
      \infer[\lor \mbox{elim \mbox{from} p} \lor
      q]{false}{\infer[\mbox{contradiction with} \lnot p]{false}{} &
      \infer[\mbox{contradiction with} \lnot q]{false}{}}
    }
    $%
  }
\end{center}
\bigskip

\noindent Both cases lead to contradiction, so the premises are inconsistent.

\bigskip

\section*{Example 12 : Best Practices for Contradiction Proofs}
\addcontentsline{toc}{section}{Example 12 : Best Practices for
Contradiction Proofs}

\noindent When using proof by contradiction:

\bigskip

\noindent 1. Clearly mark the assumption you're contradicting with
[assumption N]

\bigskip

\noindent 2. Show explicitly where false is derived

\bigskip

\noindent 3. Use [not intro from N] to discharge the assumption

\bigskip

\noindent 4. Document the contradiction (what conflicts with what)

\bigskip

\noindent 5. In natural deduction, false elim lets you conclude anything

\bigskip

\noindent 6. $Remember : contradiction$ is a classical technique
$\lnot constructive$

\bigskip

\end{document}

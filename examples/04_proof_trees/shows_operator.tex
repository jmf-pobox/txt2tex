\documentclass[a4paper,10pt,fleqn]{article}
\usepackage[margin=1in]{geometry}
\usepackage{amssymb}
\usepackage{adjustbox}
\usepackage{natbib}
\usepackage[colorlinks=true,linkcolor=blue,citecolor=blue,urlcolor=blue]{hyperref}
\usepackage{fuzz}
\usepackage{zed-maths}
\usepackage{zed-proof}
\newdimen\savedleftskip
\begin{document}

\noindent The "shows" operator represents sequent judgment, written
as ⊢ (turnstile).

\bigskip

\noindent In sequent calculus, Γ ⊢ Δ means "from context Γ, we can derive Δ".

\bigskip

\noindent Simple sequent judgment:

\bigskip

\noindent
$Gamma \shows Delta$

\noindent More concrete example with specific propositions:

\bigskip

\noindent
$(\power \land Q) \shows \power$

\noindent Sequent with multiple context items (using sets):

\bigskip

\noindent
$\{\power, \power \implies Q\} \shows Q$

\noindent We can use shows in inference rules to represent natural deduction:

\bigskip

\begin{infrule}
  Gamma \shows \power & \\
  Gamma \shows (\power \implies Q) & \\
  \derive \\
  Gamma \shows Q & \mbox{modus ponens $\in$ sequent form}
\end{infrule}

\noindent Combining shows with inference rules for natural deduction style:

\bigskip

\begin{infrule}
  Gamma \shows A & \\
  Gamma \shows B & \\
  \derive \\
  Gamma \shows (A \land B) & \mbox{conjunction introduction}
\end{infrule}

\noindent The shows operator renders as \shows in LaTeX, which displays as ⊢.

\bigskip

\end{document}

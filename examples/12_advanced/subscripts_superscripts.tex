\documentclass[a4paper,10pt,fleqn]{article}
\usepackage[margin=1in]{geometry}
\usepackage{amssymb}
\usepackage{adjustbox}
\usepackage{natbib}
\usepackage[colorlinks=true,linkcolor=blue,citecolor=blue,urlcolor=blue]{hyperref}
\usepackage{fuzz}
\usepackage{zed-maths}
\usepackage{zed-proof}
\newdimen\savedleftskip
\begin{document}

\section*{Subscripts and Superscripts}

\section*{Example 1 : Basic Subscripts}
\addcontentsline{toc}{section}{Example 1 : Basic Subscripts}

\noindent Subscripts are written with underscore:

\bigskip

\noindent
$x_1 \in S$

\noindent
$x_2 = x_1 + 1$

\noindent Common for indexed variables in sequences, arrays, or iterations.

\bigskip

\section*{Example 2 : Basic Superscripts ( Exponentiation )}
\addcontentsline{toc}{section}{Example 2 : Basic Superscripts (
Exponentiation )}

\noindent Superscripts denote exponentiation:

\bigskip

\noindent
$x^2 = x * x$

\noindent
$n^3 = n * n * n$

\noindent The caret symbol (^) creates superscripts for powers.

\bigskip

\section*{Example 3 : Multiple Subscripted Variables}
\addcontentsline{toc}{section}{Example 3 : Multiple Subscripted Variables}

\begin{axdef}
  x_1 : \nat \\
  x_2 : \nat \\
  x_3 : \nat
  \where
  x_1 = 1 \\
  x_2 = 2 \\
  x_3 = 3
\end{axdef}

\noindent Define multiple indexed variables.

\bigskip

\section*{Example 4 : Sequence Indexing}
\addcontentsline{toc}{section}{Example 4 : Sequence Indexing}

\noindent Subscripts are natural for sequence element notation:

\bigskip

\noindent
$\forall i : 1 \upto n @ s\_i \in \nat$

\noindent Here s\_i represents the i-th element of a conceptual sequence.

\bigskip

\section*{Example 5 : Matrix Notation}
\addcontentsline{toc}{section}{Example 5 : Matrix Notation}

\noindent Subscripts can indicate matrix positions:

\bigskip

\noindent M\_(i,j) = row i, column j element

\bigskip

\noindent Or with multiple subscripts:

\bigskip

\noindent M\_i\_j for element at row i, column j

\bigskip

\section*{Example 6 : Exponentiation in Expressions}
\addcontentsline{toc}{section}{Example 6 : Exponentiation in Expressions}

\noindent
$\{~ n : \nat | n < 10 @ n^2 ~\}$

\noindent Set of squares: $\{0, 1, 4, 9, 16, 25, 36, 49, 64, 81\}$

\bigskip

\section*{Example 7 : Power Function}
\addcontentsline{toc}{section}{Example 7 : Power Function}

\begin{axdef}
  power : \nat \cross \nat \fun \nat
  \where
  \forall x : \nat @ power(x, 0) = 1 \\
  \forall x, n : \nat @ n > 0 \implies power(x, n) = x * power(x, n - 1)
\end{axdef}

\noindent Recursive definition of exponentiation. Equivalent to $x^n$.

\bigskip

\section*{Example 8 : Polynomials}
\addcontentsline{toc}{section}{Example 8 : Polynomials}

\noindent Subscripts for coefficients, superscripts for powers:

\bigskip

\noindent
$a_0 + a_1 * x + a_2 * x^2 + a_3 * x^3$

\noindent Standard polynomial notation.

\bigskip

\section*{Example 9 : Combined Subscripts and Superscripts}
\addcontentsline{toc}{section}{Example 9 : Combined Subscripts and Superscripts}

\noindent Variable with subscript, raised to a power:

\bigskip

\noindent
$x\_i^2$

\noindent Meaning: (x sub i) squared.

\bigskip

\noindent Power of a variable with subscript:

\bigskip

\noindent
$x^2(\_i)$

\noindent Meaning: The i-th element of the sequence of squares.

\bigskip

\section*{Example 10 : Nested Superscripts}
\addcontentsline{toc}{section}{Example 10 : Nested Superscripts}

\noindent Power of a power:

\bigskip

\noindent
${x^2}^3 = x^6$

\noindent Or with explicit parentheses:

\bigskip

\noindent
${n^2}^{(n + 1)}$

\section*{Example 11 : Cartesian Product Powers}
\addcontentsline{toc}{section}{Example 11 : Cartesian Product Powers}

\noindent Cartesian product repeated n times:

\bigskip

\noindent
$S^2 = S \cross S$

\noindent
$S^3 = S \cross S \cross S$

\noindent Set of n-tuples from S.

\bigskip

\section*{Example 12 : Iteration Subscripts}
\addcontentsline{toc}{section}{Example 12 : Iteration Subscripts}

\noindent Subscripts often denote iteration steps:

\bigskip

\noindent
$x_0 = initial(value)$

\noindent
$x\_(i + 1) = f(x\_i)$

\noindent Defines a recurrence relation.

\bigskip

\section*{Example 13 : Prime Notation Alternative}
\addcontentsline{toc}{section}{Example 13 : Prime Notation Alternative}

\noindent For derivatives or next-state, use primes or subscripts:

\bigskip

\noindent x' (prime) or x\_next

\bigskip

\noindent txt2tex supports x' notation for primed identifiers in schemas.

\bigskip

\section*{Example 14 : State Variable Subscripts}
\addcontentsline{toc}{section}{Example 14 : State Variable Subscripts}

\begin{schema}{Counter}
  count : \nat \\
  countNext : \nat
  \where
  countNext = count + 1
\end{schema}

\noindent Next-state variables can use descriptive names like
countNext instead of primed notation.

\bigskip

\section*{Example 15 : Indexed Family of Sets}
\addcontentsline{toc}{section}{Example 15 : Indexed Family of Sets}

\begin{axdef}
  S_1 : \power \nat \\
  S_2 : \power \nat \\
  S_3 : \power \nat
  \where
  S_1 = \{1, 2, 3\} \\
  S_2 = \{4, 5, 6\} \\
  S_3 = \{7, 8, 9\}
\end{axdef}

\noindent Family of sets indexed by natural numbers.

\bigskip

\section*{Example 16 : Exponents in Constraints}
\addcontentsline{toc}{section}{Example 16 : Exponents in Constraints}

\noindent
$\forall n : \nat @ n^2 \geq n$

\noindent Constraint using $exponentiation : every natural number's
square$ is at least itself.

\bigskip

\section*{Example 17 : Summation Notation ( Conceptual )}
\addcontentsline{toc}{section}{Example 17 : Summation Notation ( Conceptual )}

\noindent While txt2tex doesn't have built-in summation, subscripts
help express it:

\bigskip

\noindent $sum\_i = x_1$ + x\_2 + ... + x\_n

\bigskip

\noindent Conceptually represents summing x\_i for i from 1 to n.

\bigskip

\section*{Example 18 : Geometric Sequence}
\addcontentsline{toc}{section}{Example 18 : Geometric Sequence}

\begin{axdef}
  r : \nat \\
  a_0 : \nat
  \where
  r = 2 \\
  a_0 = 1
\end{axdef}

\noindent For a geometric sequence with first term a\_0 and ratio $r
: a_0$, a\_0 * r, a\_0 * r * r, a\_0 * r * r * r, ...

\bigskip

\section*{Example 19 : Multiple Indices}
\addcontentsline{toc}{section}{Example 19 : Multiple Indices}

\noindent Double subscripts for 2D structures:

\bigskip

\noindent M\_1\_1, M\_1\_2, M\_2\_1, M\_2\_2

\bigskip

\noindent Or with tuple notation:

\bigskip

\noindent M\_(1,1), M\_(1,2), M\_(2,1), M\_(2,2)

\bigskip

\noindent Both styles represent matrix elements.

\bigskip

\section*{Example 20 : Complex Subscript Expressions}
\addcontentsline{toc}{section}{Example 20 : Complex Subscript Expressions}

\noindent
$x\_(i + 1) + x\_(i - 1) = 2 * x\_i$

\noindent Recurrence relation with subscript expressions.

\bigskip

\section*{Example 21 : Power Set Notation}
\addcontentsline{toc}{section}{Example 21 : Power Set Notation}

\noindent Sometimes written with superscript:

\bigskip

\noindent
$\power S \lor 2^S$

\noindent Both denote the power set of S.

\bigskip

\section*{Example 22 : Factorial Approximation}
\addcontentsline{toc}{section}{Example 22 : Factorial Approximation}

\noindent Superscripts in approximations:

\bigskip

\noindent $n^n$ $>$ n!

\bigskip

\noindent For $n \geq 1$, n to the power n exceeds n factorial.

\bigskip

\section*{Example 23 : Best Practices}
\addcontentsline{toc}{section}{Example 23 : Best Practices}

\noindent Guidelines for subscripts and superscripts:

\bigskip

\noindent 1. Use subscripts for indices, versions, or families (x\_1,
x\_i, S\_k)

\bigskip

\noindent 2. Use superscripts for exponents and powers ($x^2$, $n^k$, $2^n$)

\bigskip

\noindent 3. Parenthesize complex expressions: (x+1)^2, $\lnot x$+$1^2$

\bigskip

\noindent 4. Be consistent in indexing: start at 0 or 1, not mixed

\bigskip

\noindent 5. Document what subscripts mean (time steps, indices, identifiers)

\bigskip

\noindent 6. Avoid deeply nested subscripts—hurts readability

\bigskip

\section*{Example 24 : Common Pitfalls}
\addcontentsline{toc}{section}{Example 24 : Common Pitfalls}

\noindent Operator precedence matters:

\bigskip

\noindent x + $y^2$ means x + ($y^2$), not (x+y)^2

\bigskip

\noindent 2 * $x^2$ means 2 * ($x^2$), not (2*x)^2

\bigskip

\noindent Use parentheses when in doubt: (x + y)^2 is unambiguous.

\bigskip

\section*{Example 25 : Typographical Rendering}
\addcontentsline{toc}{section}{Example 25 : Typographical Rendering}

\noindent In LaTeX output:

\bigskip

\noindent - x\_1 renders as x with subscript 1

\bigskip

\noindent - $x^2$ renders as x with superscript 2

\bigskip

\noindent - $x_i^2$ renders as x with subscript i and superscript 2

\bigskip

\noindent - ($x^2$)^3 renders with proper nesting

\bigskip

\noindent The txt2tex compiler handles the LaTeX generation automatically.

\bigskip

\end{document}

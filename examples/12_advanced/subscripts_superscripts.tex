\documentclass[a4paper,10pt,fleqn]{article}
\usepackage[margin=1in]{geometry}
\usepackage{amssymb}
\usepackage{adjustbox}
\usepackage{natbib}
\usepackage[colorlinks=true,linkcolor=blue,citecolor=blue,urlcolor=blue]{hyperref}
\usepackage{fuzz}
\usepackage{zed-maths}
\usepackage{zed-proof}
\newdimen\savedleftskip
\begin{document}

\section*{Subscripts and Superscripts}

\section*{Example 1 : Basic Subscripts}
\addcontentsline{toc}{section}{Example 1 : Basic Subscripts}

\noindent Subscripts are written with underscore:

\bigskip

\noindent
$x_1 \in S$


\noindent
$x_2 = x_1 + 1$


\noindent Common for indexed variables in sequences, arrays, or iterations.

\bigskip

\section*{Example 2 : Basic Superscripts ( Exponentiation )}
\addcontentsline{toc}{section}{Example 2 : Basic Superscripts ( Exponentiation )}

\noindent Superscripts denote exponentiation:

\bigskip

\noindent
$x \bsup 2 \esup = x * x$


\noindent
$n \bsup 3 \esup = n * n * n$


\noindent The caret symbol creates superscripts for powers.

\bigskip

\section*{Example 3 : Multiple Subscripted Variables}
\addcontentsline{toc}{section}{Example 3 : Multiple Subscripted Variables}

\begin{axdef}
x_1 : \nat \\
x_2 : \nat \\
x_3 : \nat
\where
x_1 = 1 \\
x_2 = 2 \\
x_3 = 3
\end{axdef}

\noindent Define multiple indexed variables.

\bigskip

\section*{Example 4 : Sequence Indexing}
\addcontentsline{toc}{section}{Example 4 : Sequence Indexing}

\noindent Subscripts are natural for sequence element notation:

\bigskip

\noindent
$\forall i : 1 \upto n @ s\_i \in \nat$


\noindent Here s\_i represents the i-th element of a conceptual sequence.

\bigskip

\section*{Example 5 : Exponentiation in Expressions}
\addcontentsline{toc}{section}{Example 5 : Exponentiation in Expressions}

\noindent
$\{~ n : \nat | n < 10 @ n \bsup 2 \esup ~\}$


\noindent Set of squares.

\bigskip

\section*{Example 6 : Power Function for Nested Exponents}
\addcontentsline{toc}{section}{Example 6 : Power Function for Nested Exponents}

\noindent For nested exponents like power of a power, define a power function:

\bigskip

\begin{axdef}
pow : \nat \cross \nat \fun \nat
\where
\forall x : \nat @ pow(x, 0) = 1 \\
\forall x, n : \nat @ n > 0 \implies pow(x, n) = x * pow(x, n - 1)
\end{axdef}

\noindent Then use it for nested exponentiation:

\bigskip

\noindent
$pow(pow(x, 2), 3) = pow(x, 6)$


\noindent
$pow(pow(n, 2), n + 1)$


\noindent This avoids LaTeX double superscript issues.

\bigskip

\section*{Example 7 : Polynomials}
\addcontentsline{toc}{section}{Example 7 : Polynomials}

\noindent Subscripts for coefficients, superscripts for powers:

\bigskip

\noindent
$a_0 + a_1 * x + a_2 * x \bsup 2 \esup + a_3 * x \bsup 3 \esup$


\noindent Standard polynomial notation.

\bigskip

\section*{Example 8 : Combined Subscripts and Superscripts}
\addcontentsline{toc}{section}{Example 8 : Combined Subscripts and Superscripts}

\noindent Variable with subscript, raised to a power:

\bigskip

\noindent
$x\_i \bsup 2 \esup$


\noindent $Meaning : x sub i$, squared.

\bigskip

\section*{Example 9 : Cartesian Product Powers}
\addcontentsline{toc}{section}{Example 9 : Cartesian Product Powers}

\noindent Cartesian product repeated n times:

\bigskip

\noindent
$S \bsup 2 \esup = S \cross S$


\noindent
$S \bsup 3 \esup = S \cross S \cross S$


\noindent Set of n-tuples from S.

\bigskip

\section*{Example 10 : Iteration Subscripts}
\addcontentsline{toc}{section}{Example 10 : Iteration Subscripts}

\noindent Subscripts often denote iteration steps:

\bigskip

\noindent
$x_0 = initial$


\noindent
$x\_(i + 1) = f(x\_i)$


\noindent Defines a recurrence relation.

\bigskip

\section*{Example 11 : State Variable Subscripts}
\addcontentsline{toc}{section}{Example 11 : State Variable Subscripts}

\begin{schema}{Counter}
count : \nat \\
countNext : \nat
\where
countNext = count + 1
\end{schema}

\noindent Next-state variables can use descriptive names like countNext instead of primed notation.

\bigskip

\section*{Example 12 : Indexed Family of Sets}
\addcontentsline{toc}{section}{Example 12 : Indexed Family of Sets}

\begin{axdef}
S_1 : \power \nat \\
S_2 : \power \nat \\
S_3 : \power \nat
\where
S_1 = \{1, 2, 3\} \\
S_2 = \{4, 5, 6\} \\
S_3 = \{7, 8, 9\}
\end{axdef}

\noindent Family of sets indexed by natural numbers.

\bigskip

\section*{Example 13 : Exponents in Constraints}
\addcontentsline{toc}{section}{Example 13 : Exponents in Constraints}

\noindent
$\forall n : \nat @ n \bsup 2 \esup \geq n$


\noindent Constraint using $exponentiation : every natural number's square$ is at least itself.

\bigskip

\section*{Example 14 : Complex Subscript Expressions}
\addcontentsline{toc}{section}{Example 14 : Complex Subscript Expressions}

\noindent
$x\_(i + 1) + x\_(i - 1) = 2 * x\_i$


\noindent Recurrence relation with subscript expressions.

\bigskip

\section*{Example 15 : Power Set Notation}
\addcontentsline{toc}{section}{Example 15 : Power Set Notation}

\noindent Sometimes written with superscript:

\bigskip

\noindent
$\power S \lor 2 \bsup S \esup$


\noindent Both denote the power set of S.

\bigskip

\section*{Example 16 : Best Practices}
\addcontentsline{toc}{section}{Example 16 : Best Practices}

\noindent Guidelines for subscripts and superscripts:

\bigskip

\noindent 1. Use subscripts for indices, versions, or families

\bigskip

\noindent 2. Use superscripts for exponents and powers

\bigskip

\noindent 3. For nested exponents, define a pow function

\bigskip

\noindent 4. Be consistent in indexing: start at 0 or 1, not mixed

\bigskip

\noindent 5. Document what subscripts mean

\bigskip

\end{document}
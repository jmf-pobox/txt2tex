\documentclass[a4paper,10pt,fleqn]{article}
\usepackage[margin=1in]{geometry}
\usepackage{amssymb}
\usepackage{adjustbox}
\usepackage{natbib}
\usepackage{fuzz}
\usepackage{zed-maths}
\usepackage{zed-proof}
\usepackage[colorlinks=true,linkcolor=blue,citecolor=blue,urlcolor=blue]{hyperref}
\newdimen\savedleftskip
\begin{document}

\section*{Conditional Expressions}

\section*{Example 1 : Basic If - Then - Else}
\addcontentsline{toc}{section}{Example 1 : Basic If - Then - Else}

\noindent The if-then-else expression evaluates to one of two values
based on a condition:

\bigskip

\noindent
$\IF x > 0 \THEN 1 \ELSE -1$

\noindent If x is positive, this evaluates to 1; otherwise, it evaluates to -1.

\bigskip

\section*{Example 2 : Absolute Value Function}
\addcontentsline{toc}{section}{Example 2 : Absolute Value Function}

\begin{axdef}
  abs : \num \fun \nat
  \where
  \forall x : \num @ abs(x) = \IF x \geq 0 \THEN x \ELSE -x
\end{axdef}

\noindent The absolute value function uses if-then-else to handle
positive and negative cases.

\bigskip

\section*{Example 3 : Maximum Function}
\addcontentsline{toc}{section}{Example 3 : Maximum Function}

\begin{axdef}
  maximum : \nat \cross \nat \fun \nat
  \where
  \forall a, b : \nat @ maximum(a, b) = \IF a \geq b \THEN a \ELSE b
\end{axdef}

\noindent Returns the larger of two numbers.

\bigskip

\section*{Example 4 : Sign Function}
\addcontentsline{toc}{section}{Example 4 : Sign Function}

\begin{axdef}
  sign : \num \fun \{-1, 0, 1\}
  \where
  \forall x : \num @ \\
  \quad sign(x) = \IF x > 0 \THEN 1 \ELSE \IF x = 0 \THEN 0 \ELSE -1
\end{axdef}

\noindent Nested if-then-else for three-way branching.

\bigskip

\section*{Example 5 : Conditional in Set Comprehension}
\addcontentsline{toc}{section}{Example 5 : Conditional in Set Comprehension}

\noindent
$\{ x : \nat | x < 10 @ \IF x \mod 2 = 0 \THEN x \ELSE x + 1 \}$

\noindent For each x less than 10, select x if it's even, otherwise
select x + 1.

\bigskip

\section*{Example 6 : Pattern Matching with Conditionals}
\addcontentsline{toc}{section}{Example 6 : Pattern Matching with Conditionals}

\begin{axdef}
  filterPositives : \seq \num \fun \seq \num
  \where
  filterPositives(\langle \rangle) = \langle \rangle \\
  \forall x : \num @ \forall s : \seq \num @ filterPositives(\langle
  x \rangle \cat s) = \IF x > 0 \THEN \langle x \rangle \cat
  filterPositives(s) \ELSE filterPositives(s)
\end{axdef}

\noindent Filters a sequence to include only positive elements using
conditional expressions.

\bigskip

\section*{Example 7 : Grade Assignment}
\addcontentsline{toc}{section}{Example 7 : Grade Assignment}

\begin{zed}Grade ::= GradeA | GradeB | GradeC | GradeD | GradeF
\end{zed}

\begin{axdef}
  scoreToGrade : \nat \fun Grade
  \where
  \forall score : \nat @ \\
  \quad scoreToGrade(score) = \IF score \geq 90 \THEN GradeA \ELSE
  \IF score \geq 80 \THEN GradeB \ELSE \IF score \geq 70 \THEN GradeC
  \ELSE \IF score \geq 60 \THEN GradeD \ELSE GradeF
\end{axdef}

\noindent Multi-way branching using nested if-then-else.

\bigskip

\section*{Example 8 : Piecewise Function}
\addcontentsline{toc}{section}{Example 8 : Piecewise Function}

\begin{axdef}
  piecewise : \num \fun \num
  \where
  \forall x : \num @ \\
  \quad piecewise(x) = \IF x < -5 \THEN 0 \ELSE \IF x < 0 \THEN x + 5
  \ELSE \IF x < 5 \THEN x \ELSE \IF x < 10 \THEN x - 5 \ELSE 0
\end{axdef}

\noindent A piecewise-defined function with multiple ranges.

\bigskip

\section*{Example 9 : Conditional in Lambda}
\addcontentsline{toc}{section}{Example 9 : Conditional in Lambda}

\noindent
$\lambda x : \nat @ \IF x > 100 \THEN 100 \ELSE x$

\noindent Lambda function that caps values at 100.

\bigskip

\section*{Example 10 : Conditional Set Selection}
\addcontentsline{toc}{section}{Example 10 : Conditional Set Selection}

\begin{axdef}
  evenSet : \power \nat \\
  oddSet : \power \nat
  \where
  evenSet = \{ n : \nat | n \mod 2 = 0 \} \\
  oddSet = \{ n : \nat | n \mod 2 = 1 \}
\end{axdef}

\noindent Define sets that can be selected conditionally in expressions.

\bigskip

\section*{Example 11 : Practical Example - Tax Rate Selection}
\addcontentsline{toc}{section}{Example 11 : Practical Example - Tax
Rate Selection}

\begin{axdef}
  taxRate : \nat \fun \nat
  \where
  \forall income : \nat @ \\
  \quad taxRate(income) = \IF income < 20000 \THEN 10 \ELSE \IF
  income < 50000 \THEN 15 \ELSE \IF income < 100000 \THEN 20 \ELSE 25
\end{axdef}

\noindent Progressive tax rates with multi-level conditional.

\bigskip

\section*{Example 12 : Conditional with Complex Expression}
\addcontentsline{toc}{section}{Example 12 : Conditional with Complex Expression}

\begin{axdef}
  complexFunc : \nat \fun \nat
  \where
  \forall n : \nat @ \\
  \quad complexFunc(n) = \IF n \mod 3 = 0 \THEN n * n \ELSE \IF n
  \mod 3 = 1 \THEN n + 10 \ELSE n * 2
\end{axdef}

\noindent Different computations based on remainder.

\bigskip

\section*{Example 13 : Guard Conditions in Recursion}
\addcontentsline{toc}{section}{Example 13 : Guard Conditions in Recursion}

\begin{axdef}
  factorial : \nat \fun \nat
  \where
  factorial(0) = 1 \\
  \forall n : \nat @ \\
  \quad factorial(n) = \IF n > 0 \THEN n * factorial(n - 1) \ELSE 1
\end{axdef}

\noindent Conditional ensures factorial only recurses for positive n.

\bigskip

\section*{Example 14 : Ternary - Style Expression}
\addcontentsline{toc}{section}{Example 14 : Ternary - Style Expression}

\noindent Z notation's if-then-else is similar to C's ternary operator (? :):

\bigskip

\noindent $C : x$ $>$ 0 ? 1 : -1

\bigskip

\noindent $\num : if x$ $>$ 0 then 1 else -1

\bigskip

\noindent Both evaluate to one of two values based on a condition.

\bigskip

\section*{Example 15 : Short - Circuit Evaluation Note}
\addcontentsline{toc}{section}{Example 15 : Short - Circuit Evaluation Note}

\noindent In Z notation, if-then-else expressions should be evaluated
$lazily : only$ the selected branch should be evaluated. This matters
when branches have side effects or may be undefined.

\bigskip

\noindent If a key is not in a function's domain, accessing it would
be undefined. A conditional ensures we only evaluate the function
application when it's safe.

\bigskip

\section*{Example 16 : Best Practices}
\addcontentsline{toc}{section}{Example 16 : Best Practices}

\noindent When using if-then-else:

\bigskip

\noindent 1. Keep branches simple and readable

\bigskip

\noindent 2. For many cases, consider using free types with pattern matching

\bigskip

\noindent 3. Nest carefully—deep nesting hurts readability

\bigskip

\noindent 4. Document complex conditions

\bigskip

\noindent 5. Ensure both branches have the same type

\bigskip

\noindent 6. Consider using helper predicates for complex conditions

\bigskip

\end{document}

\documentclass[fleqn]{article}
\usepackage[left=1.55in,right=1.55in,top=1in,bottom=1in]{geometry}
\usepackage{amssymb}
\usepackage{zed-cm}
\usepackage{zed-maths}
\usepackage{zed-proof}
\begin{document}

\section*{Propositional Logic}

\bigskip

N.B. Many answers leverage the grammar's order of precedence (not, and, or, $\Rightarrow$, $\Leftrightarrow$) rather than pedagogical or clarifying parenthesis.

\bigskip

\bigskip
\noindent
\textbf{Question 1}

\medskip

(a)
\par
\vspace{11pt}
\begin{tabular}{c|c|c|c}
$p$ & $\lnot p$ & $( \lnot p \Rightarrow p )$ & $\mathbf{p \Rightarrow ( \lnot p \Rightarrow p )}$ \\
\hline
\textit{t} & \textit{f} & \textit{t} & \textbf{t} \\
\textit{f} & \textit{t} & \textit{f} & \textbf{t} \\
\end{tabular}

\bigskip

$p \Rightarrow (\lnot p \Rightarrow p)$ is a tautology as it is true for every interpretation of p as illustrated by the final column of truth values.

\bigskip

\medskip

(b)
\par
\vspace{11pt}
\begin{tabular}{c|c|c|c|c|c|c}
$p$ & $q$ & $( q \Rightarrow p )$ & $\lnot p$ & $\lnot q$ & $( \lnot p \Rightarrow \lnot q )$ & $\mathbf{( \lnot p \Rightarrow \lnot q ) \Leftrightarrow ( q \Rightarrow p )}$ \\
\hline
\textit{t} & \textit{t} & \textit{t} & \textit{f} & \textit{f} & \textit{t} & \textbf{t} \\
\textit{t} & \textit{f} & \textit{t} & \textit{f} & \textit{t} & \textit{t} & \textbf{t} \\
\textit{f} & \textit{t} & \textit{f} & \textit{t} & \textit{f} & \textit{f} & \textbf{t} \\
\textit{f} & \textit{f} & \textit{t} & \textit{t} & \textit{t} & \textit{t} & \textbf{t} \\
\end{tabular}

\bigskip

$\lnot p \Rightarrow \lnot q \Leftrightarrow q \Rightarrow p$ is a tautology as it is true for every interpretation of p, q as illustrated by the final column of truth values.

\bigskip

\medskip

(c)
\par
\vspace{11pt}
\begin{tabular}{c|c|c|c|c|c|c|c|c}
$p$ & $q$ & $r$ & $( p \Rightarrow r )$ & $( q \Rightarrow r )$ & $( ( p \Rightarrow r ) \land ( q \Rightarrow r ) )$ & $( p \lor q )$ & $( ( p \lor q ) \Rightarrow r )$ & $\mathbf{\Leftrightarrow}$ \\
\hline
\textit{t} & \textit{t} & \textit{t} & \textit{t} & \textit{t} & \textit{t} & \textit{t} & \textit{t} & \textbf{t} \\
\textit{t} & \textit{t} & \textit{f} & \textit{f} & \textit{f} & \textit{f} & \textit{t} & \textit{f} & \textbf{t} \\
\textit{t} & \textit{f} & \textit{t} & \textit{t} & \textit{t} & \textit{t} & \textit{t} & \textit{t} & \textbf{t} \\
\textit{t} & \textit{f} & \textit{f} & \textit{f} & \textit{t} & \textit{f} & \textit{t} & \textit{f} & \textbf{t} \\
\textit{f} & \textit{t} & \textit{t} & \textit{t} & \textit{t} & \textit{t} & \textit{t} & \textit{t} & \textbf{t} \\
\textit{f} & \textit{t} & \textit{f} & \textit{t} & \textit{f} & \textit{f} & \textit{t} & \textit{f} & \textbf{t} \\
\textit{f} & \textit{f} & \textit{t} & \textit{t} & \textit{t} & \textit{t} & \textit{f} & \textit{t} & \textbf{t} \\
\textit{f} & \textit{f} & \textit{f} & \textit{t} & \textit{t} & \textit{t} & \textit{f} & \textit{t} & \textbf{t} \\
\end{tabular}

\bigskip

(($p \Rightarrow r$) and ($q \Rightarrow r$)) $\Leftrightarrow$ ((p or q) $\Rightarrow$ r) is a tautology as it is true for every interpretation of p, q, r as illustrated by the final column of truth values.

\bigskip

\medskip

\bigskip
\noindent
\textbf{Question 2}

\medskip

(a)
\par
\vspace{11pt}
\[
\begin{array}{ll@{\hspace{2em}}l}
& p \Rightarrow (\lnot p \Rightarrow p) \\
&\Leftrightarrow p \Rightarrow \lnot \lnot p \lor p & [\mbox{$\Rightarrow$ disjunction}] \\
&\Leftrightarrow p \Rightarrow p \lor p & [\mbox{$\lnot$ $\lnot$}] \\
&\Leftrightarrow p \Rightarrow p & [\mbox{idempotence of p}] \\
&\Leftrightarrow true & [\mbox{tautology}]
\end{array}
\]

\bigskip

In this equivalence proof, we show that $p \Rightarrow (\lnot p \Rightarrow p)$ by using implication disjunction to replace the implication with a logical or, the definition of double negation to simplify, the application of the idempotence rule to simplify, and the fact that $p \Rightarrow p$ is a tautology. 

\bigskip

\medskip

(b)
\par
\vspace{11pt}
\[
\begin{array}{ll@{\hspace{2em}}l}
& \lnot p \Rightarrow \lnot q \\
&\Leftrightarrow \lnot \lnot p \lor \lnot q & [\mbox{$\Rightarrow$ disjunction}] \\
&\Leftrightarrow p \lor \lnot q & [\mbox{$\lnot$ $\lnot$}] \\
&\Leftrightarrow \lnot q \lor p & [\mbox{commutativity of $\lor$}] \\
&\Leftrightarrow q \Rightarrow p & [\mbox{$\Rightarrow$ disjunction}]
\end{array}
\]

\bigskip

In this equivalance proof, we show that $\lnot p \Rightarrow \lnot q \Leftrightarrow q \Rightarrow p$ by using implication disjunction to replace implication with logical or, the definition of double negation to simplify, the commutivity property of logical or to change the order of the elements, and the implication disjunction rule to reintroduce implication.

\bigskip

\medskip

(c)
\par
\vspace{11pt}
\[
\begin{array}{ll@{\hspace{2em}}l}
& (p \Rightarrow r) \land (q \Rightarrow r) \\
&\Leftrightarrow (\lnot p \lor r) \land (\lnot q \lor r) & [\mbox{$\Rightarrow$ disjunction}] \\
&\Leftrightarrow (r \lor \lnot p) \land (r \lor \lnot q) & [\mbox{commutativity of $\lor$}] \\
&\Leftrightarrow r \lor \lnot p \land \lnot q & [\mbox{distributivity 2}] \\
&\Leftrightarrow \lnot p \land \lnot q \lor r & [\mbox{commutativity of $\lor$}] \\
&\Leftrightarrow \lnot (p \lor q) \lor r & [\mbox{De Morgan}] \\
&\Leftrightarrow p \lor q \Rightarrow r & [\mbox{$\Rightarrow$ disjunction}]
\end{array}
\]

\bigskip

In this equivalence proof, we show that (($p \Rightarrow r$) and ($q \Rightarrow r$)) $\Leftrightarrow$ (p or q) $\Rightarrow$ r by using the implication disjunction rule to replace implication with logical or, the commutativity of or rule to change the order of the elements, the distributivity 2 rule to extract r, and the commutivity of logical or to change the order of the elements, De Morgan to extract negation for ($\lnot p$ and $\lnot q$), and the implication disjunction rule to reintroduce implication.

\bigskip

\medskip

\bigskip
\noindent
\textbf{Question 3}

\medskip

(a)
\par
\vspace{11pt}
\noindent
$\displaystyle
\infer[$\Rightarrow$\textrm{-intro}^{[1]}]{p \Rightarrow (\lnot p \Rightarrow p)}{
  \infer[$\Rightarrow$\textrm{-intro}^{[2]}]{\lnot p \Rightarrow p}{\ulcorner \lnot p \urcorner^{[2]} & \ulcorner p \urcorner^{[1]}}
}
$

\bigskip

In this proof, we assume p, assume $\lnot p$, use the implication introduction rule to arrive at not $p \Rightarrow p$ and to discharge $\lnot p$ as an assumption, use the implication introduction rule to arrive at p $\Rightarrow$ (not $p \Rightarrow p$) and to discharge the p assumption.

\bigskip

\medskip

(b)
nd{document}

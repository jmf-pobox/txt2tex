\documentclass[a4paper,10pt,fleqn]{article}
\usepackage[margin=1in]{geometry}
\usepackage{amssymb}
\usepackage{natbib}
\usepackage{fuzz}
\usepackage{zed-maths}
\usepackage{zed-proof}
\begin{document}

\section*{Software Engineering Mathematics}

\bigskip

October 2025

\bigskip

\bigskip

Answers use the order of precedence ($\lnot$, $\land$, $\lor$, $\Rightarrow$, $\Leftrightarrow$) \citep[Slide 20]{simpson25a}.

\bigskip

\bigskip
\noindent
\textbf{Question 1}

\medskip

\bigskip

Lecture 1 - Propositions

\bigskip

(a)
\par
\vspace{11pt}
\begin{tabular}{c|c|c|c}
$p$ & $\lnot p$ & $( \lnot p \Rightarrow p )$ & $\mathbf{p \Rightarrow ( \lnot p \Rightarrow p )}$ \\
\hline
\textit{t} & \textit{f} & \textit{t} & \textbf{t} \\
\textit{f} & \textit{t} & \textit{f} & \textbf{t} \\
\end{tabular}

\bigskip

$p \implies (\lnot p \implies p)$ is a tautology as it is true for every interpretation of p as illustrated by the final column of truth values.

\bigskip

\medskip

(b)
\par
\vspace{11pt}
\begin{tabular}{c|c|c|c|c|c|c}
$p$ & $q$ & $( q \Rightarrow p )$ & $\lnot p$ & $\lnot q$ & $( \lnot p \Rightarrow \lnot q )$ & $\mathbf{( \lnot p \Rightarrow \lnot q ) \Leftrightarrow ( q \Rightarrow p )}$ \\
\hline
\textit{t} & \textit{t} & \textit{t} & \textit{f} & \textit{f} & \textit{t} & \textbf{t} \\
\textit{t} & \textit{f} & \textit{t} & \textit{f} & \textit{t} & \textit{t} & \textbf{t} \\
\textit{f} & \textit{t} & \textit{f} & \textit{t} & \textit{f} & \textit{f} & \textbf{t} \\
\textit{f} & \textit{f} & \textit{t} & \textit{t} & \textit{t} & \textit{t} & \textbf{t} \\
\end{tabular}

\bigskip

The statements $\lnot p \implies \lnot q$, $q \implies p$ are equivalent and $\lnot p \implies \lnot q \Leftrightarrow q \implies p$ is a tautology as it is true for every interpretation of p, q as illustrated by the final column of truth values.

\bigskip

\medskip

(c)
\par
\vspace{11pt}
\begin{tabular}{c|c|c|c|c|c|c|c|c}
$p$ & $q$ & $r$ & $( p \Rightarrow r )$ & $( q \Rightarrow r )$ & $( ( p \Rightarrow r ) \land ( q \Rightarrow r ) )$ & $( p \lor q )$ & $( ( p \lor q ) \Rightarrow r )$ & $\mathbf{\Leftrightarrow}$ \\
\hline
\textit{t} & \textit{t} & \textit{t} & \textit{t} & \textit{t} & \textit{t} & \textit{t} & \textit{t} & \textbf{t} \\
\textit{t} & \textit{t} & \textit{f} & \textit{f} & \textit{f} & \textit{f} & \textit{t} & \textit{f} & \textbf{t} \\
\textit{t} & \textit{f} & \textit{t} & \textit{t} & \textit{t} & \textit{t} & \textit{t} & \textit{t} & \textbf{t} \\
\textit{t} & \textit{f} & \textit{f} & \textit{f} & \textit{t} & \textit{f} & \textit{t} & \textit{f} & \textbf{t} \\
\textit{f} & \textit{t} & \textit{t} & \textit{t} & \textit{t} & \textit{t} & \textit{t} & \textit{t} & \textbf{t} \\
\textit{f} & \textit{t} & \textit{f} & \textit{t} & \textit{f} & \textit{f} & \textit{t} & \textit{f} & \textbf{t} \\
\textit{f} & \textit{f} & \textit{t} & \textit{t} & \textit{t} & \textit{t} & \textit{f} & \textit{t} & \textbf{t} \\
\textit{f} & \textit{f} & \textit{f} & \textit{t} & \textit{t} & \textit{t} & \textit{f} & \textit{t} & \textbf{t} \\
\end{tabular}

\bigskip

The statements $(p \implies r) \land (q \implies r)$, $p \lor q \implies r$ are equivalent and $(p \implies r) \land (q \implies r)$ $\Leftrightarrow$ $p \lor q \implies r$ is a tautology as it is true for every interpretation of p, q, r as illustrated by the final column of truth values.

\bigskip

\newpage

\medskip

\bigskip
\noindent
\textbf{Question 2}

\medskip

\bigskip

Lecture 1 - Propositions

\bigskip

(a)
\par
\vspace{11pt}
\[
\begin{array}{ll@{\hspace{2em}}l}
& p \implies (\lnot p \implies p) \\
&\Leftrightarrow p \implies \lnot \lnot p \lor p & [\mbox{$\Rightarrow$ disjunction}] \\
&\Leftrightarrow p \implies p \lor p & [\mbox{$\lnot$ $\lnot$}] \\
&\Leftrightarrow p \implies p & [\mbox{idempotence of p}] \\
&\Leftrightarrow true & [\mbox{tautology}]
\end{array}
\]

\bigskip

In this equivalence proof, we show that $p \implies (\lnot p \implies p)$ by using implication disjunction to replace the implication with a logical or, the definition of double negation to simplify, the application of the idempotence rule to simplify, and the fact that $p \implies p$ is a tautology. 

\bigskip

\medskip

(b)
\par
\vspace{11pt}
\[
\begin{array}{ll@{\hspace{2em}}l}
& \lnot p \implies \lnot q \\
&\Leftrightarrow \lnot \lnot p \lor \lnot q & [\mbox{$\Rightarrow$ disjunction}] \\
&\Leftrightarrow p \lor \lnot q & [\mbox{$\lnot$ $\lnot$}] \\
&\Leftrightarrow \lnot q \lor p & [\mbox{commutativity of $\lor$}] \\
&\Leftrightarrow q \implies p & [\mbox{$\Rightarrow$ disjunction}]
\end{array}
\]

\bigskip

In this equivalance proof, we show that $\lnot p \implies \lnot q \Leftrightarrow q \implies p$ by using implication disjunction to replace implication with logical or, the definition of double negation to simplify, the commutivity property of logical or to change the order of the elements, and the implication disjunction rule to reintroduce implication.

\bigskip

\medskip

(c)
\par
\vspace{11pt}
\[
\begin{array}{ll@{\hspace{2em}}l}
& (p \implies r) \land (q \implies r) \\
&\Leftrightarrow (\lnot p \lor r) \land (\lnot q \lor r) & [\mbox{$\Rightarrow$ disjunction}] \\
&\Leftrightarrow (r \lor \lnot p) \land (r \lor \lnot q) & [\mbox{commutativity of $\lor$}] \\
&\Leftrightarrow r \lor \lnot p \land \lnot q & [\mbox{distributivity 2}] \\
&\Leftrightarrow \lnot p \land \lnot q \lor r & [\mbox{commutativity of $\lor$}] \\
&\Leftrightarrow \lnot (p \lor q) \lor r & [\mbox{De Morgan}] \\
&\Leftrightarrow p \lor q \implies r & [\mbox{$\Rightarrow$ disjunction}]
\end{array}
\]

\bigskip

In this equivalence proof, we show that $(p \implies r) \land (q \implies r)$ $\Leftrightarrow$ $p \lor q$ $\Rightarrow$ r by using the implication disjunction rule to replace implication with logical or, the commutativity of or rule to change the order of the elements, the distributivity 2 rule to extract r, and the commutivity of logical or to change the order of the elements, De Morgan to extract negation for $\lnot p \land \lnot q$, and the implication disjunction rule to reintroduce implication.

\bigskip

\newpage

\medskip

\bigskip
\noindent
\textbf{Question 3}

\medskip

\bigskip

Lecture 3 - Deductive Proofs

\bigskip

(a)
\par
\vspace{11pt}
\noindent
$\displaystyle
\infer[\Rightarrow\textrm{-intro}^{[1]}]{p \implies (\lnot p \implies p)}{
  \ulcorner p \urcorner^{[1]}
&
\infer[\Rightarrow\textrm{-intro}^{[2]}]{\lnot p \implies p}{\ulcorner p \urcorner^{[1]} & \infer[false\textrm{-elim}^{[3]}]{\lnot p}{\infer[\mathrm{false} - \mathrm{intro}]{false}{\ulcorner \lnot p \urcorner^{[2]} & \ulcorner p \urcorner^{[3]}}}}
}
$

\bigskip

For this proof we are seeking to prove that p $\Rightarrow$ $\lnot p \implies p$, which we know holds given the truth table in 1(a). Working mechanically from the bottom, I started by applying the implication introduction rule with assumption [1] to gain our conclusion, implication introduction rule with assumption [2] to gain $\lnot p \implies p$.  This statement requires that I have both p and $\lnot p$ as premisses, which is a bit odd and implied I needed to use false in the proof.  Since I had p as assumption [1], I knew I wanted $\lnot p$ as the output of the false rules.  It took some experimentation, but eventually I realized I needed three assumptions not two and it was obvious that I wanted to preserve $\lnot p$ as assumption [2], which meant I needed p as assumption [3] and that assumption [3] would be discharged by the false-elim rule.  In the final proof, all assumptions are discharged properly. The key insight was to have p as an assumption twice at different scopes.

\bigskip

\medskip

(b)
\par
\vspace{11pt}
\bigskip

This proof involves equivalence or bi-implication, which requires us to show $p \implies q$ and $q \implies p$ given this is the definition of $p \Leftrightarrow q$.

\bigskip

\bigskip

In one direction, we are seeking to prove $\lnot p \implies \lnot q$ $\Rightarrow$ $q \implies p$:

\bigskip

\noindent
$\displaystyle
\infer[\Rightarrow\textrm{-intro}^{[1]}]{(\lnot p \implies \lnot q) \implies (q \implies p)}{
  \ulcorner \lnot p \implies \lnot q \urcorner^{[1]}
&
\infer[\Rightarrow\textrm{-intro}^{[2]}]{q \implies p}{
  \ulcorner q \urcorner^{[2]}
&
\infer[false\textrm{-elim}^{[3]}]{p}{
  \infer[\mathrm{false} - \mathrm{intro}]{false}{
  \infer[\Rightarrow \mathrm{elim}]{\lnot q}{
  \ulcorner \lnot p \implies \lnot q \urcorner^{[1]}
&
\ulcorner \lnot p \urcorner^{[3]}
}
&
\ulcorner q \urcorner^{[2]}
}
}
}
}
$

\bigskip

I started working mechanically from the bottom by applying the implication introduction rule to construct the conclusion.  This gave me $q \implies p$ as the key premise to build up, which requires applying the implication introduction rule to q and p. Looking at the need for q and looking at the need for p and $\lnot p \implies \lnot q$ as assumption [1], I felt it would be best to try $\lnot p$ as an assumption to at least get going.  This enabled me to work top down to get $\lnot q$ from the implication elimination rule. From there, I realized that I needed the negation of both $\lnot q$ and $\lnot p$. I knew the false-intro and false-elim were required, but it took some experimentation to see that I could get both $\lnot q$ and $\lnot p$ transformed into p and q by making $\lnot p$ assumption [3] instead of [2] and introducing q as assumption [2]. Once that insight was made, I could arrive at p as a result of the top of the tree and retain q [2] as an assumption coming out of false-elim. This is what I needed to arrive at $q \implies p$ and to connect the bottom and top portions of my proof.  All assumptions are properly discharged.

\bigskip

\bigskip

and in the other, we are seeking to prove $q \implies p$ $\Rightarrow$ $\lnot p \implies \lnot q$:

\bigskip

\noindent
$\displaystyle
\infer[\Rightarrow\textrm{-intro}^{[1]}]{(q \implies p) \implies (\lnot p \implies \lnot q)}{
  \ulcorner q \implies p \urcorner^{[1]}
&
\infer[\Rightarrow\textrm{-intro}^{[2]}]{\lnot p \implies \lnot q}{
  \ulcorner \lnot p \urcorner^{[2]}
&
\infer[false\textrm{-elim}^{[3]}]{\lnot q}{
  \infer[\mathrm{false} - \mathrm{intro}]{false}{
  \infer[\Rightarrow \mathrm{elim}]{p}{
  \ulcorner q \implies p \urcorner^{[1]}
&
\ulcorner q \urcorner^{[3]}
}
&
\ulcorner \lnot p \urcorner^{[2]}
}
}
}
}
$

\bigskip

Working mechanically from the bottom, I apply the implication rule to $q \implies p$ as assumption [1] to gain the conclusion.  I need $\lnot p \implies \lnot q$, which shows me to use the implication rule again with $\lnot p$ as assumption [2].  From there, the question is how to get $\lnot q$.  Now working from the top, given assumption [1] has q, it makes sense to use the implication elimation rule with q as assumption [3]. This gives me p, when what I want is $\lnot q$.  I see that if I introduce $\lnot p$ as assumption [3] that I can negate q via the false-intro and false-elim rules and successfully discharge assumption [3]. This gives me $\lnot q$, which is what I need to connect to the top tree to the bottom of the proof. This proof came more quickly because $q \implies p$ as assumption [1] gave clear direction for the top portion. All assumptions are discharged and scopes are appropriate.

\bigskip

\bigskip

We then combine these two proofs with the $\Leftrightarrow$-intro rule, using the results from above:

\bigskip

\noindent
$\displaystyle
\infer[\Leftrightarrow \mathrm{intro}]{\lnot p \implies \lnot q \Leftrightarrow q \implies p}{
  (\lnot p \implies \lnot q) \implies (q \implies p)
&
(q \implies p) \implies (\lnot p \implies \lnot q)
}
$

\medskip

(c)
\par
\vspace{11pt}
\bigskip

In one direction, we are seeking to prove $p \implies r$ and $q \implies r$ $\Rightarrow$ $p \lor q \implies r$:

\bigskip

\noindent
$\displaystyle
\infer[\Rightarrow\textrm{-intro}^{[1]}]{(p \implies r) \land (q \implies r) \implies (p \lor q \implies r)}{
  \ulcorner (p \implies r) \land (q \implies r) \urcorner^{[1]}
&
\infer[\Rightarrow\textrm{-intro}^{[2]}]{p \lor q \implies r}{
  \ulcorner p \lor q \urcorner^{[2]}
&
\infer[\lor\textrm{-elim}^{[2]}]{r}{
  \ulcorner p \lor q \urcorner^{[2]}
&
\raiseproof{12ex}{\infer[\Rightarrow \mathrm{elim}]{r}{
  \ulcorner p \urcorner^{[2]}
&
\infer[\land\textrm{-elim-1}]{p \implies r}{
  \ulcorner (p \implies r) \land (q \implies r) \urcorner^{[1]}
}
}}
&
\hskip 6em \raiseproof{30ex}{\infer[\Rightarrow \mathrm{elim}]{r}{
  \ulcorner q \urcorner^{[2]}
&
\infer[\land\textrm{-elim-2}]{q \implies r}{
  \ulcorner (p \implies r) \land (q \implies r) \urcorner^{[1]}
}
}}
}
}
}
$

\bigskip

In this proof, I start by working from the bottom mechanically, taking $(p \implies r) \land (q \implies r)$ as assumption [1] and applying the implication introduction rule to arrive at our conclusion and discharge assumption [1]. From there, I can see that I need $p \lor q$ and r as inputs to the implication introduction rule.  I take $p \lor q$ as assumption [2] and discharge this assumption via this rule.  That leaves me with r as the result of my bottom up process.  As I have already made assumptions $(p \implies r) \land (q \implies r)$ [1] and $p \lor q$ [2] and they are related, I start by working with both top-down. I begin with $p \lor q$ [2] and create cases for p and q working towards or elimination.  The p and q sub-branches are marked with [2] as the assumption as these are just the cases for $p \lor q$ [2] \citep[slide 41]{simpson25c}. First, I apply and elimination in each case to get $p \implies r$, $q \implies r$ with p, q respectively in the sub-branches.  From there, I apply implication elimination to arrive at r in both cases.  This enables me to use the or elimation together with $p \lor q$ [2] to simplify to r.  At this stage, I do not discharge assumption [2], as this is deferred until the next step. I use implication introduction together with $p \lor q$ [2] and r from the or elimination to discharge assumption [2], which connects the top-down and bottom-up process. All assumptions are properly discharged.  

\bigskip

\bigskip

In the other direction, we are seeking to prove $p \lor q \implies r$ $\Rightarrow$ $(p \implies r) \land (q \implies r)$:

\bigskip

\noindent
$\displaystyle
\infer[\Rightarrow\textrm{-intro}^{[1]}]{(p \lor q \implies r) \implies (p \implies r) \land (q \implies r)}{
  \ulcorner p \lor q \implies r \urcorner^{[1]}
&
\infer[\land \mathrm{intro}]{(p \implies r) \land (q \implies r)}{
  \infer[\Rightarrow\textrm{-intro}^{[2]}]{p \implies r}{
  \ulcorner p \urcorner^{[2]}
&
\infer[\Rightarrow \mathrm{elim}]{r}{
  \ulcorner p \lor q \implies r \urcorner^{[1]}
&
\infer[\lor\textrm{-intro-1}]{p \lor q}{
  \ulcorner p \urcorner^{[2]}
}
}
}
&
\infer[\Rightarrow\textrm{-intro}^{[3]}]{q \implies r}{
  \ulcorner q \urcorner^{[3]}
&
\infer[\Rightarrow \mathrm{elim}]{r}{
  \ulcorner p \lor q \implies r \urcorner^{[1]}
&
\infer[\lor\textrm{-intro-2}]{p \lor q}{
  \ulcorner q \urcorner^{[3]}
}
}
}
}
}
$

\bigskip

Working mechanically from the bottom, we start by taking $p \lor q \implies r$ as assumption [1] and applying the implication introduction rule to discharge assumption [1].  This leaves us with $p \implies r$ and $q \implies r$, which is derived from $p \implies r$, $q \implies r$ and the application of the and introduction rule. Fairly clearly, we need two sub-branches to be developed top-down. Since we have $p \lor q \implies r$ and we want $p \implies r$ and $q \implies r$, I start with p [2] and q [3] to start the top-down sub-trees.  I struggled here until I recalled that I could use the or intro rule to satisfy $p \lor q$.  I also had to recall that this does not generate additional assumptions to discharge \citep[slide 38]{simpson25c}. From here, it was not difficult to use implication elimation to arrive at r in both sub-trees and to apply p [2] and q [3] in each tree respectively with the implication introduction rule to discharge those assumptions.  This gave me $p \implies r$, $q \implies r$ and I used the and introduction rule to connect the top and bottom portions of the proof. All assumptions are properly discharged.  

\bigskip

\bigskip

We then combine these two proofs with the $\Leftrightarrow$-intro rule, using the results from above:

\bigskip

\noindent
$\displaystyle
\infer[\Leftrightarrow \mathrm{intro}]{(p \implies r) \land (q \implies r) \Leftrightarrow p \lor q \implies r}{
  (p \implies r) \land (q \implies r) \implies (p \lor q \implies r)
&
(p \lor q \implies r) \implies (p \implies r) \land (q \implies r)
}
$

\medskip

\bigskip
\noindent
\textbf{Question 4}

\medskip

\bigskip

We are defining a generic function that will tell us the difference in length between the longest and shorest sequences from an input type that is a non-empty sequences of sequences. The input could be $\langle  \langle x, y, z \rangle, \langle \rangle  \rangle$ yielding 3 as output or $\langle \langle x \rangle \rangle$ yielding 0, for example. I specify three generic functions: maxSeq yields the longest sequence from a non-empty sequence of sequences; minSeq yields the shortest sequence from a non-empty sequence of sequences; and maxDiff, which fulfills the specification for Question 4.

\bigskip

\begin{gendef}[X]
  maxSeq: \seq_1 (\seq X) \fun \seq X \\
  minSeq: \seq_1 (\seq X) \fun \seq X \\
  maxDiff: \seq_1 (\seq X) \fun \nat
\where
  \forall ss : \seq_1 (\seq X) @ maxSeq(ss) = (\mu s : \ran ss | \forall t : \ran ss @ \# s \geq \# t) \\
  \forall ss : \seq_1 (\seq X) @ minSeq(ss) = (\mu s : \ran ss | \forall t : \ran ss @ \# s \leq \# t) \\
  \forall ss : \seq_1 (\seq X) @ maxDiff(ss) = \# (maxSeq(ss)) - \# (minSeq(ss))
\end{gendef}

\bigskip

For maxSeq and minSeq, a mu function is defined that selects a single element meeting the specification based on comparing each sequence s with every sequence t. First, the range operator moves us from a sequence of sequences with ss values such as $\langle  1  \mapsto  \langle x, y, z \rangle, 2  \mapsto  \langle \rangle  \rangle$ to sequences s, t with values like $\langle x, y, z \rangle$, $\langle \rangle$ to enable seq operators. maxSeq and minSeq use the length operators and maxDiff uses the length operators with subtraction. For maxDiff, $maxSeq \geq minDiff$ is a valid pre-condition by virtue of maxSeq and minSeq and therefore the subtraction is guaranteed to return a natural number as defined in the specification. This solution passes fuzz validation.

\bigskip

\bigskip
\noindent
\textbf{Question 5}

\medskip

\bigskip

The relational composition operator is associative:

\bigskip


\hspace*{1em} $(R \circ S) \circ T = R \circ (S \circ T)$

\bigskip

if $R \in W$ $\rel$ X, $S \in X$ $\rel$ Y, and $T \in Y$ $\rel$ Z.

\bigskip

\bigskip

In this equivalence proof, we show that composition is associative using the definition of relational composition and by repositioning quantifiers and predicates where we are permitted to do so, which is only when there is no impact to the truthity of the predicates involved. Each such repositioning or regrouping will be justified. 

\bigskip

\bigskip

Given a relation $R \in X$ $\rel$ Y and a second relation $S \in Y$ $\rel$ Z, and two elements $x \in X$ and $z \in Z$, x $\mapsto$ $z \in R$ $\circ$ S $\Leftrightarrow$ $\exists y : Y @ x \mapsto y \in R \land y \mapsto z \in S$ \citep[page 107]{simpson-notes}.

\bigskip

\[
\begin{array}{ll@{\hspace{2em}}l}
& w \mapsto z \in (R \circ S) \circ T \\
&\Leftrightarrow \exists y : Y @ w \mapsto y \in (R \circ S) \land y \mapsto z \in T & [\mbox{definition of $\circ$}] \\
&\Leftrightarrow \exists y : Y @ (\exists x : X @ w \mapsto x \in R \land x \mapsto y \in S) \land y \mapsto z \in T & [\mbox{definition of $\circ$}] \\
&\Leftrightarrow \exists y : Y @ \exists x : X @ w \mapsto x \in R \land x \mapsto y \in S \land y \mapsto z \in T & [\mbox{x is $\lnot$ free $\in$ y $\mapsto$ z $\in$ T}] \\
&\Leftrightarrow \exists x : X @ w \mapsto x \in R \land (\exists y : Y @ x \mapsto y \in S \land y \mapsto z \in T) & [\mbox{y is $\lnot$ free $\in$ w $\mapsto$ x $\in$ R}] \\
&\Leftrightarrow \exists x : X @ w \mapsto x \in R \land x \mapsto z \in (S \circ T) & [\mbox{definition of $\circ$}] \\
&\Leftrightarrow w \mapsto z \in R \circ (S \circ T)
\end{array}
\]

\bigskip

In step 1, we apply the composition equivalence to the outer composition of (R $\circ$ S) $\circ$ T. In step 2, we apply our equivalence to the inner composition (R $\circ$ S).  In step 3, we include the predicate y $\mapsto$ $z \in T$ in the scope of x : X. We can do this because this does not affect the truthity of y $\mapsto$ $z \in T$ as x is not free in this predicate. In step 4, we reposition the exists y : Y quantifier to be after the w $\mapsto$ $x \in R$ predicate without affecting its truthity because y is not free in w $\mapsto$ $x \in R$.  In step 5, we apply the equivalence for composition to S $\circ$ T.  In step 6, we apply the equivalence for composition to R $\circ$ (S $\circ$ T).

\bigskip

\bigskip

I worked on this proof both top-down and bottom-up.  After doing the obvious equivalence substitutions, I was left with step 3 and 4 to work out. In working out step 3, I could see from step 6 that I wanted S $\circ$ T together.  In working out step 4, I could see that y : Y needed to be repositioned to setup step 6.  The reasoning about both of these steps was not obvious at first, but I realized that these moves had no impact on the truth values of the affected predicates.  This led me to think about free and bound variables and what rules might apply in various cases.  I was not able to find a specific justification in the materials that maps to the idea that one can reposition quantifiers or predicates provided there is no inpact to the truthity, but this seems logical and sound provided x is not $free \in P$ for the affected P.

\bigskip

\newpage

\bigskip
\noindent
\textbf{Question 6}

\medskip

\bigskip

Lecture 4 - Sets, Types, and Definitions

\bigskip

(a)
\par
\vspace{11pt}
\begin{axdef}
SquaresOfEvens : \power \num
\where
SquaresOfEvens = \{ z : \num | z \mod 2 = 0 @ z * z \}
\end{axdef}

\medskip

(b)
\par
\vspace{11pt}
\begin{axdef}
CubesOfSquaresOfEvens : \power \num
\where
CubesOfSquaresOfEvens = \{ z : \num | z \mod 2 = 0 @ z * z * (z * z) * (z * z) \}
\end{axdef}

\medskip

(c)
\par
\vspace{11pt}
\begin{axdef}
SquareCubePairsOfEvens : \power (\num \cross \num)
\where
SquareCubePairsOfEvens = \{ z_1, z_2 : \num | z_1 \mod 2 = 0 \land z_2 \mod 2 = 0 @ (z_1 * z_1, z_2 * z_2 * z_2) \}
\end{axdef}

\medskip

(d)
\par
\vspace{11pt}
\begin{axdef}
SquareCubeIdentityPairsOfEvens : \power (\num \cross \num)
\where
SquareCubeIdentityPairsOfEvens = \{ z : \num | z \mod 2 = 0 @ (z * z, z * z * z) \}
\end{axdef}

\medskip

(e)
\par
\vspace{11pt}
\begin{axdef}
SquareOrCubeIntegers : \power \num
\where
SquareOrCubeIntegers = \{ z, z_0 : \num | z = z_0 * z_0 \lor z = z_0 * z_0 * z_0 @ z \}
\end{axdef}

\medskip

\bigskip
\noindent
\textbf{Question 7}

\medskip

\begin{zed}[ShowId, PeopleId]\end{zed}

\begin{zed}
EpisodeId == \nat_1
\end{zed}

\begin{zed}
Timestamp == \nat
\end{zed}

\begin{gendef}[X, Y]
  fst: X \cross Y \fun X
\where
  \forall a : X @ \forall b : Y @ fst(a, b) = a
\end{gendef}

\begin{gendef}[X, Y]
  snd: X \cross Y \fun Y
\where
  \forall a : X @ \forall b : Y @ snd(a, b) = b
\end{gendef}

\bigskip

This needs to be simplified $using <=> and$ by using dom(function) to declare the types.

\bigskip

\begin{schema}{PodcastPlatform}
shows : \finset ShowId \\
show\_episodes : ShowId \pfun \finset EpisodeId \\
episode\_info : ShowId \cross EpisodeId \pfun \finset PeopleId \cross Timestamp \\
show\_info : ShowId \pfun \finset PeopleId
\where
shows \subseteq \dom show\_episodes \land \dom show\_episodes \subseteq shows \\
\forall es : \ran show\_episodes @ \exists n : \nat @ es = 1 \upto n \\
\forall s : ShowId @ \forall e : EpisodeId @ s \in \dom show\_episodes \land e \in show\_episodes(s) \implies (s, e) \in \dom episode\_info \\
\forall s : ShowId @ \forall e : EpisodeId @ (s, e) \in \dom episode\_info \implies s \in \dom show\_episodes \land e \in show\_episodes(s) \\
shows \subseteq \dom show\_info \land \dom show\_info \subseteq shows \\
\forall s : ShowId @ \forall p : PeopleId @ \exists e : EpisodeId @ (s, e) \in \dom episode\_info \land \\
& \quad p \in fst(episode\_info(s, e)) \implies p \in show\_info(s)
\end{schema}

\newpage

\section*{Bibliography}


\begin{thebibliography}{Simpson, n.d.}





\bibitem[Simpson, n.d.]{simpson-notes} Simpson, A. (n.d.). \textit{From Discrete Mathematics to State-Based Models, SEM version}. University of Oxford Department of Computer Science. Unpublished course notes.





\bibitem[Simpson, 2002]{simpson02} Simpson, A.C. (2002). \textit{Discrete Mathematics by Example}. London: McGraw-Hill.





\bibitem[Simpson, 2025a]{simpson25a} Simpson, A. (2025a). \textit{Introduction and propositions}. Lecture slides for Software Engineering Mathematics. University of Oxford Department of Computer Science. Lecture 01.





\bibitem[Simpson, 2025b]{simpson25b} Simpson, A. (2025b). \textit{Predicate logic and equality}. Lecture slides for Software Engineering Mathematics. University of Oxford Department of Computer Science. Lecture 02.





\bibitem[Simpson, 2025c]{simpson25c} Simpson, A. (2025c). \textit{Deductive proofs}. Lecture slides for Software Engineering Mathematics. University of Oxford Department of Computer Science. Lecture 03.





\bibitem[Simpson, 2025d]{simpson25d} Simpson, A. (2025d). \textit{Sets, types and definitions}. Lecture slides for Software Engineering Mathematics. University of Oxford Department of Computer Science. Lecture 04.





\bibitem[Simpson, 2025e]{simpson25e} Simpson, A. (2025e). \textit{Relations}. Lecture slides for Software Engineering Mathematics. University of Oxford Department of Computer Science. Lecture 05.





\bibitem[Simpson, 2025f]{simpson25f} Simpson, A. (2025f). \textit{Functions and sequences}. Lecture slides for Software Engineering Mathematics. University of Oxford Department of Computer Science. Lecture 06.





\bibitem[Simpson, 2025g]{simpson25g} Simpson, A. (2025g). \textit{Free types and induction}. Lecture slides for Software Engineering Mathematics. University of Oxford Department of Computer Science. Lecture 07.





\bibitem[Spivey, 1992]{spivey92} Spivey, J.M. (1992). \textit{The Z Notation: A Reference Manual}. Upper Saddle River, NJ: Prentice Hall.





\bibitem[Woodcock and Davies, 1996]{woodcock96} Woodcock, J. and Davies, J. (1996). \textit{Using Z: Specification, Refinement and Proof}. Upper Saddle River, NJ: Prentice Hall.





\end{thebibliography}

\end{document}
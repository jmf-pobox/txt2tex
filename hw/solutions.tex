\documentclass[fleqn]{article}
\usepackage[left=1.55in,right=1.55in,top=1in,bottom=1in]{geometry}
\usepackage{zed-cm}
\usepackage{zed-maths}
\usepackage{zed-proof}
\usepackage{amsmath}
\usepackage{amssymb}
\begin{document}

\section*{Propositional Logic}

\bigskip
\noindent
\textbf{Question 1}

\medskip

(a)
\par
\vspace{11pt}
\begin{tabular}{c|c|c|c}
$p$ & $\lnot p$ & $( \lnot p \Rightarrow p )$ & $\mathbf{p \Rightarrow ( \lnot p \Rightarrow p )}$ \\
\hline
t & f & t & \textbf{t} \\
f & t & f & \textbf{t} \\
\end{tabular}

\bigskip

p $\Rightarrow$ (not $p \Rightarrow p$) is a tautology as it is true for every interpretation.

\bigskip

\medskip

(b)
\par
\vspace{11pt}
\begin{tabular}{c|c|c|c|c|c|c}
$p$ & $q$ & $( q \Rightarrow p )$ & $\lnot p$ & $\lnot q$ & $( \lnot p \Rightarrow \lnot q )$ & $\mathbf{( \lnot p \Rightarrow \lnot q ) \Leftrightarrow ( q \Rightarrow p )}$ \\
\hline
t & t & t & f & f & t & \textbf{t} \\
t & f & t & f & t & t & \textbf{t} \\
f & t & f & t & f & f & \textbf{t} \\
f & f & t & t & t & t & \textbf{t} \\
\end{tabular}

\bigskip

(not $p $\Rightarrow$  not$ q) $\Leftrightarrow$ ($q \Rightarrow p$) is a tautology as it is true for every interpretation of its constituent parts. 

\bigskip

\medskip

(c)
\par
\vspace{11pt}
\begin{tabular}{c|c|c|c|c|c|c|c|c}
$p$ & $q$ & $r$ & $( p \Rightarrow r )$ & $( q \Rightarrow r )$ & $( ( p \Rightarrow r ) \land ( q \Rightarrow r ) )$ & $( p \lor q )$ & $( ( p \lor q ) \Rightarrow r )$ & $\mathbf{\Leftrightarrow}$ \\
\hline
t & t & t & t & t & t & t & t & \textbf{t} \\
t & t & f & f & f & f & t & f & \textbf{t} \\
t & f & t & t & t & t & t & t & \textbf{t} \\
t & f & f & f & t & f & t & f & \textbf{t} \\
f & t & t & t & t & t & t & t & \textbf{t} \\
f & t & f & t & f & f & t & f & \textbf{t} \\
f & f & t & t & t & t & f & t & \textbf{t} \\
f & f & f & t & t & t & f & t & \textbf{t} \\
\end{tabular}

\bigskip

(($p \Rightarrow r$) and ($q \Rightarrow r$)) $\Leftrightarrow$ ((p or q) $\Rightarrow$ r) is a tautology as it is true for every interpretation of its constituent parts.

\bigskip

\medskip

\bigskip
\noindent
\textbf{Question 2}

\medskip

(a)
\par
\vspace{11pt}
\vspace{-10pt}
\begin{align*}
& p \Rightarrow (\lnot p \Rightarrow p) \\
&\Leftrightarrow p \Rightarrow \lnot \lnot p \lor p & \hfill \text{[$\Rightarrow$ disjunction]} \\
&\Leftrightarrow p \Rightarrow p \lor p & \hfill \text{[$\lnot$ $\lnot$]} \\
&\Leftrightarrow p \Rightarrow p & \hfill \text{[idempotence of p]} \\
&\Leftrightarrow true & \hfill \text{[tautology]}
\end{align*}

\bigskip

In this equivalence proof, we show that p $\Rightarrow$ (not $p \Rightarrow p$) by using implication disjunction, the definition of double negation, the application of the idempotence rule and the definition of true.

\bigskip

\medskip

(b)
\par
\vspace{11pt}
\vspace{-10pt}
\begin{align*}
& \lnot p \Rightarrow \lnot q \\
&\Leftrightarrow \lnot \lnot p \lor \lnot q & \hfill \text{[$\Rightarrow$ disjunction]} \\
&\Leftrightarrow p \lor \lnot q & \hfill \text{[$\lnot$ $\lnot$]} \\
&\Leftrightarrow \lnot q \lor p & \hfill \text{[commutativity of $\lor$]} \\
&\Leftrightarrow q \Rightarrow p & \hfill \text{[$\Rightarrow$ disjunction]}
\end{align*}

\medskip

(c)
\par
\vspace{11pt}
\vspace{-10pt}
\begin{align*}
& (p \Rightarrow r) \land (q \Rightarrow r) \\
&\Leftrightarrow (\lnot p \lor r) \land (\lnot q \lor r) & \hfill \text{[$\Rightarrow$ disjunction]} \\
&\Leftrightarrow r \land (\lnot p \lor \lnot q) & \hfill \text{[distributivity 2]} \\
&\Leftrightarrow (\lnot p \lor \lnot q) \land r & \hfill \text{[associativity of $\lor$]} \\
&\Leftrightarrow p \lor q \Rightarrow r
\end{align*}

\medskip

\bigskip
\noindent
\textbf{Question 3}

\medskip

(a)
\par
\vspace{11pt}
\noindent
$\displaystyle
\infer[$\Rightarrow$\text{-intro}^{[1]}]{p \Rightarrow (\lnot p \Rightarrow p)}{
  \infer[$\Rightarrow$\text{-intro}^{[2]}]{\lnot p \Rightarrow p}{\ulcorner \lnot p \urcorner^{[2]} & \ulcorner p \urcorner^{[1]}}
}
$

\medskip

(b)
\par
\vspace{11pt}
\bigskip

In one direction:

\bigskip

\noindent
$\displaystyle
\infer[$\Rightarrow$\text{-intro}^{[1]}]{(\lnot p \Rightarrow \lnot q) \Rightarrow (q \Rightarrow p)}{
  \ulcorner \lnot p \Rightarrow \lnot q \urcorner^{[1]} & \infer[$\Rightarrow$\text{-intro}^{[2]}]{q \Rightarrow p}{
  \ulcorner q \urcorner^{[2]} & \infer[false\text{-elim}^{[3]}]{p}{
  \infer[\text{false intro}]{false}{
  \infer[\text{$\Rightarrow$ elim}]{\lnot q}{
  \ulcorner \lnot p \Rightarrow \lnot q \urcorner^{[1]} & \ulcorner \lnot p \urcorner^{[3]}
} & \ulcorner q \urcorner^{[2]}
}
}
}
}
$

\bigskip

and the other:

\bigskip

\noindent
$\displaystyle
\infer[$\Rightarrow$\text{-intro}^{[1]}]{(q \Rightarrow p) \Rightarrow (\lnot p \Rightarrow \lnot q)}{
  \ulcorner q \Rightarrow p \urcorner^{[1]} & \infer[$\Rightarrow$\text{-intro}^{[2]}]{\lnot p \Rightarrow \lnot q}{
  \ulcorner \lnot p \urcorner^{[2]} & \infer[false\text{-elim}^{[3]}]{\lnot q}{
  \infer[\text{false intro}]{false}{
  \infer[\text{$\Rightarrow$ elim}]{p}{
  \ulcorner q \Rightarrow p \urcorner^{[1]} & \ulcorner q \urcorner^{[3]}
} & \ulcorner \lnot p \urcorner^{[2]}
}
}
}
}
$

\bigskip

We then combine these two proofs with the $\Leftrightarrow$-intro rule, using the results from above as premises.

\bigskip

\noindent
$\displaystyle
\infer[\text{$\Leftrightarrow$ intro}]{\lnot p \Rightarrow \lnot q \Leftrightarrow q \Rightarrow p}{
  (\lnot p \Rightarrow \lnot q) \Rightarrow (q \Rightarrow p) & (q \Rightarrow p) \Rightarrow (\lnot p \Rightarrow \lnot q)
}
$

\medskip

(c)
\par
\vspace{11pt}
\bigskip

In one direction:

\bigskip

\noindent
$\displaystyle
\infer[$\Rightarrow$\text{-intro}^{[1]}]{(p \Rightarrow r) \land (q \Rightarrow r) \Rightarrow (p \lor q \Rightarrow r)}{
  \ulcorner (p \Rightarrow r) \land (q \Rightarrow r) \urcorner^{[1]} & \infer[$\Rightarrow$\text{-intro}^{[2]}]{p \lor q \Rightarrow r}{
  \ulcorner p \lor q \urcorner^{[2]} & \infer[$\lor$\text{-elim}^{[3]}]{r}{
  \raiseproof{10ex}{\infer[$\land$\text{-elim-1}]{p \Rightarrow r}{
  \ulcorner (p \Rightarrow r) \land (q \Rightarrow r) \urcorner^{[1]}
} & \infer[\text{$\Rightarrow$ elim}]{r}{
  p \Rightarrow r & \ulcorner p \urcorner^{[3]}
}} & \hskip 6em \raiseproof{26ex}{\infer[$\land$\text{-elim-2}]{q \Rightarrow r}{
  \ulcorner (p \Rightarrow r) \land (q \Rightarrow r) \urcorner^{[1]}
} & \infer[\text{$\Rightarrow$ elim}]{r}{
  q \Rightarrow r & \ulcorner q \urcorner^{[3]}
}}
}
}
}
$

\medskip

\end{document}
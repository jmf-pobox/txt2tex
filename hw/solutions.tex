\documentclass[fleqn]{article}
\usepackage[left=1.55in,right=1.55in,top=1in,bottom=1in]{geometry}
\usepackage{fuzz}
\usepackage{zed-maths}
\usepackage{zed-proof}
\usepackage{amsmath}
\usepackage{amssymb}
\begin{document}

\section*{Propositional Logic}

\bigskip

N.B. Many answers leverage the grammar's order of precedence (not, and, or, $\Rightarrow$, $\Leftrightarrow$) rather than pedagogical or clarifying parenthesis.

\bigskip

\bigskip
\noindent
\textbf{Question 1}

\medskip

(a)
\par
\vspace{11pt}
\begin{tabular}{c|c|c|c}
$p$ & $\lnot p$ & $( \lnot p \Rightarrow p )$ & $\mathbf{p \Rightarrow ( \lnot p \Rightarrow p )}$ \\
\hline
\textit{t} & \textit{f} & \textit{t} & \textbf{t} \\
\textit{f} & \textit{t} & \textit{f} & \textbf{t} \\
\end{tabular}

\bigskip

$p \Rightarrow (\lnot p \Rightarrow p)$ is a tautology as it is true for every interpretation of p as illustrated by the final column of truth values.

\bigskip

\medskip

(b)
\par
\vspace{11pt}
\begin{tabular}{c|c|c|c|c|c|c}
$p$ & $q$ & $( q \Rightarrow p )$ & $\lnot p$ & $\lnot q$ & $( \lnot p \Rightarrow \lnot q )$ & $\mathbf{( \lnot p \Rightarrow \lnot q ) \Leftrightarrow ( q \Rightarrow p )}$ \\
\hline
\textit{t} & \textit{t} & \textit{t} & \textit{f} & \textit{f} & \textit{t} & \textbf{t} \\
\textit{t} & \textit{f} & \textit{t} & \textit{f} & \textit{t} & \textit{t} & \textbf{t} \\
\textit{f} & \textit{t} & \textit{f} & \textit{t} & \textit{f} & \textit{f} & \textbf{t} \\
\textit{f} & \textit{f} & \textit{t} & \textit{t} & \textit{t} & \textit{t} & \textbf{t} \\
\end{tabular}

\bigskip

$\lnot p \Rightarrow \lnot q \Leftrightarrow q \Rightarrow p$ is a tautology as it is true for every interpretation of p, q as illustrated by the final column of truth values.

\bigskip

\medskip

(c)
\par
\vspace{11pt}
\begin{tabular}{c|c|c|c|c|c|c|c|c}
$p$ & $q$ & $r$ & $( p \Rightarrow r )$ & $( q \Rightarrow r )$ & $( ( p \Rightarrow r ) \land ( q \Rightarrow r ) )$ & $( p \lor q )$ & $( ( p \lor q ) \Rightarrow r )$ & $\mathbf{\Leftrightarrow}$ \\
\hline
\textit{t} & \textit{t} & \textit{t} & \textit{t} & \textit{t} & \textit{t} & \textit{t} & \textit{t} & \textbf{t} \\
\textit{t} & \textit{t} & \textit{f} & \textit{f} & \textit{f} & \textit{f} & \textit{t} & \textit{f} & \textbf{t} \\
\textit{t} & \textit{f} & \textit{t} & \textit{t} & \textit{t} & \textit{t} & \textit{t} & \textit{t} & \textbf{t} \\
\textit{t} & \textit{f} & \textit{f} & \textit{f} & \textit{t} & \textit{f} & \textit{t} & \textit{f} & \textbf{t} \\
\textit{f} & \textit{t} & \textit{t} & \textit{t} & \textit{t} & \textit{t} & \textit{t} & \textit{t} & \textbf{t} \\
\textit{f} & \textit{t} & \textit{f} & \textit{t} & \textit{f} & \textit{f} & \textit{t} & \textit{f} & \textbf{t} \\
\textit{f} & \textit{f} & \textit{t} & \textit{t} & \textit{t} & \textit{t} & \textit{f} & \textit{t} & \textbf{t} \\
\textit{f} & \textit{f} & \textit{f} & \textit{t} & \textit{t} & \textit{t} & \textit{f} & \textit{t} & \textbf{t} \\
\end{tabular}

\bigskip

(($p \Rightarrow r$) and ($q \Rightarrow r$)) $\Leftrightarrow$ ((p or q) $\Rightarrow$ r) is a tautology as it is true for every interpretation of p, q, r as illustrated by the final column of truth values.

\bigskip

\medskip

\bigskip
\noindent
\textbf{Question 2}

\medskip

(a)
\par
\vspace{11pt}
\vspace{-10pt}
\begin{align*}
& p \Rightarrow (\lnot p \Rightarrow p) \\
&\Leftrightarrow p \Rightarrow \lnot \lnot p \lor p & \hfill \text{[$\Rightarrow$ disjunction]} \\
&\Leftrightarrow p \Rightarrow p \lor p & \hfill \text{[$\lnot$ $\lnot$]} \\
&\Leftrightarrow p \Rightarrow p & \hfill \text{[idempotence of p]} \\
&\Leftrightarrow true & \hfill \text{[tautology]}
\end{align*}

\bigskip

In this equivalence proof, we show that $p \Rightarrow (\lnot p \Rightarrow p)$ by using implication disjunction to replace the implication with a logical or, the definition of double negation to simplify, the application of the idempotence rule to simplify, and the fact that $p \Rightarrow p$ is a tautology. 

\bigskip

\medskip

(b)
\par
\vspace{11pt}
\vspace{-10pt}
\begin{align*}
& \lnot p \Rightarrow \lnot q \\
&\Leftrightarrow \lnot \lnot p \lor \lnot q & \hfill \text{[$\Rightarrow$ disjunction]} \\
&\Leftrightarrow p \lor \lnot q & \hfill \text{[$\lnot$ $\lnot$]} \\
&\Leftrightarrow \lnot q \lor p & \hfill \text{[commutativity of $\lor$]} \\
&\Leftrightarrow q \Rightarrow p & \hfill \text{[$\Rightarrow$ disjunction]}
\end{align*}

\bigskip

In this equivalance proof, we show that $\lnot p \Rightarrow \lnot q \Leftrightarrow q \Rightarrow p$ by using implication disjunction to replace implication with logical or, the definition of double negation to simplify, the commutivity property of logical or to change the order of the elements, and the implication disjunction rule to reintroduce implication.

\bigskip

\medskip

(c)
\par
\vspace{11pt}
\vspace{-10pt}
\begin{align*}
& (p \Rightarrow r) \land (q \Rightarrow r) \\
&\Leftrightarrow (\lnot p \lor r) \land (\lnot q \lor r) & \hfill \text{[$\Rightarrow$ disjunction]} \\
&\Leftrightarrow (r \lor \lnot p) \land (r \lor \lnot q) & \hfill \text{[commutativity of $\lor$]} \\
&\Leftrightarrow r \lor \lnot p \land \lnot q & \hfill \text{[distributivity 2]} \\
&\Leftrightarrow \lnot p \land \lnot q \lor r & \hfill \text{[commutativity of $\lor$]} \\
&\Leftrightarrow \lnot (p \lor q) \lor r & \hfill \text{[De Morgan]} \\
&\Leftrightarrow p \lor q \Rightarrow r & \hfill \text{[$\Rightarrow$ disjunction]}
\end{align*}

\bigskip

In this equivalence proof, we show that (($p \Rightarrow r$) and ($q \Rightarrow r$)) $\Leftrightarrow$ (p or q) $\Rightarrow$ r by using the implication disjunction rule to replace implication with logical or, the commutativity of or rule to change the order of the elements, the distributivity 2 rule to extract r, and the commutivity of logical or to change the order of the elements, De Morgan to extract negation for ($\lnot p$ and $\lnot q$), and the implication disjunction rule to reintroduce implication.

\bigskip

\medskip

\bigskip
\noindent
\textbf{Question 3}

\medskip

(a)
\par
\vspace{11pt}
\noindent
$\displaystyle
\infer[$\Rightarrow$\text{-intro}^{[1]}]{p \Rightarrow (\lnot p \Rightarrow p)}{
  \infer[$\Rightarrow$\text{-intro}^{[2]}]{\lnot p \Rightarrow p}{\ulcorner \lnot p \urcorner^{[2]} & \ulcorner p \urcorner^{[1]}}
}
$

\bigskip

In this proof, we assume p, assume $\lnot p$, use the implication introduction rule to arrive at not $p \Rightarrow p$ and to discharge $\lnot p$ as an assumption, use the implication introduction rule to arrive at p $\Rightarrow$ (not $p \Rightarrow p$) and to discharge the p assumption.

\bigskip

\medskip

(b)
\par
\vspace{11pt}
\bigskip

This proof involves equivalence or bi-implication, which requires us to show $p \Rightarrow q$ and $q \Rightarrow p$ given this is the definition of $p \Leftrightarrow q$.

\bigskip

\bigskip

In one direction, we are seeking to prove (not $p => not$ q) $\Rightarrow$ $q \Rightarrow p$:

\bigskip

\noindent
$\displaystyle
\infer[$\Rightarrow$\text{-intro}^{[1]}]{(\lnot p \Rightarrow \lnot q) \Rightarrow (q \Rightarrow p)}{
  \ulcorner \lnot p \Rightarrow \lnot q \urcorner^{[1]} & \infer[$\Rightarrow$\text{-intro}^{[2]}]{q \Rightarrow p}{
  \ulcorner q \urcorner^{[2]} & \infer[false\text{-elim}^{[3]}]{p}{
  \infer[\text{false intro}]{false}{
  \infer[\text{$\Rightarrow$ elim}]{\lnot q}{
  \ulcorner \lnot p \Rightarrow \lnot q \urcorner^{[1]} & \ulcorner \lnot p \urcorner^{[3]}
} & \ulcorner q \urcorner^{[2]}
}
}
}
}
$

\bigskip

and in the other, we are seeking to prove ($q \Rightarrow p$) $\Rightarrow$ (not $p => not$ q):

\bigskip

\noindent
$\displaystyle
\infer[$\Rightarrow$\text{-intro}^{[1]}]{(q \Rightarrow p) \Rightarrow (\lnot p \Rightarrow \lnot q)}{
  \ulcorner q \Rightarrow p \urcorner^{[1]} & \infer[$\Rightarrow$\text{-intro}^{[2]}]{\lnot p \Rightarrow \lnot q}{
  \ulcorner \lnot p \urcorner^{[2]} & \infer[false\text{-elim}^{[3]}]{\lnot q}{
  \infer[\text{false intro}]{false}{
  \infer[\text{$\Rightarrow$ elim}]{p}{
  \ulcorner q \Rightarrow p \urcorner^{[1]} & \ulcorner q \urcorner^{[3]}
} & \ulcorner \lnot p \urcorner^{[2]}
}
}
}
}
$

\bigskip

We then combine these two proofs with the $\Leftrightarrow$-intro rule, using the results from above:

\bigskip

\noindent
$\displaystyle
\infer[\text{$\Leftrightarrow$ intro}]{\lnot p \Rightarrow \lnot q \Leftrightarrow q \Rightarrow p}{
  (\lnot p \Rightarrow \lnot q) \Rightarrow (q \Rightarrow p) & (q \Rightarrow p) \Rightarrow (\lnot p \Rightarrow \lnot q)
}
$

\medskip

(c)
\par
\vspace{11pt}
\bigskip

In one direction:

\bigskip

\noindent
$\displaystyle
\infer[$\Rightarrow$\text{-intro}^{[1]}]{(p \Rightarrow r) \land (q \Rightarrow r) \Rightarrow (p \lor q \Rightarrow r)}{
  \ulcorner (p \Rightarrow r) \land (q \Rightarrow r) \urcorner^{[1]} & \infer[$\Rightarrow$\text{-intro}^{[2]}]{p \lor q \Rightarrow r}{
  \ulcorner p \lor q \urcorner^{[2]} & \infer[$\lor$\text{-elim}^{[3]}]{r}{
  \raiseproof{10ex}{\infer[$\land$\text{-elim-1}]{p \Rightarrow r}{
  \ulcorner (p \Rightarrow r) \land (q \Rightarrow r) \urcorner^{[1]}
} & \infer[\text{$\Rightarrow$ elim}]{r}{
  p \Rightarrow r & \ulcorner p \urcorner^{[3]}
}} & \hskip 6em \raiseproof{26ex}{\infer[$\land$\text{-elim-2}]{q \Rightarrow r}{
  \ulcorner (p \Rightarrow r) \land (q \Rightarrow r) \urcorner^{[1]}
} & \infer[\text{$\Rightarrow$ elim}]{r}{
  q \Rightarrow r & \ulcorner q \urcorner^{[3]}
}}
}
}
}
$

\newpage

\medskip

\section*{Bibliography}

\bigskip

Course Notes

\bigskip

\bigskip

Simpson A n.d. From Discrete Mathematics to State-Based Models SEM version. University of Oxford Department of Computer Science. Unpublished course notes.

\bigskip

\bigskip

Primary Textbook

\bigskip

\bigskip

Woodcock J. and Davies J. 1996 Using Z Specification Refinement and Proof. Upper Saddle River NJ Prentice Hall.

\bigskip

\bigskip

Lecture Slides

\bigskip

\bigskip

Simpson A 2025a Lecture schedule. Lecture slides for Software Engineering Mathematics. University of Oxford Department of Computer Science. Lecture 00.

\bigskip

\bigskip

Simpson A 2025b Introduction and propositions. Lecture slides for Software Engineering Mathematics. University of Oxford Department of Computer Science. Lecture 01.

\bigskip

\bigskip

Simpson A 2025c Predicate logic and equality. Lecture slides for Software Engineering Mathematics. University of Oxford Department of Computer Science. Lecture 02.

\bigskip

\bigskip

Simpson A 2025d Deductive proofs. Lecture slides for Software Engineering Mathematics. University of Oxford Department of Computer Science. Lecture 03.

\bigskip

\bigskip

Simpson A 2025e Sets types and definitions. Lecture slides for Software Engineering Mathematics. University of Oxford Department of Computer Science. Lecture 04.

\bigskip

\bigskip

Simpson A 2025f Relations. Lecture slides for Software Engineering Mathematics. University of Oxford Department of Computer Science. Lecture 05.

\bigskip

\bigskip

Simpson A 2025g Functions and sequences. Lecture slides for Software Engineering Mathematics. University of Oxford Department of Computer Science. Lecture 06.

\bigskip

\bigskip

Simpson A 2025h Free types and induction. Lecture slides for Software Engineering Mathematics. University of Oxford Department of Computer Science. Lecture 07.

\bigskip

\bigskip

Course Materials

\bigskip

\bigskip

Simpson A 2025i Exercises for Software Engineering Mathematics. University of Oxford Department of Computer Science. Unpublished course materials.

\bigskip

\bigskip

Simpson A 2025j Solutions for Software Engineering Mathematics. University of Oxford Department of Computer Science. Unpublished course materials.

\bigskip

\bigskip

Additional References from Course Materials

\bigskip

\bigskip

Abrial J-R 1996 The B-Book Assigning Meanings to Programs. Cambridge Cambridge University Press.

\bigskip

\bigskip

Simpson A.C. 2002 Discrete Mathematics by Example. London McGraw-Hill.

\bigskip

\bigskip

Spivey J.M. 1992 The Z Notation A Reference Manual. Upper Saddle River NJ Prentice Hall.

\bigskip

\end{document}